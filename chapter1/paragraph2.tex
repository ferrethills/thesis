\subsection{Человек -- Потребитель -- Знак}

Понятие \emph{человек} сегодня используется сразу нескольким
направлениям гуманитарных наук. Каждая из гуманитарных дисциплин,
используя свой категориальный аппарат, строит свою модель \emph{человека},
стремясь объяснить типичное в поведении индивидов, и тем самым
предвидеть его. При этом \emph{человек} очерченный дисциплинарными
рамками, скажем, психологии -- не тот, что у историков, а социологический
\emph{человек} отличается от философского \emph{человека}. Различаются человек
\emph{экономический} и человек \emph{политический}, человек \emph{потребитель} и
человек \emph{созидатель} и т.д. В данном параграфе мы используем термин
\emph{человек} сначала в широком культурно-историческом смысле, человек как
производитель знаков. Затем в узком социально\hyp{}психологическом смысле --
как потребитель знаков. Или иначе: человек сознательный и человек массы.

\subsubsection{Человек как производитель знаков}
\label{2.1}
В социальной жизни человека знак выполняет главным образом две
функции:
\begin{enumerate*}[label=\asbuk*)]
    \item приобщения к социокультурной традиции в обществе и
    \item регулирования психической и поведенческой активности индивида в
данном обществе.
\end{enumerate*}
С точки зрения современного научного знания обе
функции исторически и эволюционно возникли ориентировочно 160-200
тысяч лет тому назад, собственно, в период появления вида современного
человека разумного, homo sapiens\autocite{pagel2012wired}. Прогрессивная
разумность нового человека появилась в первую очередь в
конструировании самобытного механизма адаптации, называемого в
зарубежной научной литературе <<социальное обучение>>
(social learning)\autocites{ormrod1999human}{miller2010theories}{online:sociallearning}.
Антропологически социальное обучение -- это манифестация
нашей способности приобретать в процессе эволюции новые полезные
модели поведения, наблюдая и подражая другим, что, в исторической
перспективе, ускорило рост нашего вида по траектории, именуемой
<<кумулятивной культурной эволюцией>>. Кумулятивная культурная
эволюция подразумевает последовательную разработку, создание и
совершенствование эффективных практик, способов и приемов
жизнедеятельности, орудий и технологий производства, распространения,
коммуникации и проч. Новая форма адаптации, базирующаяся на
совместном образе жизни, кооперации и взаимовыгодном сотрудничестве,
потребовала немедленного решения насущной задачи, а именно,
предотвращения <<визуальных краж>> (visual theft), кражи лучших идей и
технологий у более талантливых или способных индивидов или общностей,
не прилагая собственных усилий и не предлагая эквивалентного обмена\autocite{pagel2012wired}.
Знак, или шире~-- знаково-символические системы, согласно этой
гипотезе, и стали таким решением. Появляется язык: как средство общения,
как средство аккумуляции и трансляции культурной традиции, как
инструмент формирования социальной идентичности, как инструмент
мышления и регулятор психической жизни, как отдельного индивида, так и
общности в целом.

Слово как знак, бесспорно, имеет особый статус в понимании жизни
человека в культуре. Во-первых, слово, по М.М. Бахтину, является чистым знаком. Во-вторых,
оно является нейтральным знаком. Весь остальной знаковый материал, как
отмечает Бахтин, необходимо распадается по обособленным областям
культурного творчества\autocite[][18]{voloshinov1993}. Более того,
Бахтин также полагает, что <<знак может возникнуть лишь на
межиндивидуальной территории (\ldots) между двумя homo sapiens знак не
возникает. Необходимо, чтобы два индивида были социально
организованы, -- составляли коллектив: лишь тогда между ними может
образоваться знаковая среда>>\autocite[][17]{voloshinov1993}. Соответственно, знаки могут
возникать лишь в процессе взаимодействия между социально
организованными индивидуальными сознаниями. При этом само
индивидуальное сознание как таковое наполнено знаками, которые, в
сущности, и составляют его главное когнитивное содержание. Сознание
становится собственно сознанием, по Бахтину, только наполняясь
культурным, знаковым содержанием, оформляясь словесно, в процессе
социального взаимодействия. Таким образом, проблема знака неизбежно
переформулируется как проблема сознания, поскольку действительность
внутренней жизни есть всегда действительность знака. Во всяком случае,
так она разрабатывается в отечественной психологии в работах Л.С.~Выготского,
А.Н.~Леонтьева, А.Р.~Лурии, Д.А.~Эльконина, Л.И.~Божович,
П.Я.~Гальперина и др., где вопрос знаковой природы сознания является
альфой и омегой психологической теории. Сделаем краткий обзор.

В 1925 г. Л.С.~Выготский публикует статью <<Сознание как проблема
психологии поведения>>, в которой он развивает мысль о том, центральным
для психологии вопросом является вопрос о природе сознания, о его
системном и смысловом строении, через исследование строения различных
психических процессов. Суть подхода Выготского и его последователей
можно обобщить так:
\begin{enumerate*}[label=\asbuk*)]
\item ключ к индивидуальному сознанию человека
  находится в его образе жизни, его сознание находится здесь, в предметном
  мире;
\item жизнь рождает психику, отражение, а жизнь человека формирует
  его сознание;
\item понимание морфологии сознания определяется
  пониманием морфологией деятельности;
\item психические функции
  находятся в межфункциональных отношениях, возникающие в ходе их
  развития, изменяющиеся и перестраивающиеся в социогенезе.
\end{enumerate*}
Восприятие, например, прежде всего, связано с памятью, затем с мышлением. Память --
прежде с аффектом, затем с образом восприятия и далее с мышлением и
понятием. Так формируются первичные связи. Но возникают также и
вторичные и третичные связи. В результате, создается иерархическая
система функций. Следует подчеркнуть, что у человека эти связи носят
особый характер. Их особенность заключается в том, что они замыкаются
<<искусственно>> или <<культурно>>, по мере того как человек овладевает
знаками, опосредующими процесс отражения в сознании. Овладение знаком
приводит функции в новое соотношение. Хрестоматийный пример: узелок,
завязанный на память, не просто узелок на платке, это узелок конкретных
психических процессов, ведущих к припоминанию. Итак, главное здесь:
сознание характеризуется системностью психических процессов. Еще
важнее: оно имеет смысловое строение. Обратимся теперь к знаковой
природе сознания.

Психологически знак есть то, что опосредует процесс отражения в
сознании. Другими словами, сознание есть опосредованное отражение
объективного мира. Бытие не просто отражается в знаке, но преломляется в
нем, делая его потенциально многоакцентным, гибким и изменчивым в
соответствии с той или иной социальной ситуацией. Знак опосредует
сознание через свое значение -- выработанное в процессе общественной
практики обобщение. Тем самым, знак -- это обобщенное отражение
социокультурной, общественно-исторической практики, объективно
закрепленное как конкретный исторический факт человеческого знания. И
если классическая форма знака -- слово, то в некотором смысле язык и есть
подлинное сознание. Язык, будучи деятельностью, моделирует мир, но
одновременно он моделирует и самого носителя этого языка, как мы
пытались показать в предыдущем разделе.

Таким образом, образуется рабочая формула: иметь сознание -- значит
владеть языком, владеть языком -- владеть значениями, а значение есть
единица сознания. Еще раз повторим: значение -- это обобщение,
обобщенное отражение действительности. Значение может принадлежать
только знаку и вне такого становится фикцией. Должно быть понятно, что
обобщения могут быть разными, поэтому главный вопрос здесь -- это
вопрос о пластичности обобщений, вопрос о том, как происходит развитие
значений в социальной практике. Мы отметим здесь два существенных
момента. Первое: различие обобщений -- это различие обобщаемого
содержания. Разное вещественное содержание требует разных психических
процессов для того, чтобы оно могло быть обобщено. Более того, одно и то
ж содержание может быть по-разному понято и обобщено в конкретных
социальных контекстах конкретными реципиентами. Второе: человек
встречается с предметным миром не в одиночку, а во взаимодействии с
другими людьми и под их влиянием. И поскольку сознание человека
формируется в общении, то обобщенное отражение мира и общение с
другими людьми предполагают друг друга. Можно сказать, что каково
общение таково и обобщение, поскольку обращенное слово обязательно
ориентировано на конкретного собеседника. Общение приводит к развитию
значений и, соответственно, развитию сознания в процессе взаимодействия
значений -- реальных (конкретного значения предмета) и идеальных
(языковых, понятийных значений). Динамика здесь, по все видимости,
такова:
\begin{enumerate*}
\item Значение предмета есть его свойство или совокупность свойств, в
  котором данный предмет непосредственно выступает для субъекта.
  Непосредственное значение предмета инстинктивно, биологически
  обусловлено и неотделимо от самого предмета.
\item Вместе с тем, значение
  предмета становится идеей, отделяясь от предмета в языке и превращаясь в
  значение слова. Из этого следует, что только когда вещь в ее свойствах
  может быть мысленно представлена, тогда она осознается.
\item Иными
  словами, в значении слов реализуются значения предметов, а само значение
  слова есть форма <<идеального присвоения>> человеком его, человеческой
  действительности.
\end{enumerate*}

Еще один важный момент. Значение предмета не тождественно смыслу предмета,
так как смысл принадлежит не предмету, а деятельности\autocite{leontev2005lectures}.
Лишь в деятельности предмет выступает как смысл.
Наше личное отношение к значению (к предмету, как значению), сообщает ему смысл.
Если в значении кристаллизируется социокультурное отношение к предмету,
то в смысле кристаллизируется индивидуальное, личностное отношение.
По этой причине значение относительно константно, в то время как смысл динамичен.
А.Н.~Леонтьев называл смысл <<значением для меня значения>>\autocite{leontev2005lectures}
На этом основании мы можем улучшить свою предварительную рабочую формулу сознания --
значение есть единица сознания. Проблема развивающегося смысла и есть проблема
сознания. Смысл всегда есть смысл чего-то, рассматриваемый в отношении к кому-то,
т.е. смысл предмета для субъекта. Развитие жизни -- тождественно развитию мотивации --
тождественно развитию смысла. Жизненный мотив, собственно, и выступает как жизненный
смысл, как смысл жизни. Следует подчеркнуть, что смысл и значение, по-видимому,
совпадают на самом раннем этапе развития человеческого сознания, когда нет \emph{я},
нет самосознания, когда личность и коллектив совпадают. Следовательно, совпадают и
индивидуальное сознание с общественным сознанием. Однако по мере разделения,
дифференциации самих отношений между людьми, в силу разделения труда и проч.
необходимо происходит и разделение смысла и значения и социальная дифференциация
смыслов. Таким образом, отношение значения и смысла есть отношение главных
образующих структуры человеческого сознания. Рождение новых мотивов и новых
потребностей ведет к дальнейшей качественной дифференциации сознания\footnote{Более подробно об отечественной теории сознания в: \autocites{leontevan1967}{leontevan1975}{leontevan1983}{vigotski1982}{leontevaa2001}}.


Подытожим наш краткий обзор и выделим главное для нашего исследования:
\begin{enumerate}
\item Сознание есть психическое отражение действительности.
\item Отражение опосредовано знаком, в первую очередь словом, языковым знаком.
\item Значение знака является первой образующей сознания, в то время как смысл
  является его второй образующей.
\item Посредством знака сознание формируется, дифференцируется и развивается
  в общении, взятом в широком смысле слова.
\item Индивидуальное сознание производно от общественного сознания,
  но необязательно им детерминировано.
\end{enumerate}

Сказанное дает нам все основания полагать, что если сознание формируемо,
то процесс его формирования при соответствующих условиях может быть
\begin{enumerate*}[label=\asbuk*)]
\item направляемым и
\item управляемым.
\end{enumerate*} В сущности, в этом заключается весь исторический пафос
философии просвещения и массового образования. Но в этом же, по-видимому,
зачастую заключается и идеологический замысел политических, коммерческих,
речевых, поведенческих и проч. форм социально-психологического воздействия и
манипуляций. Влияние на качество формируемого психического образа, как нам
представляется, происходит посредством дифференцированного воздействия на
знак и его значение, с одной стороны, и на порождение личностных смыслов, с другой.
В первом случае речь идет об адекватности языка отражения, а во втором --
о воздействии на мотивационную сферу личности, на опредмечивание индивидуальных
потребностей.

\subsubsection{Человек как потребитель знаков}

Начнем с уточнения предмета -- человека потребителя. Изначально <<потребитель>> --
это категория экономической науки, упрощенно толкующая человека как
статистическую единицу общественной жизни, чье поведение измеряется в
терминах его потребительских возможностей и наклонностей. Так, могут изучаться
вопросы о том, как потребители распределяют свои доходы в соответствии со своими
вкусами и предпочтениями, как это сказывается на спросе на отдельные виды товаров
и услуг, как изменения в доходах и ценах влияют на спрос, и почему спрос на отдельные
товары более чувствителен к изменениям цен и доходов по сравнению с другими.
Методологической платформой для современных экономических и социологических
исследований служит удобное представление о человеке как о некоей организованной
системе потребностей: физиологических, духовных, нравственных, эстетических и проч.
В идеальной экономической или социологической модели эти потребности является
движущей силой производства и всего социально-экономического прогресса, когда
производство и экономика подчиняются удовлетворению человеческих потребностей
и предпочтений, и, с другой стороны, в определенной степени сами их формируют\autocites{ballestrem1999}{book:bodriyar}{bunkina2000}{klein2003}{kuli2000}{livshits2001}{markuze1994}{maslow2011}{sibruk2005}{fukuyama2004}{alias2001}.
Словами Ж.~Бодрийяра: <<Весь дискурс о потреблении направлен на то, чтобы сделать
из потребителя Универсального человека, всеобщее, идеальное и окончательное
воплощение Человеческого рода.>>\autocite{bodriyar_society}.
% % (Литература: Баллестрем К.Г. Homo oeconomicus? Образы человека в классическом
% % либерализме // Вопросы философии, № 4, 1999, с. 42 53;//
% % Бодрийяр Ж. В тени молчаливого большинства или конец социального. Екатеринбург:
% % Издательство Уральского Университета, 2000;//
% % Бункина М.К., Семёнов A.M. Экономический человек: в помощь изучающим экономику,
% % психологию, менеджмент: Учеб. пособие. М.: Дело, 2000;//
% % Кляйн Н. NO LOGO. Люди против брэндов. М.: ООО «Добрая книга», 2003;// .
% % Кули Ч.Х. Человеческая природа и социальный порядок. М.: Идея-Пресс,
% % Дом интеллектуальной книги, 2000;// Лившиц Р.Л. Потребление и потребительство //
% % Свободная мысль, № 6, 2001, с. 81-89;//
% % Маркузе Г. Одномерный человек. М.: ООО «Издательство ACT»: ЗАО НПП «Ермак», 2003;//
% % Маслоу А.Г. Мотивация и личность. СПб.: Евразия, 1999;//
% % Сибрук Д. Nowbrow. Маркетинг культуры и культура маркетинга. М.: Издательство «Ад
% % Маргинем», 2005;//
% % Фукуяма Ф. Доверие: социальные добродетели и путь к процветанию. М.: ООО
% % «Издательство ACT»: ЗАО НИИ «Ермак», 2004;//
% % Элиас Н. Общество индивидов. М.: Праксис, 2001)
Соответственно, универсальный человек живет в универсальном обществе потребления,
понимаемом в двух смыслах:
\begin{enumerate*}[label=\asbuk*)]
\item как социологическая концепция (У.~Ростоу, А.~Тоффлера и др.), исторически
  возникшая в американской социологии в середине 20~в. в связи с популярными
  представлениями о возможности за счет экономического роста и технических
  нововведений обеспечить каждому члену общества высокий уровень потребления;
\item как оценочно-негативное обозначение общества, в котором деятельность
  людей направлена на неумеренное расширение накопления и потребления материальных
  благ, и в котором способность потреблять рассматривается как признак социального
  успеха и единственный мотив трудовой деятельности\autocite{pankruhin1991}.
\end{enumerate*}
Поскольку нас здесь в первую очередь интересуют семиотические параметры явлений,
мы переходим непосредственно к обозначениям.

\emph{Потребитель} -- калька с английского <<consumer>> этимологически отсылает нас
в Англию начала XV века, где потребителями назывались расточители, растратчики
и транжиры. Негативная коннотация слова, как видно из определения общества
потребления выше, сохранилась по сей день с минимальными изменениями. В тоже время,
как отмечает К.С. Льюис в фундаментальной работе по истории генезиса лексических
значений, слова, изначально обозначающие статус человека в обществе, через какое
то время становятся предписывающими терминами для обозначения типов поведения,
темпераментов и нравов, классов и типов личности\autocite{lewis1990studies}.
Точно так же в самом начале Великой индустриальной революции слово <<потребитель>>
перешло в разряд экономических терминов в 1785 году, в качестве бинарной оппозиции
термину \emph{производитель} (producer). Массовое производство естественным образом
ориентировано на массового потребителя, на человека массы. Ключевые функциональные
производные теперь уже от термина \emph{потребитель} -- \emph{товары потребления}
(consumer goods) и \emph{индекс потребительских цен} (C.P.I.) --
оформились соответственно в 1890 и 1919 гг.\autocite{dictionary}
Наконец, \emph{общество потребления} (consumer society), как уже отмечалось,
вошло в общественный дискурс в виде социологического деривата в середине 20-х
годов прошлого столетия. Такова вкратце история слов как чистых знаков.

Но бытие, мы сказали ранее, не просто отражается в знаке, но преломляется в
нем, делая его потенциально многоакцентным, гибким и изменчивым в соответствии
с изменениями в социальной среде. Для термина \emph{общество потребления}
социальная среда решительно изменилась в 1970-е годы с появлением новой философской
методологии культурного анализа -- постструктурализма (позже постмодернизма).
Постструктуралисты (Ж.~Бодрийяр, Ж.~Делёз, Ж.~Деррида, Р.~Барт, Ю.~Кристева, Ж.~Лакан,
М.~Фуко и др.), провозгласив разрыв с догматичностью -- как им казалось --
сложившейся гуманитарной традиции, ввели в оборот новые аналитические категории:
ризома -- структура-лабиринт, деконструкция -- новый тип понимания, текст,
идеология, игра слов и логомахия, маргинальность и ирония, неразделимость означаемого
и означающего, антигуманизм -- превосходство массового над индивидуальным\autocite{ilyin1996}.
Проще говоря, они предложили установку на условность, знаковость, конвенциональность
реальности как таковой, на реальность, превращенную в слова. С такой реальностью
можно играть, поскольку бытие существует лишь в своем не существовании,
то есть в словах. Рассмотрим это выборочно на примере одной из самых известных работ
Ж.~Бодрийяра <<Общество потребления>> (1970).

В самом упрощенном виде позиция Бодрийяра сводится к тезису, что в обществе
потребления потребляются не вещи как таковые, а знаки и символы, олицетворяемые
этими вещами. В таком обществе <<не только есть предметы и товары, которые желают
купить, но где само потребление потреблено в форме мифа>> \autocite[][3]{bodriyar_society}
Социальность растворяется, и атомизированные потребители <<окружены не столько,
как это было во все времена, другими людьми, сколько объектами потребления>> \autocite[][5]{bodriyar_society}.
Потребитель, включенный в систему потребления, в погоне за удовлетворением своих
потребностей, занят преимущественно <<накоплением знаков счастья>> \autocite[][12]{bodriyar_society}
В то время как сама система потребления <<является системой манипуляции знаками>>.
\autocite[][14]{bodriyar_society} В сущности, по Бодрийяру семиотическая реальность есть единственно
доступная нам реальность: <<Мы живем (\ldots) под покровом знаков и в отказе от
действительности>> \autocite[][15]{bodriyar_society}. Мало того, в некоем семиотическом порыве мы
добровольно отказываемся от действительности <<на основе жадного и умножающегося изучения ее
знаков>> \autocite[][16]{bodriyar_society}.

Иначе говоря, для Бодрийяра в данной работе общество потребления -- это общество
самообмана, в котором невозможны ни подлинные чувства, ни культура.
Даже изобилие в нем является следствием тщательно маскируемого и защищаемого дефицита.
Ключевое понятие социального устройства в таком обществе -- счастье --
рассматривается как абсолютизированный принцип. Счастье наделяется количественными
характеристиками, измеряемыми посредством атрибутов социальной дифференциации.
Этот принцип, по его мнению, лежит в основе современной демократии, смысл которой
сводится к равенству всех людей перед знаками успеха, благосостояния и проч.
Идеология потребления утверждает, что обладание нужными предметами приводит к
преодолению социальных различий, тем самым поддерживая веру человека в демократию и
распространяя миф о равенстве людей. Для Бодрийяра, напротив, это лишь иллюзия
демократии, которая оперирует знаками и предлагает <<социальную игру>>
вместо реального участия людей в общественной жизни. В итоге, демократия знаков и
сопутствующее ей счастье эффектно маскируют реальную дискриминацию в обществе\autocite[][73--116]{bodriyar_society}.

Далее, потребитель в обществе потребления также персонализируется в знаках.
При этом в сфере знаковых различий не остаётся места для подлинного различия,
основанного на реальных особенностях личности. Всё входит в канон исключительно
социальных различий, навязываемой самой системой как кодом, через декларируемые
демократические идеалы быть равно желанными для всех \autocite[][117--128]{bodriyar_society}.
Поэтому культура общества потребления по определению может быть только массовой. Ее отношение
к традиционной культуре подобно отношению моды к предметам, полагает Бодрийяр.
Если в основе моды лежит неизбежное устаревание предметов, так и в основе массовой
культуры лежит принцип старения традиционных ценностей. Массовая культура, другими
словами, изначально может создаваться лишь для непродолжительного использования.
Это особая среда, в которой постоянно сменяются бессмысленные знаки актуальности,
современности, функциональной пригодности для потребителя. Формируется некий общий
минимум таких знаков, необходимый и обязательный для каждого <<культурного>> человека.
Этот минимум Бодрийяр определяет как <<наименьшую общую культуру>>,
выполняющую в массовом сознании роль <<свидетельства культурного гражданства>>.
\autocite[][136--143]{bodriyar_society} В качестве его неизменного атрибута предлагается считать
китч -- никчёмный предмет, не имеющий своей сущности, но распространяющийся
мемически. Соответственно, потребление предмета-китча есть покупка отличительного признака,
своеобразная симуляция приобщения к моде \autocite[][144--146]{bodriyar_society}.

И последнее. Средства массовой информации и реклама в обществе потребления,
уверен Бодрийяр, только отражают и закрепляют его тоталитарный характер, <<гомогенизуя>>
события, извлекая из живого мира лишь те события, содержание которых сводится лишь
к бесконечной отсылке друг на друга. Тем самым СМИ формируют <<неореальность>>,
не имеющую категорий истинности и ложности. Такая <<неореальность>> -- в создании
которой также участвует и реклама -- состоит из <<псевдособытий>>, достоверность
которых не нуждается в анализе достоверности, а требует, подобно рекламному обещанию,
веры в себя \autocite[][150--166]{bodriyar_society}.

В заключении, обобщая проблематику всей книги, Бодрийяр вслед за Г.~Маркузе делает
вывод о конце трансцендентного в общественной жизни. Современный упрощённый миф,
в котором человек становится, по выражению Фрейда, <<богом на протезах>>,
приходит на смену мифу трансцендентному. Мир знаков снимает традиционные противоречия
реальности. Но в этом мире растворяется и сам человек, поскольку он больше не
является индивидуальностью, а состоит лишь из знаков социального статуса.
<<Это -- профилактическая белизна пресыщенного общества, общества без головокружения
и без истории, не имеющего другого мифа, кроме самого себя>>, -- резюмирует Бодрийяр.
\autocite[][245]{bodriyar_society}
З.~Фрейд в статье <<Будущее одной иллюзии>> описывает случай со своим маленьким
сыном, который рано начал проявлять интерес к объективности: <<Когда детям
рассказывали сказку, которую они завороженно слушали, он подошел и спросил:
``Это правдивая история?'' Получив отрицательный ответ, он удалился с пренебрежительной
миной>>\autocite[][40]{freud1992}. По всей видимости, виртуозный нарратив Бодрийяра --
пример философии <<как если бы>>, философии фикций, во всяком случае в том виде
как ее понимал немецкий философ Г.~Файхингер (<<Философия как если бы>> (1911)):
<<Мы относим к области фикции не только нейтральные, теоретические операции, но
и понятийные построения, измышленные благородными людьми, приковывающие к себе
сердца более благородной части человечества, которое не позволит лишить себя их.
Не сделаем этого, конечно, и мы -- позволим всему этому существовать в качестве
практической фикции, в качестве же теоретической истины это отомрет>>\autocite{freud1992}.
Но у Бодрийяра, помимо всего прочего, принцип <<как если бы>>,
вполне приемлемый в философском дискурсе, синкретично сливается с принципом
абсурдности -- сredo, quia absurdum. Иначе говоря, мы имеем дело со спекулятивным
постмодернистским текстом с соответствующими установками на иронию,
игру слов, и прочие аналитические категории, которые мы перечислили в начале.
Все это делает книгу полемически заостренной, но, в конечном итоге, бедной в
содержательной части. Можно было бы сказать, что она сама и является наилучшей
иллюстрацией, предлагаемой Бодрийяром теоретической модели, тождественной языку
изложения, тому же знаку с плавающим означаемым. Плавающим вне пределов светлого
поля сознания.

З.~Фрейд, рассуждая о структурной модели сознания, сравнивает внутреннюю
инстанцию <<сознательного Я>> с маленьким гарнизоном в захваченном городе\autocite[][116]{freud1992}.
Цель психоанализа, если развить чуть дальше его метафору города, заключается в том,
чтобы сделать обширные неподконтрольные территории частью этого гарнизона.
Иначе: сделать неосознанное осознанным. Предназначение же доминантных симулятивных
и манипулятивных механизмов организации жизни в современно обществе потребления, о
которых по-своему экстравагантно, но, по сути, верно рассуждает Ж.~Бодрийяр, сделать
осознанное неосознанным. То есть, получается как бы психоанализ наоборот, когда
последние остатки этого гарнизона становятся бессознательными и последний маленький
гарнизон автономии <<сознательного Я>>.

В этом смысле, массовая культура постмодернизма есть просветление наоборот.
Цель ее -- реверсия, переориентация индивидуального <<сознательного Я>> в <<Оно>>.
Чтобы то, что некогда было только вашим личным, теперь стало бы общим, массовым.
Чтобы проактивное отношение к действительности уступило место стандартизированному
реактивному поведению и <<ложному сознанию>>\autocite[][89]{eagleton1991ideology}.

Ранее (см.~\ref{2.1}) мы утверждали, что сознание опосредовано знаком, в первую
очередь словом, языковым знаком. Мы также утверждали, что сознание формируемо и
процесс его формирования при соответствующих условиях может быть
\begin{enumerate*}[label=\asbuk*)]
    \item направляемым и
    \item управляемым.
\end{enumerate*}
Влияние на качество формируемого психического образа осуществляется
посредством дифференцированного воздействия на знак и его значение, с одной стороны,
и на порождение личностных смыслов, с другой. Затем мы рассмотрели пример
радикальной современной философской рефлексии, абсолютизирующей роль знака в
обустройстве жизни человека-потребителя в современном обществе потребления.
Согласно этой методологии, знаки не только занимают особое место в культуре
сегодня, но и создают некую тоталитарную экосистему обезличенных, бессмысленных
символов, полностью замещающую объективную реальность в сознании человека.
Поскольку объективная действительность для нас все-таки по-прежнему реально
существует, мы обратимся теперь к менее интеллектуальным, но, думается, более
объективным закономерностям влияния знака на качество жизни современного человека в
обществе. Возвращаемся к вопросу о массовизации сознания и принципах воздействия
на него через суггестивные возможности знака. В первую очередь слова родного языка.

\subsubsection{Сознание потребителя как объект знаковой манипуляции}
\label{2.3}
В современной науке язык исторически объясняется как продукт эволюции
человека как биологического вида, как уникальный случай нейро-аккустической технологии,
позволяющей, помимо всего прочего, эффективно влиять на психическое состояние
других людей. <<Вы обладаете  мощнейшим и опаснейшим разрушительным свойством из всех свойств, когда-либо  созданных естественным отбором. Это нейронная аудио технология для влияния на сознание других людей. Она дает возможность имплантировать свои мысли и идеи непосредственно в сознание другого человека, который, в свою очередь, имеет аналогичную  возможность  влиять на ваше сознание, и все это без необходимости какого бы то ни было хирургического вмешательства.>>\autocite{pagel2012wired}.
С самого начала язык, по-видимому, выполнял двоякую функцию: способствовал выделению
человека из массы и способствовал приобщению его к массе. Речевая <<инструкция>>,
как известно, способна активизировать двигательные отделы мозга и стимулировать
прямое ответное действие. На слове построено, в частности, гипнотическое воздействие,
введение в транс, полностью подчиняющее слушающего воле говорящего. На суггестии
в самом прямом смысле слова держатся также все магические ритуалы, шаманство и
колдовство. В первобытные времена, например, транс, вызванный сначала звуком,
а потом и словом, мог переходить в ритуальный танец, мог готовить к охоте или
возбуждать перед сражением\autocite{porshnev1974}. Такой обряд, без сомнения,
укреплял психологическую общность племени, эффективно создавал массовое единство
сознания и действия. Что знак -- уже в силу своей природы -- рассчитан на явную или
внутреннюю поведенческую реакцию утверждал и Л.С.~Выготский\autocite[][179]{vigotsky1956}.

Между тем, помимо суггестии, знак может вызывать и противоположное действие,
именуемое <<отрицательная индукция>> -- процесс торможения, возникший вокруг очага
возбуждения или в том же самом месте после его прекращения. Другими словами,
в некоторых случаях вербальная инструкция приводит в действие противоположную,
индивидуальную волевую аутоинструкцию. Суть аутоинструкции -- <<в способности
отдельного индивида противостоять психологии массы и попыткам массовизации,
не поддаться ни внушению, ни заражению и уклониться от подражания, а также в
появлении психически самоуправляемого индивида.>>\autocite[][96]{olshansky}.
По форме она является разновидностью внутренней речи, противопоставляющей чужому
внушению собственное сознание, магии слова -- самовнушение или собственное размышление.
Б.Ф.~Поршнев называет индивидуальное осмысление чужого слова <<контрсуггестией>>\autocite{porshnev1974}.
Котрсуггестия в этом смысле, разумеется, становится важным индикатором степени
зрелости личности и качественной характеристикой индивидуального сознания.
Итак, повторим еще раз: суггестия есть произносимые одним человеком слова,
жестко предопределяющие поведение другого, при условии, если они не наталкиваются
на контрсуггестию. Или в виде формулы: суггестия = высказывание -- контрсуггестия.

Однако наша формула в таком виде пока еще только обозначение лишь одного
структурного звена в социальной истории человека разумного, поскольку
<<история человеческого общества насыщена множеством средств пресечения всех и
всяческих проявлений контрсуггестии>>\autocite{porshnev1974}. Это важно.
Получается, что предрасположенность к индивидуализации сознания и деятельности
неизбежно влечет за собой выработку новых приемов и способов их повторной
массовизации. Вырываясь из под гнета одной зависимости, будь то первичные
инстинкты, образы или вербальная суггестия, человек неизбежно сталкивается
только с новыми механизмами закабаления. К ним относятся, очевидно, в какой-то
момент и физическое насилие, и воинский призыв, и вера в небесные и земные
авторитеты, поскольку доверие (вера) и суггестия образуют семантическую пару.
Из современных -- пропаганда, реклама, СМИ, интернет, социальные сети -- постоянно
пополняющийся репертуар инструментов обязательной массовизации индивидуального
сознания и поведения. Печально, казалось бы, что на сегодняшний день, пройдя
долгий и сложный путь психологической эволюции, развивая индивидуальную психику
и освобождаясь от власти массового сознания, в психическом смысле, человек
по-прежнему внимает экстатическим воплям шамана у костра в темной пещере,
как двести тысяч лет до нашей эры. С другой стороны, история столкновений
психологии одного человека с психологией масс никогда по-настоящему не
заканчивается. На смену одним формам суггестии всегда будут приходить другие,
как, конечно же, и новые формы контрсуггестии, индивидуальные формы защиты,
дающие хотя бы временное освобождение от их засилья.

Однако рассмотрим более подробно, какие обстоятельства обусловливают эффективность
суггестивного воздействия знака в массовом обществе. В предыдущем параграфе мы выдвинули тезис о том, что объединение людей в массы
происходит под влиянием психологических феноменов заражения, внушения, подражания.
Каждый из них выполняют свою специфическую роль. Опишем их последовательно.
\begin{enumerate}
    \item Заражение подразумевает такую форму массового поведения, когда
    эмоционально возбужденная масса привлекает к себе новых индивидов независимо
    от их желаний, тем самым как бы заражая подобно вирусу.
    \autocites{freid_mass}{petrovsky1990}{porshnev1979}{olshansky}
    По этому принципу обыкновенно действуют, например, мемы, в том числе и интернет-мемы в
    виде картинки, анимации или видеоклипа (из последних: Stoned Fox, Gangnam Style). Феномен заражения в
    основе своей имеет психофизиологическую реакцию, но ею не исчерпывается.
    В крайнем пределе, заражение вовлекает человека уже не просто эмоционально,
    но и содержательно. Тогда речь идет о смысловом заражении личности.
    Из чего следует, что сила заражения напрямую зависит от готовности индивида
    подражать массе, поскольку именно, наравне с предрасположенность к внушению
    формируют условия эффективного заражения. Тем не менее, со строго научной точки
    зрения, заражение -- <<скорее яркий художественный образ, чем реальный
    психологический механизм формирования массы>>\autocite[][86]{olshansky}.

    \item Внушение (суггестия) подразумевает воздействие на человека как вербальными,
    так и невербальными средствами. О первых мы уже упомянули, о вторых разговор еще
    впереди. Их задача -- вызывать определенные состояния, формировать представления,
    побуждать к действию. Но и сама масса изнутри, равно как и сопутствующие ей
    внешние обстоятельства (например, паника), также могут оказывать внушающее
    воздействие\autocites{behterev1898}{porshnev1979}. Далее, поскольку эффективное
    внушение практически никогда не происходит самопроизвольно, внушение предполагает
    присутствие некоего суггестора, лидера или вожака, или даже просто
    неперсонифицированного <<авторитета>>, некоего качества, обладающего
    авторитетностью для данной массы, как, например, модный бренд.
    Важно подчеркнуть, что роль внушения исключительно высока в <<искусственных массах>>\autocite[][69]{freid_mass}, к каковым можно было
    бы с легкостью причислить лояльных бренду потребителей. И значительно
    ослаблена в <<естественных>> массах, возникающих стихийно, как, например,
    при задержке рейса в аэропорту. Иначе говоря, как и прежде, действенное внушение
    предполагает в человеке склонность к заражению и предрасположенность к подражанию.

    \item Подражание подразумевает отказ от собственного выбора принятия
    индивидуального решения и следование чьему-либо примеру или образцу.
\end{enumerate}
Подражание (осознанное или неосознанное) действию, поступку, одежде, стилю,
речи и т.д. Интересно, что именно готовность к подражанию делает эффектным
внушение и обеспечивает успешность заражения\autocites{pagel2012wired}{freid_mass}{porshnev1979}.
Подражание отчасти объясняется принципом экономичности деятельности мозга и,
следовательно, психической жизни человека. Действительно, творческое, самостоятельное
поведение занимает сравнительно небольшую часть времени в нашей повседневной жизни.
По большей части она состоит из стереотипных, автоматизированных, будничных действий,
некогда усвоенных через имитацию каких-то образцов. Имитационное поведение в чистом
виде наблюдается в экстремальных ситуациях, или даже на предпраздничных массовых
распродажах (например, Черная Пятница в конце ноября в США), когда сознание
потребителя временно отключается как из-за нехватки времени на самостоятельный
анализ происходящего, так и из-за включения рефлекторных механизмов.
Менее очевидный для саморефлексии вариант -- подражание в повседневном поведение,
которое вообще редко осознается как подражание. (Например, синдром лемминга при
переходе улицы в неположенном месте.) Более комплексный характер имеет подражание
личностно значимому авторитету или референтной группе, с которой человек отождествляет
себя (поведение спортивных болельщиков и других молодежных субкультур).
Наконец, заниженная самооценка также может подталкивать человека к имитативному
поведению, и неуверенный в себе человек готов подражать кому и чему угодно.
Хотя в другом пределе, подражание может быть и вполне расчетливым действием,
позволяющим достичь определенных выгод (принцип <<быть не хуже других>>).

Таким образом, когда подражание становится самодовлеющей потребностью,
оно побуждает человека к воспроизводству копируемого им поведения других людей,
побуждает его следовать предлагаемым образцам регуляции своих эмоций. Заражение,
внушение и подражание работают комплексно, считают психологи. По это причине
взаимоотношения индивида и массы никак нельзя объяснить исключительно действием
какого-то одного из них. Каждый механизм представляет собой лишь один фактор,
обеспечивающий все многообразие вариантов отношений между ними, из которых, в
конечном итоге, формируется единый механизм возникновения и развития массового
поведения и формируется чувство общности <<мы>> -- эмоциональная матрица психологии
любой массы. В том числе и массы, именуемой <<потребители>>.

Итак, мы попыталась показать, что современный человек, вопреки пессимистичным
уверениям Ж.~Бодрийяра, не состоит исключительно из знаков социального статуса, и
реальность в которой он живет далеко не стерильна, но по-прежнему полна противоречий
и эволюционных изменений. Знак, по крайней мере, языковой знак, по-прежнему выполняет
свою социальную функцию посредника, хотя, разумеется, с поправкой на время.
Точно так же «неореальность» , создаваемая СМИ и рекламой - фикция лишь по форме и
содержанию, но вполне реалистична и читаема по своему замыслу и назначению.

Проследим это на примере хотя бы коммерческой рекламы.

\subsubsection{Человек как потребитель рекламы}
\label{2.4}

Коммерческая реклама исторически есть один из способов формирования массового
потребительского поведения. В качестве примера самого древнего рекламного текста
обычно приводят надпись, высеченную на камне в руинах Мемфиса где-то 2500 лет
назад: <<Я, Рино с острова Крит, по воле богов толкую сновидения>>\autocite{katernuk2001}.

Современная реклама как массовое социально-психологическое явление возникла,
очевидно, в период интенсивного становления массового машинного производства в
Европе и Америке. Такое производство приводило к перепроизводству, излишками
продукции, которые нужно было реализовывать. Для этого потребовалось создание и
воспитание соответствующей массы потребителей, людей, прежде живших натуральным
хозяйством. Соответственно, реклама стала инструментом формирования таких масс,
устанавливать новые стандарты поведения в быту, равно как и новые стандарты в
сознании населения. В 70-е годы прошлого века французский социолог и культуролог
Жак Эллюль подчеркивал: <<Массовое производство требует массового потребителя,
но массовое потребление не может существовать без широко распространенных взглядов
на то, что считается жизненно необходимым\ldots
Поэтому необходимо фундаментальное психологическое единство, на котором может
с уверенностью играть реклама, манипулируя общественным мнением\ldots
Таким образом комфортность жизни и комфортность мысли связаны неразрывно>>.
(Цитата по: \autocite{golova}) Или, согласно прагматичному маркетинговому принципу:
рекламе следует заниматься массовым производством потребителей точно так же,
как фабрики заняты массовым производством товаров.

Примечательно, что единого определения рекламы не существует, а в существующих
определениях акцент обычно делается на то, что она делает или должна делать --
информировать, пропагандировать, одурачивать, но не на ее целевую функцию, т.е.
для чего она собственно нужна. Но дело в том, что реклама есть не просто канал
донесения информации, но что гораздо важнее -- механизмом формирования особой массы,
в нашем случае -- массы потребителей. Слабым пониманием этого механизма, по-видимому,
отчасти объясняется низкая эффективность большинства современной рекламы.
Парадоксально: c одной стороны, в поисках уникальных коммерческих предложений
она, казалось бы, постоянно совершенствует технику и приемы воздействия,
делая их более конкретными, утонченными и дифференцированными по социальным
срезам и демографическим признакам, она все-таки <<постепенно утрачивает свое
главное и естественное свойство: быть механизмом формирования (\ldots) масс
потребителей, толпящихся в очередях, изнывающих от неукротимого желания
употребить что-то, предлагаемое рекламой>>\autocite[][312]{olshansky}. Примеров
рекламы, не выполняющей своей главной функции удовлетворения эмоционально
значимых потребностей множество. Вот, например, как это художественно описывает
Н.В.~Гоголь в первой главе <<Мертвых душ>>. Чичиков въезжает в губернский город NN:
<<Попадались почти смытые дождем вывески с кренделями и сапогами, кое-где с
нарисованными синими брюками и подписью какого-то Аршавского портного;
где магазин с картузами, фуражками и надписью: ``Иностранец Василий Федоров'';
где нарисован был биллиард с двумя игроками во фраках, в какие одеваются
у нас на театрах гости, входящие в последнем акте на сцену.
Игроки были изображены с прицелившимися киями, несколько вывороченными
назад руками и косыми ногами, только что сделавшими на воздухе антраша.
Под всем этим было написано: ``И вот заведение''.>>\autocite[][11--12]{gogol2006}.
При этом, как ни странно, людей ни снаружи, ни внутри <<заведения>> видно не было.

Обычный перечень решаемых рекламой конкретных задач включает в себя:
\begin{enumerate*}[label=\asbuk*)]
\item извещение;
\item заражение;
\item убеждение;
\item внушение;
\item напоминание,
\end{enumerate*}
поскольку именно таким образом она производит массового
человека-потребителя товаров, как в реальной, так и виртуальной среде.
В тоже время в современном обществе реклама принимает на себя важную
функцию социального контроля, психологически формируя массы
потребителей под наличное производство и обеспечивая адекватное взаимное
развитие\autocite{feofanov1987}. Соответственно, ответы на вопрос -- почему люди поддаются рекламе --
следует искать не столько в количестве и качестве самой рекламы,
а сколько в некоторых психологических особенностях людей. Точнее, в
психологических особенностях массового человека, о котором мы рассуждали в
предыдущем параграфе, и который так или иначе дремлет внутри каждого из нас.
Вспомним его базовые потребности: идентификация себя с большой группой и
конформизм, чтобы таким образом отрегулировать свои эмоциональные проблемы.

Конечная цель рекламного воздействия хорошо известна -- соблазнить массового
покупателя совершить покупку, заразить его таким желанием (см. \ref{2.3}).
Исследования показывают, что это можно сделать двумя путями:
\begin{enumerate*}[label=\asbuk*)]
    \item сформировать новую потребность, или
    \item актуализировать старую, находившуюся в <<дремлющем>> состоянии.
\end{enumerate*}
Потребность, как писал А.Н.~Леонтьев, представляет собой не что иное,
как опредмеченную нужду\autocite{leontev2012}.
Любая нужда есть состояние дискомфорта, дисбаланс каких-то состояний в организме.
О поведении человека, который не может разобраться в своих желаниях,
иронически писал М.Е.~Салтыков-Щедрин в сатире <<Культурные люди>> (гл. 1):
<<Чего-то хотелось: не то конституции, не то севрюжины с хреном, не то кого-нибудь
ободрать.>>\autocite[][295]{saltikov2000}. Реклама предлагает опредметить нужду,
подсовывая некий мотив -- потенциальный предмет, могущий удовлетворить его
потребность. В коммерческой рекламе важно, чтобы потребитель захотел именно
севрюжины, и уж совсем не конституции или кого-нибудь ободрать.

Мотив обыкновенно выполняет две функции: с одной стороны, реальное побуждение
к действию, чтобы овладеть предметом и удовлетворить возникшую потребность.
С другой, образование личностного смысла. Иначе говоря, наличие мотива
позволяет придать некий индивидуальный смысл деятельности потребителя по
удовлетворению своих потребностей.

По словам Г.~Лебона, идеи мало как влияют на поведение масс, пока они не переведены
на язык чувств. Получается, что рекламе поддаются просто потому, что хотят
ей поддаться. Потому, что <<дело тут не в тонкостях различий разных рекламных
носителей, не в графике текста, и даже, часто, не в слогане (хотя это,
возможно, единственное, что имеет принципиальное значение за счет все той же,
непоколебимой для массовой психологии суггестивной силы вербального внушения).
Прежде всего, дело в эмоциональных состояниях и в стоящих за ними потребностях>>\autocite[][315]{olshansky}.
Добавим в развитие: через обращение преимущество к оптимистическим эмоциональным
состояниям, к вере человека в то, что он что-то приобретет, а не потеряет
(loss aversion).

Многочисленные психологические эксперименты демонстрируют, что человек,
даже вопреки очевидности, обладает почти исключительной склонностью верить
именно в то, во что ему хочется верить. Понятно, что специалисты в области
рекламы быстро сделали из этого разумный вывод: нет необходимости прибегать
к логическим доказательствам, когда речь идет только о том, что потребитель
хочет услышать. Можно оперировать исключительно на уровне лимбической системы,
отвечающей за инстинктивное поведение, настроение и эмоции. Соответственно,
реклама помогает оптимизировать в оптимистическую сторону настроение тех,
на кого она воздействует, практически сразу же снимая отдельные эмоциональные
проблемы. Так, у человека может появляться ощущение безопасности, защищенности,
расслабленности, поднимается самооценка и т.д. Мало того, он может начать
ощущать себя членом некой виртуальной общности эмоционально позитивных людей,
которым дано особое право на выбор
\autocites{ariely2009predictably}{dittrich2008upside}{martin2012}{lindstrom2010}.
Тем самым реклама <<искусно использует весь спектр эмоционального воздействия,
апеллируя к желанию человека быть здоровым и благополучным, к его тщеславию,
стремлению сохранить или повысить свой социальный статус, т.е. ко всему, чем
жив человек>>\autocite{feofanov1987}.

Итак, повторим главное: психологически, реклама, демонстрируя тот или иной мотив,
либо формирует, либо выводит в поле сознания <<дремлющую>> потребность.
Тем самым она возбуждает желание заполучить этот внезапно возникший предмет
потребности.

Затем, через уже названные нами ранее механизмы массовизации, реклама задействует
внушение и заражение для побуждения потребителей к овладению желанным предметом.
После этого она разными способами поощряет массовое подражание, во-первых,
визуальному и вербальному ряду рекламного сообщения, и, во-вторых, тому,
что уже делают другие потребители, зараженные данной рекламой. При этом реклама,
конечно же, всегда стремится активно опираться на ранее сформированные ею тренд или
моду и использовать специальные приемы <<стимулирования спроса>>. В конечном итоге,
реклама приучает человека не только к конкретному бренду, товару или услуге,
но и к рекламе вообще и формирует массу потребителей вообще. Сегодня люди --
потребители элитных смартфонов и планшетов, завтра -- часов или очков, послезавтра --
чего-нибудь еще.
\hyphenation{прос-нулся}

В заключение приведем еще одну развернутую цитату из монографии Д.В.~Ольшанского:
<<Реклама смогла разбудить суггестивные механизмы самоорганизации психологии масс.
В результате внутри миллионов людей проснулся первобытный ``массовый человек'' и,
совсем немного посопротивлявшись своему индивидуальному ``я'', преклонился перед
новым богом-вождем-шаманом -- которого в этот раз называют рекламой. Он уже готов
заразиться. (\ldots) Он стал Потребителем давно, еще в пещере, когда тот же вождь
выдавал ему кусок мяса, расхваливая именно данный кусок. Он вновь стал им в эпоху
массового производства, когда на него обрушились первые потоки рекламы.
Так что\ldots куда он денется?>>\autocite[][321]{olshansky}.
\hyphenation{ус-пех}

Из сказанного, разумеется, не вытекает, что рекламная деятельность -- это гладкий,
хорошо отлаженный процесс, гарантирующий обязательный успех. Деться массовому
потребителю, конечно же, как и прежде, есть куда. В индивидуальное сознание,
в <<гарнизон>> Фрейда, где инстинктивные душевные порывы, магические знаковые
комбинации и прочие хитроумные приемы массообразования критически осмысливаются и
нейтрализуются аналитическими знаковыми построения другого порядка.
При этом потребитель перестает быть потребителем и становится личностью.
Пусть только на время.

Резюмируем:
\begin{enumerate}
\item Мы последовательно представили триаду человек-потребитель-знак с позиции
  \begin{enumerate*}[label=\asbuk*)]
  \item отечественной психологической теории знака;
  \item семиотической интерпретации общества потребления Ж.~Бодрийяром;
  \item социально-психологической теории массового поведения и
    массообразования.
  \end{enumerate*}
\item Особый акцент был сделан на социальное происхождение языкового знака и
  сложную динамику его функционирования в общественном и индивидуальном
  сознании. Использование знака, его значение и смысл определяются конкретной
  практикой жизни человека в обществе.
\item Мы не согласились с необходимостью абсолютизировать роль знака в
  современном обществе, предложенной Ж.~Бодрийяром, и в качестве альтернативы
  попытались представить знак в контексте объективных психологических
  процессов формирования массового поведения и массового сознания.
\item Мы также попытались показать, что манипулятивное использование знака
  возможно лишь при условии возникновения эмоционально заряженной
  массовой общности, функционирующей на базе ключевых механизмов массообразования:
  заражения, внушения (суггестии) и подражания.
\item Присоединение к массе и выход из нее конкретного индивида,
  как нам представляется, носят необходимо цикличный характер и в целом совпадают с циклами, с одной стороны, удовлетворения эмоциональной потребности,
  и, с другой -- повышения критичности восприятия информации или контрсуггестии.
\item Появление человека-потребителя, равно как и его специфической среды
  обитания -- общества потребления и массовой культуры, мы полагаем, исторически
  и по сей день является целенаправленным действием и закономерным
  следствием возникновения и развития массового производства.
\item Реклама, вопреки расхожему представлению о том, что ее главная цель --
  манипуляция сознанием, по нашему мнению, является одной из движущих сил
  в организации и саморегуляции процессов формирования массовых общностей
  потребителей, равно как и структурировании повседневной жизни людей в
  современном обществе.
\end{enumerate}

Теперь мы существенно сузим фокус нашего изложения и сосредоточим все внимание
на одном конкретном коммерческом знаке -- логотипе. Он интересен нам по нескольким
причинам. Во-первых, это знак, который имеет ярко выраженную рекламную
направленность. Следовательно, в свете всего сказанного о рекламе выше, он
также принимает участие в формировании потребительских масс и ориентирован на
возбуждение желаний, внушение, заражение и подражание потребителей. Во-вторых,
логотип привлекателен своей потенцией психо-семантической мутации, превращения в
бренд или даже в символ, наделенный личностным смыслом, субстанциональным тождеством
идеи и вещи, по А.Ф.~Лосеву. Наконец, он интересен и тем, что является конкретным
реально действующим знаком, а никак не симулякром, не имеющим референта в
объективной реальности, если доверчиво следовать логике французских
постструктуралистов. Итак, переходим к логотипу.
