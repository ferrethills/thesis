\section{Генезис логотипа}

\subsection{Ментальность массового общества в свете семиотики}\label{1}
Слово \emph{общество} означает совокупность людей, объединенных социальными формами
совместной жизни и деятельности. Понятие \emph{общество} является одним из самых многозначных в
современных гуманитарных науках. Так, в социальной философии оно может рассматриваться как обособившаяся
от природы часть материального мира, развивающаяся форма жизнедеятельности людей,
равно как и особая, высшая ступень развития живых систем. В исторических науках \emph{общество}
-- это совокупность исторически сложившихся форм совместной деятельности людей.
В психологии -- это единая социальная группа, внутренняя организация которой представляется
совокупностью характерных, типических связей. В социологии -- социум, которому присуще
производственное и социальное разделение труда, а в культурологии оно нередко выступает в
качестве синонима термину культура. Поскольку терминология -- это орудие, посредством
которого делается точное наименование, или, по Аристотелю, подлежащее и сказуемое суждения,
первоначально обратимся к уточнению рабочих границ терминов, заявленных в заглавии данного
параграфа. Последовательно рассмотрим термины: \emph{массовое общество}, \emph{ментальность},
\emph{семиотика}.\autocite{society}

% Слово  означает совокупность людей, объединённых социальными формами
% совместной жизни и деятельности. Понятие \emph{общества} является одним из
% самых многозначных в современных гуманитарных науках.
% Так, в социальной философии оно может рассматриваться как обособившаяся от
% природы часть материального мира,
% как развивающаяся форма жизнедеятельности людей, как особая,
% высшая ступень развития живых систем.
% В исторических науках \emph{общество} -- это совокупность исторически
% сложившихся форм совместной деятельности людей. В психологии
% это единая социальная группа, внутренняя организация которой представляется
% совокупностью характерных, типических связей, в социологии -- социум,
% которому присуще производственное и социальное разделение
% труда, а в культурологии оно нередко выступает в качестве синонима к термину \emph{культура}.
% Поскольку терминология -- это орудие, посредством которого выполняется точное
% наименование, или, по Аристотелю, подлежащее и сказуемое суждения,
% первоначально стоит обратиться к уточнению рабочих границ терминов,
% обозначенных в заглавии данного параграфа. Последовательно рассмотрим термины:
% \emph{массовое общество}, \emph{ментальность}, \emph{семиотика}.\autocite{society}

% При первом приближении термин \emph{массовое общество} кажется тавтологическим или
% по крайней мере парадоксом. Действительно, \emph{общество} по определению
% является формой множественности, массовый или собирательный характер которой становится
% её единственным оформлением. Оппозицией слову \emph{общество} в этом смысле,
% по всей видимости, будет выступать \emph{индивид}. Но массовое общество,
% определенное количественным признаком, вряд ли образует продуктивное противопоставление
% слову \emph{индивид}, поскольку они находятся в разных качественных категориях.
% Массовое общество нуждается в количественной оппозиции. Было бы естественно предположить,
% что ею должно стать некое не-массовое общество, в крайнем пределе общество, выстроенное по принципу единичности. Понятно, что такого общества крайнего предела не существует. Ведь Робинзон до встречи с Пятницей жил жизнью или бога, или зверя, если следовать за логикой Аристотеля. Следовательно, требуется иная плоскость рассуждения, каковой может послужить для наших целей, например, плоскость стилистическая. Здесь массовое общество противостоит уже не единичности в точном, объективном смысле, а единичности как избранности, элитарности некоей малой группы внутри целого общества. Массовое общество – это оценочный эпитет с негативной коннотацией, то, что в английском языке получило статус non-U, т.е. не принятое в высшем обществе. http://www.unz.org/Pub/Encounter-1955sep-00005.

\subsubsection{Рабочие границы терминов \emph{массовое общество}, \emph{ментальность}, \emph{семиотика}} \label{1.1}
\paragraph{Массовое общество.}\label{1.1.1}

При первом приближении термин \emph{массовое общество} кажется тавтологическим
образованием или, по крайней мере, парадоксом. Действительно, \emph{общество} по
определению является формой множественности, массовый или собирательный характер
которой является ее единственным оформлением. Оппозицией слову \emph{общество} в этом смысле,
по всей видимости, будет выступать индивид. Но массовое общество, определенное
количественным признаком, вряд ли образует продуктивную оппозицию со словом \emph{индивид},
поскольку они находятся в разных качественных категориях. Массовое общество нуждается в
количественной оппозиции. Было бы естественно предположить, что ею должно стать
некое немассовое общество, в крайнем пределе общество, выстроенное по принципу единичности.
Понятно, что такого общества крайнего предела не существует. Ведь Робинзон до встречи с
Пятницей жил жизнью или бога, или зверя, если следовать за логикой Аристотеля.
Следовательно, требуется иная плоскость рассуждения, каковой может послужить для наших целей,
например, плоскость стилистическая. Здесь массовое общество противостоит уже не единичности в
точном, объективном смысле, а единичности как избранности, элитарности некоей малой
группы внутри целого общества. \emph{Массовое общество} -- это оценочный эпитет с
негативной коннотацией, то, что в английском языке получило статус non-U, т.е.
не принятое в высшем обществе. \autocite{english_aristocracy}

И действительно, история термина служит тому подтверждением.
Так, Д.В.~Ольшанский в монографии "Психология масс" берет за точку отсчета
публицистику Эдмунда Берка и Жозефа Де Местра на рубеже 18-19 веков,
пророчествовавшим об угрозе ``толпы'' или ``массы'' аристократическим институтам
в Европе. Оценочный эпитет претерпел семиотическую мутацию на протяжение 19-го века,
нарастив свою абстрактную составляющую, и актуализировался в работах Гюстава Лебона,
первого признанного теоретика \emph{массового общества}, уже в качестве научного понятия.
Лебон рассматривает толпу в качестве психологического феномена, возникающего при прямом
взаимодействии индивидов, независимо от их социального статуса, рода деятельности или
национальности. По его мнению, в толпе формируется некое социально-психологическое
единство массы, наделенное общими чувствами, взаимовнушением, высоким уровнем энергетики,
нейтрализующими сознательную личность. К более поздним серьезным и разноплановым
теоретическим исследованиям \emph{массового общества}, написанным главным образом в
20-60-е годы ХХ века, можно было бы отнести работы Габриеля Тарда, Георга Зиммеля,
Макса Вебера, Карла Маннгейма, Эмиля Ледерера, Герберта Блумера, Дэвида Рисмена,
Даниеля Белла, Хорхе Ортеги-и-Гассета, Ханны Арендт, Льюиса Мемфорда, Фридриха Юнгера,
Эриха Фромма, Герберта Маркузе, Уильяма Корнхаузера, Жана Бодрийяра, Эдварда Шилза и др.
Анализ теоретических платформ названных авторов не входит в задачи данной работы.
Нас интересует здесь другое, а именно: разнообразие концептуальных интерпретаций
понятия \emph{массы}.
Д.В.~Ольшанский в своем историческом экскурсе\autocite{book:olshansky}
перечисляет семь таких подходов:
\begin{enumerate}[label={\arabic*)}]
    \item как толпа (традиции Г.~Лебона);
    \item как публика (последователи Г.~Тарда);
    \item как гетерогенная аудитория, противостоящая классам и относительно гомогенным
    группам (Э.~Ледерер и Х.~Арендт, например, считали массы продуктом дестратификации
    общества, своего рода \emph{антиклассом});
    \item как \emph{агрегат людей, в котором не различаются группы или индивидуумы} (У.~Корнхаузер);
    \item как уровень некомпетентности, как снижение цивилизации (X.~Ортега-и-Гассет);
    \item как продукт машинной техники и технологии (Л.~Мамфорд);
    \item как \emph{сверхорганизованное} (К.~Маннгейм) бюрократизированное общество,
    в котором господствуют тенденции к униформизму и отчуждению.
\end{enumerate}

В сущности, список можно с легкостью продолжить, добавив, например, постмодернистскую
трактовку Ж.~Бодрийяра, к чьим работам мы будем неоднократно обращаться позднее.
Бодрийяр, как известно, ввел альтернативный термин для описания массы (\emph{молчаливого большинства}) --
\emph{общество потребления}, эволюции массового общества на новом историческом этапе.
В книге ``В тени молчаливого большинства, или Конец социального''~\autocite{book:bodriyar} он, в частности,
утверждает, что масса -- это разновидность \emph{черной дыры}, падение в самих в
себя, которое не обладает ни атрибутом, ни предикатом, ни качеством, ни референцией.
Масса, по Бодрийяру, радикально неопределима и не имеет социологической реальности.
Она не является ни субъектом истории, ни ее объектом. Любая репрезентация масс -- всего-навсего
ее симулякр. Остановимся здесь. Продолжать нет смысла, поскольку мы рискуем полностью
потерять научный предмет и погрузиться в область метафорических образов, риторических
тропов и фигур речи.

Термин \emph{массовое общество} имеет риторическую, публицистическую основу и в этом,
думается, его слабость. Но если литература -- это речетворчество, высвобождающее слово
из-под давления обязательных предписаний и ограничений, язык в движении, то наука,
напротив, это строгая система терминов. П.А.~Флоренский в ``Водоразделах мысли''\autocite{book:florensky1} посвящает
специальный раздел антиномии языка, т.е. его художественному и научному потенциалу.
Суть науки, подчеркивает Флоренский, в построении и упорядочении терминологии.
``Не ищите в науке ничего кроме терминов, данных в из соотношениях:
все содержание науки, как таковой, сводится именно к терминам в их связях,
которые (связи) первично даются определениями терминов.''~\autocite{book:florensky2}
В другом месте он говорит о термине как границах, межах мысли, oratio rei essentiam significans
-- фразе или слове, обозначающем сущность вещи.
\emph{Массовое общество}, как мы постарались схематично показать выше, тяготеет к
постоянному расширению границ за счет включения термина в различные знаковые системы,
различающиеся не только, и не сколько по степени абстрактности составляющих их знаков,
сколько по неустойчивости формирующихся причинно-следственных связей между ними.
Аналогичная проблема, как мы увидим, будет возникать всякий раз, когда мы будем
обращаться к определению каждого последующего термина или понятия. Но других терминов нет.
Поэтому, полностью осознавая неизбежность возникновения противоречий, субъективности
и возможной тенденциозности наших суждений, перейдем к рассмотрению термина \emph{ментальность}.
Но предварительно сформулируем свое рабочее определение \emph{массового общества}
для данной работы.

\emph{Массовое общество} -- это теоретическая модель, интерпретирующая неизбежное расширение
эгалитарных тенденций в современном обществе, вызванных интенсивной индустриализацией,
повышением роли городов, развитием технологий, СМИ, всеобщим образованием,
демократизацией политики и т.д.

\paragraph{Ментальность}
\label{mentality}
Лингвистически, ментальность -- это существительное высокой степени абстрактности,
образованное от абстрактного прилагательного \emph{ментальный}, т.е. относящийся к уму,
к умственной деятельности. Этимологически, ментальность как понятие, по всей видимости,
восходит к латинскому \emph{mens}, \emph{mentis} -- разум, ум, интеллект, и, как таковое,
не имеет аналога в русском языке. Здесь оно обычно толкуется как особенности мировоззрения,
обусловленные культурно-национальной идентичностью, воспитанием, образованием и образом жизни.
Слово \emph{ментальность} имеет хождение также и в современных европейских языках, в частности,
в английском -- \emph{mentality}. Но если в русском языке оно носит преимущественно нейтральный,
книжный характер, то в английском это далеко не всегда так. Например, в Оксфордском словаре
английского языка~\autocite{oxford_dictionary} оно регистрируется параллельно в категориях
исчисляемых и неисчисляемых существительных, и в первом корневом значении описывается как
(часто уничижительное): характерный способ мышления отдельного человека или группы.
Во втором (устаревшем) значении -- способность к понятийному мышлению. И только после этого
предлагаются более понятные русскоязычному читателю значения: ум, интеллект, мировоззрение
и проч.~\autocite{oxford_dictionary}\autocite{oxford_american}\autocite{collins}\autocite{merriam}
Согласно этимологическому словарю, слово \emph{mentality} появляется в английском языке в
конце 17-го века и постепенно начинает использоваться в философской литературе в значении
``умственная предрасположенность, душевный склад'', но пока еще не в качестве термина.
Собственно научное использование слова \emph{ментальность} принято связывать с публикациями
французского этнолога и антрополога Люсьена Леви-Брюля, исследовавшего дологическое мышление и
\emph{коллективные представления} -- \emph{ментальности} бесписьменных народов.
С другой стороны, философское истолкование понятия \emph{ментальности} связывают с работами
немецкого философа Эрнста Кассирера, предлагавшего систематизировать ментальности по способам
восприятия окружающего мира. Понятие о \emph{коллективных представлениях} детализировалось и
получило дальнейшее развитие в трудах Эриха Фромма~\autocite{book:davydov}, Гастона Бутуля,
Люсьена Февра и Марка Блока~\autocite{book:febvre}, Эмануэля Ле Руа, Жака Ле Гоффа,
Мишеля Фуко~\autocite{book:arch}, Карла Густава Юнга~\autocite{book:yung},
Йохана Хейзинги~\autocite{book:heizenga}, Ганса-Вернера Гетса, Фернана Броделя, Михаила Михайловича
Бахтина~\autocite{book:tamarchenko}, Арона Яковлевича Гуревича и др.

Следует признать, тем не менее, что во многом понятие \emph{ментальность} продолжает
оставаться рабочим понятием многих современных исследований. Частотность его использования
в различных гуманитарных дисциплинах очень высока, хотя строгое терминологическое определение
ментальности пока еще не оформилось. В самом общем виде \emph{ментальность} определяется как
видение мира и его восприятие, образ мысли и нормы поведения, в которых сочетаются
сознательные и бессознательные моменты.

Можно было бы выделить следующие обобщенные трактовки понятия \emph{ментальность}:
\begin{enumerate}[label={\arabic*)}]
    \item ментальность -- это общий тип поведения, свойственный индивиду и представителям
    определенной социальной группы, в котором выражено их понимание мира в целом и
    их собственного мира в нем;
    \item ментальность -- это эмоциональная и дологическая предрасположенность,
    бессознательные и неотрефлексированные способы поведения и реакций (Г.В.~Гетц).
\end{enumerate}

Как видим, вторая точка зрения прямо противоположна первой. Здесь ментальность -- это то,
что \emph{обладает} человеком, т.е. картина мира, которая не сформулирована и в принципе не
поддается формулировке ее носителем. Или еще один подход:
\begin{enumerate}
    \item[3)] ментальность есть сама психология, взятая в контексте социальных условий; это обыденность,
    средний человек и способы чувствования, мышления, силы, формирующие привычки, отношения,
    безличный культурный контекст.~\autocite{online:kulturolog}
\end{enumerate}

Иначе говоря, в самой идее ментальности синкретично соединяются характеристики психологической и
непсихологической (культурной и социальной) реальностей. Важно помнить, что ментальность,
так или иначе, находит свое выражение в повседневной жизни людей. А поскольку повседневностью живет
все-таки каждый человек, то категория <<повседневности>> в данном случае будет служить краеугольным
камнем в нашем исследовании. По Ф.~Броделю, повседневность внеисторична, расплывчата, неопределенна
и неопределима. Тем не менее, к ее организующим принципам можно отнести, во-первых, \emph{социальную иерархию}
(\emph{верхи} и \emph{низы}). \emph{Низовой}, массовой культуре, присущи стабильность,
консерватизм, устойчивость к переменам. \emph{Элита}, верхи общества, определяют
ценностные ориентиры и создают иерархические порядки. Во-вторых, понятие стиля или стилизации.
При этом стиль, как некая культурная целостность, влияет как на материально-предметный мир,
так и на поведенческие нормы, проявляясь в речи, манерах и пр. В-третьих, категорию времени и
пространства. Позднее мы подробнее остановимся на этом вопросе. А пока сформулируем понимание
ментальности в рамках данного исследования.

Ментальность -- это обобщенное представление о природе типической предрасположенности
в осмыслении действительности в неотрефлексированном, неосознанном поведении человека или
социальной группы в повседневной жизни.

\paragraph{Семиотика}\label{1.1.3}
Самое распространенное определение семиотики предлагает считать ее наукой о знаках и
знаковых системах. В первом приближении, знак -- это некоторая значащая условность,
которой может наделяться любое изображение, предмет, звук или звуковой комплекс,
действие или жест в социальной жизни человека в обществе. Сами по себе изображения,
предметы, звуки и т.д. присутствуют в реальной жизни как бы в двух ипостасях:
\begin{enumerate*}[label=\asbuk*)]
\item как тождественные самим себе и
\item в качестве условных заместителей некоторого другого предмета, явления или свойства.
\end{enumerate*}

Именно во втором случае мы получаем достаточные основания,
чтобы вести речь о знаках как материализованных носителях образа обозначаемого предмета,
необходимо ограниченного его функциональным предназначением. Иначе говоря, каждый отдельно взятый
элементарный знак в своей материальной основе является не-знаком, равно как и сам знак принципиально
не тождественен обозначаемому им предмету. Знак -- это инструмент (или орудие),
с помощью которого решаются две основные задачи:
\begin{inparaenum}[\itshape 1\upshape)]
    \item быстрого получения, передачи, переработки и
    \item надежного хранения знаний/информации в обществе.
\end{inparaenum}

Или же, в определении Ю.М.~Лотмана: ``Знак -- это материально выраженная замена
предметов, явлений, понятий в процессе обмена информацией в коллективе.''~\autocite{wiki:symbol}
Здесь, по-видимому, следует сделать важное уточнение. Знаком заменяется не столько сам
обозначаемый предмет, взятый в своей целокупности, сколько его отдельные свойства или
качества, метонимически представляющие его целостность в индивидуальном и коллективном сознании.
Отсюда естественным образом вытекает важнейшая ценностная образующая знака -- его значение.
Помимо указания на предмет, значение знака есть не что иное, как наше представление об означаемом
предмете, коллективно признанное и индивидуально осознанное. Более подробно мы рассмотрим
этот вопрос в последующих разделах. А пока продолжим наши предварительные замечания о семиотике как науке.

Хотя литература по семиотике обширна, ее институциональный статус по-прежнему не определен.~\autocite{sirotkin}
Более того, современная семиотическая теория на данный момент также не имеет ни
единой методологии, ни общей теории.~\autocite{gorny}\autocite{gasparov}. Исторически, тем не менее,
первым подлинным семиотиком-теоретиком в истории семиотической мысли принято считать работы
христианского теолога Августина Блаженного предвосхитившего современное понимание
знака, как связанного с обозначаемым предметом, с одной стороны, и с представлением об этом
обозначаемом в нашем сознании, с другой. При этом он рассуждает о принципе двойственном подчинении знака:
\begin{enumerate*}[label=\asbuk*)]
    \item как обозначающего нечто другое, отличное от самого себя и
    \item как подчиненного нашему представлению об этом обозначаемом.~\autocite{gorny}
\end{enumerate*}
Аналогичные интуиции мы находим и в трудах ученых, стоявших у истоков современных семиотических проектов --
Ч.~Пирса, Ч.~Морриса, Ф.~де~Соссюра, Ю.~Лотмана, У.~Эко и заложивших фундамент
современного семиотического знания. Тезисно назовем только те, которые значимы для нашей работы.

\begin{enumerate}
\item Знаки, или точнее -- семиозис, возможны только в человеческой деятельности, и невозможны в
  живой и неживой природе. (Ч.~Моррис)
\item Знаки в человеческой деятельности функционируют не сами по себе и не произвольно,
  а образуют особый вид реальности, называемый семиотической реальностью.~\autocite{gorny}
  Эта реальность необходимо отличается от объективной реальности, под которой понимается.
  \begin{enumerate*}[label=\asbuk*)]
  \item Первая природа, т.е. естественная природа, данная нам в готовом виде и
  \item Вторая природа, результат человеческого труда -- культура.
  \end{enumerate*}
  Семиотическая реальность занимает промежуточное положение между ними,
  опосредуя наш контакт с действительностью. Гносеологически, знаки здесь решают две
  основные задачи: обозначение рассматриваемых явлений, их описание и шифрование конечных результатов
  наших исследований. Далее, полученный знаковый результат включается в коммуникацию
  с другими людьми, где он может быть проверен еще раз, и, в конечном итоге, стать отправной
  точкой для планирования качественного улучшения наличной материальной реальности.
  Таким образом, происходит переход из знаковой семиотической реальности в то, что мы назвали выше,
  второй природой, или, с другой стороны, в искусство, духовную культуру, науку и обучение.
  Скажем проще: утилитарно знаковая семиотическая реальность необходима для фиксирования результатов
  анализа первой природы и организации наших действий с реальностями второго плана.
\item Семиотическая реальность носит динамический характер. Знаки вступают в различные
  виды отношений с элементами семиозиса и объединяются в знаковые системы различной
  степени абстрактности, определяемой иерархически по степени удаленности от взаимодействия
  с первичной объективной реальности. Соответственно, анализ этих отношений, равно как и анализ
  самих знаков, с подачи Ч.~Морриса, происходит по следующим параметрам:
  \begin{enumerate*}[label=\asbuk*)]
  \item семантика, понимаемая как отношение знаков к обозначаемым ими предметам, свойствам, связям и качествам;
  \item синтаксис, или взаимоотношения знаков между собой и их связи внутри своей системы;
  \item прагматика, т.е. взаимоотношения знаков и человека, интерпретирующего их значение.
  \end{enumerate*}
  Примечательно, что именно прагматика, оформившаяся в самостоятельное направление прикладной
  семиотики, наиболее интенсивно и, в определенном смысле, стихийно развивается в современном мире.
  Прикладная семиотика занимается, в частности, языком жестов, исследованием общества и его
  составляющих в качестве семиотических объектов (социосемиотика), стратегиями маркетинга,
  технологиями рекламы и т.д. Она не опирается на широкую семиотическую теорию, поскольку
  таковая отсутствует, и представляет собой обширный и постоянно пополняющийся тезаурус
  самых разнообразных прикладных разработок и исследований семиотики всего на свете:
  политики, права, кулинарии, архитектуры, мебели, музыки, восприятия, визуальности и прочее.
  Именно в этом прикладном смысле, за неимением лучшего, мы будем пользоваться термином \emph{семиотика}
  в данной работе.
\end{enumerate}

\subsubsection{Семиотические параметры ментальности}\label{1.2}
Ментальность семиотична -- имеет знаковое выражение -- главным образом в двух смыслах.
Во-первых, она по определению рефлексивна, т.е. указывает на самою себя, на термин \emph{ментальность} и
как таковая имеет высокую степень абстрактности. Во-вторых, знаковое оформление ментальности
имеет ценность лишь в той степени, в какой оно адекватно отражает конкретные бессознательные
реалии жизни человека в обществе. Ниже мы продолжим разговор о словах и понятиях, поскольку
неразрывное единство языка и мировоззрения и представляет собой ментальность в ее целостности.~\autocite{lurie}
В этом смысле, ментальность есть сумма объясняющих моделей, устанавливающих причинно-следственную
связь между явлениями, и представлений о их функционировании. Соответственно понятийная система
народа неотделима от понятийной системы его языка.

Само понятие абстракции происходит от латинского слова abstractio, обозначающего отвлечение.
Как мы знаем, термин ввел в обиход Боэциий для удобства перевода греческого термина,
который использовал Аристотель для обозначения \emph{формирования образов реальности
(представлений, понятий, суждений) путем отвлечения и пополнения}.~\autocite{arnoldov}
В старой философской традиции различие между конкретным и абстрактным проводилось по линии
противопоставления чувственно данного многообразия единичных явлений и продуктов философской рефлексии.
~\autocite{bogomolov, sokolov} В современный философский дискурс категорию абстрактного
и конкретного ввел Гегель, предложивший понимать под конкретным расчлененную целостность,
а под абстрактным этап движения этого конкретного. Иначе: абстрактное -- по мысли Гегеля,
есть неразвернувшееся, неразвившееся, нераскрывшееся конкретное.\autocite{arnoldov}
Так, он сравнивает абстрактное и конкретное с желудем и дубом, с почкой и плодом~\autocite{gegel},
чтобы объяснить идею развития и становления самой человеческой мысли:
от приписывания абстрактного смысла, души, конкретным вещам (анимизм) -- к метафорическому приписыванию
образов идеальным сущностям, каковыми и являются абстрактные понятия. Такие образы,
по мысли Гегеля, ведут к нашему пониманию. В данном случае, образ дерева является одним
из самых древних и самых универсальным в мировой культуре~
\autocite{toporov} и, как таковое, вбирает в себя многие значимые для человека смыслы.
Поэтому достаточно \emph{присоединить} к нему абстрактное рассуждение, и оно непременно
обретает объяснительную силу. Отметим главное здесь. Абстрактное для понимания трактуется
через конкретное, и успешность понимания зависит от удачности выбора этого самого конкретного.
Если такое конкретное берется из некоего общекультурного фонда образов, метафор,
аллегорий, уже многократно истолкованных культурой, как, например, в случае с деревом --
по крайней мере индоевропейской, то субъект, воспринимающий адресованное ему сообщение,
(реципиент) воспримет послание максимально адекватно. Если же если конкретное --
образ, аллегория будут необычными (<<смелый образ>>), то перед нами уже будет не попытка
растолковать абстрактное для понимания, а другой феномен, относящий не к повседневной практике жизни,
а к области художественного творчества. Другой важный момент: противоположности, о которых
рассуждает Гегель как диалектик, всё-таки обнаруживаются исключительно в идеальном,
но никак не в реальном мире. В философском противопоставлении, равно как и в
лингвистической антонимии, само противопоставление возможно лишь в рационалистической системе
рассуждения по какому то одному из многих возможных признаков и отражают совсем не универсалию,
а всего лишь один отдельно взятый способ мышления. Скажем, лингвистически, всякий выделяемый
признак или даже действие абстрактны уже потому, что того, что могут характеризовать объекты
из принципиально различных классов, абстрагируясь от специфического. Именно поэтому среди них
очень легко образуются антонимические пары. Даже прилагательным, образованным от существительных,
обозначающих конкретные вещества или материал, можно без труда найти или образовать антонимы:
медовый пряник -- сметанный пряник и проч. Другими словами, противоположности (антонимы) --
или на семиотическом языке -- бинарные структуры -- как феномен придумываются
человеком и приписываются отраженной его в сознании действительности, и тем самым
характеризуют по преимуществу его способ мышления, но не саму действительность.
Следовательно, соответствующий закон диалектики распространяется только на мир идей,
но не на мир вещей. Он описывает закономерности мышления, но не закономерности или универсалии
объективной реальности. Иначе говоря, ментальность, как и любой абстрактный термин,
есть лишь идеальное представление в сознании исследователя о конкретных процессах и
явлениях наблюдаемых в жизни общества, семиотически замкнутое на себя. Ментальность
указывает на ментальность. В то же время естественно, что ментальность как знак,
функционально ориентирован на адекватное понимание динамики реальной жизни человека в обществе.
Августин Блаженный толковал это так: ``Signum \ldots est res praeter speciem, quam semsibus,
aliud aliquid ex es se faciens in cogitationem venire''. (``Знак есть некая вещь, представленная
нашим чувствам, но обозначающая в нашем постижении другое, а не только саму себя''.)~\autocite{gorny}
Поэтому обратимся ко второй семиотической составляющей ментальности. А именно к тому,
что она как знак обозначает, к означаемому и обозначаемому (референту), в терминах
Ф.~де~Соссюра. Начнем с содержательной стороны знака, нашего представления об означаемом.
В самом схематичном виде мы обозначили это проблему в \ref{mentality}.
Теперь нас интересует механизм возникновения представления как такого.

Разговор об означаемом целесообразно начать с разведения двух других терминов \emph{понятие} и
\emph{значение}, нередко используемых в качестве его синонимов. На первый взгляд различие между
ними исключительно ведомственное. Понятие -- это категория философии и логики, в то время как значения --
категория лингвистическая. Тем не менее, граница между ними обычно плохо соблюдается и в
результате оба термина употребляются наравне друг с другом. Одни исследователи отождествляют
понятие с лексическим значением слова, а другие принципиально отказываются признавать их связь
~\autocite{lingvo_dictionary}. Терминологическая путаница возникает уже на первой ступени анализа.
Так, в структуре понятия принято выделять три компонента: сигнификат, интенсионал и денотат,
между которыми существуют <<сложные и до конца не выясненные отношения>>~
\autocite{lingvo_dictionary}\autocite{apresyan}\autocite{artunova}\autocite{liuis}.
В тоже время в языкознании первые два компонента зачастую вообще не различаются.
Философский сигнификат здесь может именоваться \emph{языковым понятием}, \emph{наивным понятием},
\emph{десигнатом}, \emph{денотатом языковым}, \emph{концептом} и даже \emph{коннотацией}.
А коннотация, т.е. ассоциативные признаки, окружающие понятие, могут называться \emph{интенционалом}.
Наконец, лингвистический энциклопедический словарь резюмирует: ``понятие, лежащее в основе лексического
значения слова, характеризуется нечеткостью, размытостью границ''.~\autocite{lingvo_dictionary}
Понятно, что выстраивать самые различные цепочки отношений можно по-разному,
равно как и приписывать этим отношениям, как и самим звеньям цепи, самые различные признаки.
Как бы то ни было, для нас важно сейчас то, что как понятие, так и значение,
есть та идеальная сущность, которая стоит за материальной стороной знака.
За ней, в свою очередь, расположена также идеальная или же материальная реальность,
цепочка референций.

Референция, как мы знаем, есть функция лингвистического знака, отсылающая нас
к объекту внешнего мира, реальному или идеальному.~\autocite{general_lingvo}
Важно, что эта функция устанавливает связь с миром реальных вещей не напрямую,
а через <<внутренний>> мир идей, присущих той или иной культуре.
Следовательно, референт отсылает нас не к реальному, а к мыслимому объекту,
специфичному для того или иного типа национального сознания. Можно различать два вида
референтов: сильный и слабый.~\autocite{golovanskaya} Если сильный референт имеет свой прототип
в мире вещей, то слабый референт такой опоры в предметном мире не имеет.
В результате, слабый референт, в силу нечеткой ограниченности границ, сопровождается
характерной для абстрактных понятий синонимии, описывающей идеальные, неуловимые сущности.
Действительно, трудно определить, чем идея отличается от мысли или ментальность от
менталитета, образа мысли, образа мира, картины мира, но легко понять, чем мобильный телефон
отличается от пейджера. Слабому референту также присущи и достаточно подвижные
антонимические отношения. Например, совсем не очевидно, что является антонимом, или шире --
оппозицией рождения, смерть или жизнь, и ментальности -- энергетика, физические качества
человека, материалистичность или невежество. Абстрактное понятие живет вещественными коннотациями
и что еще важнее -- метафоричностью, поскольку метафорический концепт, позволяет связывать мир идей
с миром вещей и делать этот мир идей осязаемым.

Метафора есть способ унификации абстрактного и конкретного. Ментальность как термин,
в этом смысле, необходимо метафоричен. Более того, часто, оперируя абстрактными понятиями,
мы сознательно или подсознательно сравниваем или соединяем их с понятиями конкретными и
более понятными. Так, в обычном речевом употреблении ментальность при случае можно ``прививать'',
``разрушать'', ``переводить'', ``модернизировать'' , ``менять'', на ней можно ``паразитировать'',
по ней можно ``быть кирпичом'', она может быть ``освободительной'',
``террористической'', ``экономической'', ``антикапиталистической'', ``правовой'', ``языковой'',
может быть ``ментальностью жертвы'', ``общества'', ``орды'' и проч. При этом наше
воображение не только обыкновенно опирается на абстрактные сущности, но в случае необходимости
и с легкостью подменяет их. Типичный пример -- как легко мы принимаем виртуальный
мир кибернетической реальности за привычный, фактически за среду обитания,
тождественную естественной, поставив в своем воображении знаки равенства между обычными действиями --
получить, открыть, отправить, запомнить -- и идеальными абстрактными сущностями.~
\autocite{ivanov_virtual} М.К.~Голованивская называет это волшебной палочкой метафоры,
которая умножает ``наши интеллектуальные построения кратно совместимости их с реальностью осязаемой''.
По ее мысли, ассоциирование абстрактного понятия с конкретными осязаемыми предметами есть
единственная возможность унификации мира идей и мира вещей для создания однородного
реального мира. Она, в частности, пишет: ``Отождествляя абстрактные понятия с предметами материального мира,
мы ощущаем их как реальные сущности. Абстрактные понятия становятся одушевленными или неодушевленными,
активными или пассивными, \emph{хорошими}, то есть действующими в интересах человека, или
\emph{плохими}, то есть наносящими ему урон и пр.'' Метафора, взятая расширительно как непрямое
говорение~\autocite{lingvo_dictionary} служит решению этой задачи. Экстраполяция --
важное свойство метафоры, поскольку «она строится на основе реального сходства,
проявляющегося в пересечении двух значений, и утверждает полное совпадение этих значений.
Она присваивает объединению двух значений признак, присущий их пересечению.~\autocite{razlogova}
По всей видимости, вот это свойство метафоры и поддерживает поддержать слабый референт
абстрактных понятий и значений, делая его более осязаемым, и по этой причине, более
реальным и понятным. Более того, в последние годы к метафоре стали относиться как к ключу
к пониманию деятельности мышления и сознания, специфически национального видения мира или опять-таки
ментальности. Тот факт, что метафора есть едва ли не единственный способ понять и осознать
объекты высокой абстракции, позволяет совершить переход к по-настоящему концептуальной установке:
метафора дает ``эпистемический доступ'' к понятию как таковому.~\autocite{boyd}
При этом роль метафоры значительно расширяется. Уже в первой половине 20 века
Хосе Ортега-и-Гассет констатирует: ``От наших представлений о сознании зависит наша
концепция мира, а она в свою очередь предопределяет нашу мораль, нашу политику,
наше искусство. Получается, что все огромное здание вселенной, преисполненное жизни,
покоится на крохотном и воздушном тельце метафоры''.~\autocite{metaphors}

Далее, главное свойство метафоры -- переносить, отождествлять образы или различно
задуманные содержания, по всей видимости, является основным способом мифологического
переживания и мышления. В мифологическом мышлении и сознании, как отмечают многие
исследователи мифа З.~Фрейд, Г.~Ле~Бон, К.Г.~Юнг, Э.~Кассирер, Э.~Фромм, К.~Леви-Строс,
А.Ф.~Лосев, Ю.М.~Лотман и другие, реальное и идеальное, вещь и образ не расчленяются.
Стоящий за метафорой образ возникает объективно как устойчивая понятная связь. У абстрактного
понятия он приобретает конкретные черты, и само понятие начинает восприниматься как условно,
мифологически конкретное. И если Ю.М.~Лотман рассматривает такой тип семиозиса как специфический
процесс номинации, когда знак в мифологическом сознании становится аналогичным собственному имени\autocite{name_culture},
то М.К.~Голованивская вслед за В.А.~Успенским предлагает называть возникающие конкретные образы
вещественной коннотацией.\autocite{uspensky} Они существуют объективно, мотивируются объективно,
задают законы употребления и ассоциирования, являются фактами языка, но не речи. Будучи атрибутом
коллективного бессознательного, вещественная коннотация образует свое рода метафорический концепт
понятия и отражает специфику национальной ментальности.

Итак, что такое ментальность для нас. По опыту мы знаем, что у ментальности может быть разный субъект,
и одни и те же вещи могут по-разному пониматься разными людьми. В этом смысле, индивидуальную
систему ценностей каждого человека можно считать его менталитетом (ментальностью). Мы также знаем,
что менталитет (ментальность) может быть разной не только у отдельных людей, но и у различных
социальных групп, каждая из которых имеет свой способ решения жизненных проблем и свои стандарты поведения.
Скажем, современный городской житель, как утверждает Ж. Бодрияр, это образцовый гражданин общества
потребления, потребляющий вещи как знаки, в форме мифа потребляющий само потребление.\autocite{bodriyar_society}
Сельский житель, напротив, потребляет и рассматривает свое потребление преимущественно как удовлетворение
насущных потребностей. Различия между мировоззренческими системами часто концентрируются на двух моментах:
установлении жизненной причинно-следственной связи и ассоциативного сцепления, характерного для той или
иной системы восприятия. Так, деньги могут быть средством к существованию для студента, но инструмент инвестирования
для бизнесмена. И, как следствие, оформляется набор ассоциаций: для студента деньги -- сегодняшние удовольствия,
развлечение и радость, для бизнесмена -- опытно-конструкторские разработки, риск, испытание, и
нтеллектуальный драйв.

Наконец, сопоставительное изучение, пусть спорадическое, знаковых систем разных культур -- культурных кодов,
языка, ритуалов, традиций, кухни, архитектуры и проч. культурологами, антропологами, дипломатами,
политиками и переводчиками раскрывает специфику национальных ментальностей. Национальная ментальность
в данном случае -- это своего рода игра в ассоциации, устанавливающая связи между бaзoвыми понятиями
для этого народа, которые выглядят нетривиально с точки зрения другого народа.
Например, в американском менталитете есть такая установка как <<всё что со мной происходит -- зависит от меня>>.
А в российском часто можно встретить характерное <<они, правительство виноваты, милиция, врачи>>.
То есть кто угодно кроме нас самих. Русские обычно воспринимают государство как неуклюжую, непродуктивную,
агрессивную машину, с которой лучше не связываться. Напротив, для американцев государство это инструмент в
их руках, а не они в руках государства. Общество заведомо выше государства, а поскольку еще в 18 веке
пионеры-первопроходцы формировали законы в каждом городке сами, они считают, что и сегодня законы пишутся
ими, и государство не может им сказать, что делать. В этом смысле у них менталитет, прямо
противоположный российскому.

Обобщим сказанное:

Ментальность -- это система абстракций и стоящих за ними метафорическими образами,
которые регулируют нашу жизнь, правила поведения, и через которую мы соизмеряем себя
с нашими внутренними структурами, формирующие наше <<я>>.

В отличие от законченных и продуманных доктрин и идеологических конструкций, ментальность
рассеяна в культуре и в обыденном сознании и как таковая может изменяться.
Более того, очень часто она не осознается самими людьми и проявляется в их поведении
и речевых высказываниях как бы независимо от их желаний и воли.

Наконец, ментальность вбирает в себя не столько индивидуальные личностные установки,
сколько <<внеличную сторону общественного сознания, будучи имплицированы в языке и
других знаковых системах, в обычаях, традициях и верованиях.>>\autocite{gurevich_history}

Переходим к массовому обществу.

\subsubsection{Семиотические параметры массового общества}
\label{1.3}

В 1841 году американский философ Р.У.~Эмерсон опубликовал знаменитый очерк <<О доверии к себе>> (Self-Reliance),
где он, в частности, утверждает следующее: <<Society everywhere is in conspiracy against the manhood of
every one of its members. Society is a joint-stock company, in which the members agree, for
the better securing of his bread to each shareholder, to surrender the liberty and culture of the eater.
The virtue in most request is conformity. Self-reliance is its aversion. It loves not realities and
creators, but names and customs.>>\autocite{emerson1972self} В сокращенном парафразе это звучит примерно
так: <<Любое общество всегда находится в заговоре против человека. Конформизм считается добродетелью;
уверенность в себе - грехом. Общество любит не человека и жизнь, а имена и обычаи>>. Иначе говоря,
любое общество любит знаки и знаковые системы, и, если это общество массовое, то оно по определению
должно любить знаки и знаковые системы массового общества.
Известные теории массового общества (см.~\ref{1.1.1}) традиционно называют <<массовой>>
современную социальную организацию общества, при которой человек превращается в безликий
элемент бездушной социальной машины с ущербной нулевой ментальностью. Он ощущает себя жертвой
обезличенного социального процесса и занят главным образом тем, чтобы адекватно синхронизировать
свои потребности с потребностями этой машины. Основой массового общества при таком подходе принято
считать массовое производство стандартизированных вещей и соответствующее манипулирование взглядами,
вкусами, психологией <<одномерных>>, в терминах Г.~Маркузе, людей.

Первая целостная концепция <<массового общества>> оформилась в трудах испанского философа X.~Ортеги-и-Гассета\autocite{ortera_i_gasset_mass}.
По его мнению, <<неблагодарные массы>>, не обладая способностью управлять, <<рвутся к власти>>
и стремятся изгнать элиту из ее законных сфер -- политики и культуры. В этом он видит главную причину
социальных потрясений и катаклизмов прошлого столетия. Две основные теории массового общества
с политическим уклоном оформились в середине ХХ века: леворадикальная (Р.~Миллс)
и либерально-критическая (Э.~Фромм, Д.~Рисмен, К.~Маннгейм). Оба лагеря направили острие своей критики
на отчуждающую бюрократизацию власти, усиления контроля над личностью и конформизацию людей.
Наконец, в 1960-1970-е~гг. американские социологи Э.~Шилз и Д.~Белл в поисках компромисса объявили
наличные теории массового общества <<неоправданно критическими>> и попытались направить их в русло
официальной идеологии. Шилз, например, считал, что благодаря массовым коммуникациям <<народные массы>>
усваивают ценности и нормы, предлагаемые элитой, и тем самым общество движется в сторону
преодоления социальных антагонизмов. Не только адаптированные массы интегрируются в систему
социальных институтов <<массового общества>>, но и само общество в ХХ веке выступает как масса.
Эта концепция также была подвергнута резкой критике и на этом эволюция теоретического осмысления массового
общества затормозилась. Д.В.~Ольшанский видит причину в провале теоретического моделирования
массового общества в следующей ошибке: <<Дело в том, что само понятие <<масса>> было взято философами,
политологами и социологами из социальной психологии. Оно было сформулировано на основе
конкретных эмпирических наблюдений за ситуативно возникавшими (а значит, и ситуативно распадавшимися)
множествами людей и стихийными формами их поведения. Стихийные -- значит, неструктурированные, не
закрепленные, неформализованные. Главная особенность <<массы>> -- временность ее существования. ($\ldots$)
Наконец, масса возникает и функционирует на основе собственных внутренних, психологических,
а не внешних (социологических, философских и т. п.) закономерностей (\ldots)>>\autocite{book:olshansky}
В обществе роль масс становится заметной во время разрыва групповых связей и размывания
межгрупповых границ, когда общество переживает период социальных потрясений и деструктурируется.
В этом смысле, <<массы>> -- категория кризисного общества в нестабильное, <<смутное>> время, в периоды
социальных революций, крупномасштабных социальных реформ,
политических переворотов, войн и проч. Следовательно, говоря о массах, мы ведем речь о функциональном явлении,
базирующемся на временном психологическом единстве людей образующих массу. Такое единство внутренних психологических
характеристик можно назвать по-другому -- массовое сознание. И поскольку массу очень трудно определить морфологически,
попробуем взглянуть на нее со стороны массового сознания, как одного из видов общественного сознания и
наиболее реальной формой его практического воплощения.

Массовое сознание -- это специфический вид общественного сознания, присущий крупным
неструктурированным множествам людей. <<Массовое сознание определяется как совпадение в какой-то момент
(совмещение или пересечение) основных и наиболее значимых компонентов сознания большого числа весьма
разнообразных <<классических>> групп общества (больших и малых), однако оно несводимо к ним. Это новое качество,
возникающее из совпадения отдельных фрагментов психологии деструктурированных по каким-то причинам <<классических>>
групп.>>\autocite{book:olshansky} С точки зрения содержания, оно реализуется в массе индивидуальных сознаний,
хотя и не совпадает с каждым из них в отдельности. В этом смысле массовое сознание представляется надындивидуальным
и надгрупповым по содержанию, но индивидуальным по форме функционирования.

Далее, в отличие от группового сознания, для зарождения и функционирования массового сознания совсем не
обязательна совместная деятельность членов массовой общности. Запечатленные в нем знания, нормы, ценности и
образцы поведения вырабатываются спонтанно в процессе общения людей между собой, равно как и коллективного
восприятия социальной, политической и проч. информации, как, например, на политического митинге или на рок-концерте.
Из этого следует, что в основе массового сознания обычно лежит яркое эмоциональное переживание некоего
события или социальной проблемы, вызывающей всеобщую озабоченность. При этом интенсивность переживания
выступает как системообразующий фактор массового сознания. 3.~Фрейд в своем обличительном очерке о массовой
психологии утверждал: <<Масса импульсивна, изменчива и возбудима. Ею почти исключительно руководит
бессознательное.>>\autocite{freid_mass} Отсюда естественно вытекают основные характеристики массового сознания,
а именно: аморфность, заразительность, изменчивость, мозаичность, неоднородность, противоречивость и
размытость.\autocite{ashin}\autocite{flier}\autocite{prokudin}\autocite{heveshi}\autocite{hevishi2001tolpa}\autocite{streltsov1970}\autocite{dodonov1999}

Поскольку детальный анализ массового общества все-таки не входит в задачи данного исследования,
мы просто ограничимся здесь кратким резюме:
\begin{itemize}
\item[Первое.] В отличие от обычных устойчивых социальных групп, массы являются временными,
  ситуативными общностями. Хотя они разнородны по составу, они объединяются по признаку сопричастной
  значимости психических переживаний входящих в них людей. К главным различительным особенностям масс
  относятся:
  \begin{enumerate*}[label=\asbuk*)]
  \item их размеры,
  \item устойчивость существования во времени,
  \item степень компактности в социальном пространстве,
  \item уровень сплоченности или рассеянности,
  \item преобладание факторов организованности или стихийности в возникновении.
  \end{enumerate*}
\item[Второе.] Принципиальное отличие массы от традиционно выделяемых социальных групп,
  классов и слоев общества состоит в наличии особого, не организованного, самопорождающегося и
  плохо структурированного массового сознания -- обыденной разновидности общественного сознания,
  объединяющей представителей разных классических групп общими переживаниями. Такие переживания
  актуализируются при особых обстоятельствах. При этом само сознание может расширяться, вовлекая
  все новых людей из разных социальных групп, но может и сужаться, тем самым уменьшая размеры массы.
  Характерно, что общественное мнение и массовые настроения обычно принято считать ведущими макроформами
  массового сознания.
\end{itemize}
Таким образом, изменчивость границ массы и массового общества существенно затрудняет создание сколь
либо исчерпывающей типологии массового сознания и сопутствующей ему ментальности. В таких случаях,
как это принято в науке, единственным выходом становится моделирование изучаемого явления, здесь --
построение апроксмированных моделей массового сознания. При этом важно помнить, что любая модель
необходимо условна, абстрактна и, следовательно, семиотична по определению, включенной в семиотическую реальность.
И как таковая, такая знаковая система прежде всего свидетельствует о ментальности исследователя (см.~\ref{1.2}) и
только затем, опосредованно говорит о моделируемом явлении.
Чем сильнее упрощение и чем больше описываемая модель не похожа на исходный референт объективной реальности,
тем обширнее культурологическая составляющая при ее расшифровке. По мере повышения абстрактности используемых
знаков повышается и общий уровень абстракции всей создаваемой из них знаковой системы. Наконец, по мере
повышения степени абстрактности знака и знаковой системы усиливается и ригидность формулируемых правил,
регулирующих их использование. В результате, любое моделирование реальности шифруется в знаках метаязыка и
оформляется в виде заданных метаязыковых параметров и установок. Одной из таких важных, но потенциально опасных
установок является прямая проекция предлагаемой модели в целях ее верификации на реальную жизнь общества.
Опасность заключается в аналоговом отожествлении, метафоризации и мифологизации, имманентных абстрактному
построению. Для наглядности рассмотрим это на примере модели массового общества, предложенной М.И.~Найдорфом\autocite{ocherki}.

За исходную посылку берется здесь тезис о том, что современное общество является массовым по определению.
Жизнь в современном массовом обществе носит коллективный характер и <<основывается на разнообразных технологиях
массообразования, взятых вместе со средствами символизации массовых общностей.>>\autocite{ocherki}
К таковым прежде всего относятся, по мнению исследователя, политическая и торговая реклама. Приемлемость
и объединяющую разумность массового способа жизни находят свое отражение в системе новых массовых
ценностей, таких как политкорректность и толерантность, идеалы равенства и демократии, философия комфорта
и потребления, идеология развлечений, и проч.

Другой исходной посылкой служит утверждение о том, что массовое общество <<нуждается в иначе устроенном
человеке и производит его>>. Как и в каждой культуре, известной нам из истории, в
массовой культуре необходимо присутствует представление о <<своём» человеке>>.\autocite{ocherki} Такое представление,
объясняет Найдорф, задается его <<местом>> в этом мире, ограниченного культурно допустимым и возможным поведением.\autocite{naydof}
Чье это представление, и каким образом массовое общество вообще может <<производить>>
человека автор пока не поясняет. Вместо этого он развивает свой тезис дальше и аксиоматически
строит обратное утверждение: <<Если верно, что тип массовой культуры полагает соответствующий ему тип человека,
то верно и обратное: современный человек не может жить иначе как в массовом обществе>>.\autocite{ocherki}
Массовый человек воспроизводит массовый способ жизни в своей повседневной практике, <<реализуя возможности,
преимущества и ограничения своего <<места>> всей совокупностью средств, очерченных представлениями массовой
культуры о массовом человеке>>.\autocite{ocherki} Из чего мы узнаем, что субъектом представлений о массовом человеке --
как это не парадоксально выглядит -- является сама массовая культура. Соответственно, полагает Найдорф,
человек массовой культуры или <<массовый человек>> -- это такой человек, который добровольно считает
массовую культуру своей. Наконец, главное: массовый человек -- <<это человек, действующий так, как его
мотивирует массовое сознание>>.\autocite{ocherki}

Разумеется, в такой степени методологически редуцированный человек, производимый массовым обществом и
мотивируемый массовым сознанием, живет полноценной жизнью лишь в рамках абстрактной знаковой модели,
которая, как мы говорили, дает лишь опосредованное представление о предмете. Вести речь о ментальности
здесь, по-видимому, не имеет смысла, так как она представлена лишь одним субъектом: исследователем.
Нас, в свою очередь, интересует совсем не абстрактная модель человека, сознания и общества, а реальный человек,
реальное сознание и реальное общество. По этой причине, термин \emph{массовое общество} для нас, главным образом,
тавтология, прием смещения акцента и привлечения внимания к стихийным, неструктурированным массовым процессам,
объединениям, со-обществам и общностям внутри общества, базирующихся на временном психологическом единстве.
Напротив, мы будем исходить из того классического представления, что человек -- это биосоциальное существо,
развивающиеся по биологическим законам, но отличающиеся от других высших животных членораздельной речью и
сознанием, способностью производить орудия труда и созидательной активностью. Адаптационный <<стадный инстинкт>>
присущ ему в точно такой же мере, как и многим видам животных, живущих в группах, стадах, стаях и проч.

В узком смысле, стадный инстинкт определяет взаимоотношения особей одного вида в группах, расселение молодых животных,
отделившихся от взрослых и т.д. В социальном смысле, в человеческой истории, по всей видимости, масс никогда бы не
было, если бы не возникало особой потребности объединятся в такие массы. Таким образом из внутренней потребности
человека рождается особый мотив -- объединяться с подобными себе <<ради самосохранения, достижения: каких-то выгод
или некоторого внутреннего состояния.>>\autocite{book:olshansky} 3.~Фрейд формулировал это так: <<Отдельный человек чувствует
себя незавершенным, если он один>>\autocite{freid_mass} И если самосохранение и достижение выгод все-таки не
нуждаются в специальном разъяснении, то достижение некоторых внутренних состояний требует расшифровки.
Соответственно, ниже мы будем вести разговор о тех внутренних психических внезнаковых реалиях, которые,
собственно и составляют основное содержание массовой ментальности как таковой.

%% TODO: где там же то?
Речь идет о неосознаваемой потребности в эмоционально-аффективных состояниях, как положительных,
так и отрицательных, требующих от человека пребывания в массе или своими действиями способствовать ее возникновению.
Внешне, все происходит как бы стихийно, само собой, и масса возникает независимо от людей. Психологически, однако,
все выглядит иначе. Д.В.~Ольшанский, например, разбирает частный случай массы -- банальной <<случайной>> толпы
прохожих-зевак, обсуждающих столкновение двух автомобилей. Любопытно, что прохожих останавливают эти самые водители и
<<это их отрицательно-эмоциональный заряд заставлял их ругаться, обвинять друг друга, кричать, привлекая внимание к
случившемуся -- вплоть до прямых призывов ``быть свидетелем''.>> (там же) Мотивация водителей вполне понятна.
Во-первых, уберечь себя от ответственности. Затем, получить хотя бы какую-то выгоду -- определить кто будет платить
штраф, оплачивать ремонт и проч. Наконец, главное -- облегчить свое эмоциональное состояние, уменьшить стресс от
произошедшего. Иначе говоря, психологически, в основе формирования массы лежат индивидуальная потребность в
самоидентификации с большим количеством людей для регуляции своих эмоциональных состояний. При этом важно, как
отмечают психологи, что <<эта потребность обычно актуализируется в тех случаях, когда речь идет о сильных
эмоциональных состояниях, с которыми сам индивид справиться не может. Тогда ему и необходима особая идентификация --
не психологическое отождествление себя с другими людьми, а физическое соединение с ними.>> (там же) Добавим лишь,
что эти потребность в самоидентификации с массой может иметь разную полярность -- как разрядки определенных
(обычно резко отрицательных) эмоциональных состояний, так и усиления состояний связанных с эмоциями положительности,
наблюдаемых, например, в ритуалах и традициях народных гуляний и празднований, семейных торжествах, спортивных
состязаниях, предпраздничных распродажах, сообществах в социальных сетях и т.д.

Таким образом, индивид предшествует массе. Он присоединяется к ней инстинктивно и добровольно.
Потребность в массе мотивируется, в свою очередь, потребностью стабилизировать свои эмоциональные состояния,
усиливая положительные и уменьшая отрицательные эмоции. Другой вопрос, что удовлетворение этой потребности
ведет к временной нейтрализации рациональных компонентов психики, таких как снижение критичности восприятия
и чувства собственного <<я>>. Происходит некоторая временная деиндивидуализация человека. С психологической
точки зрения, потребность раствориться в массе, хотя и регрессивная по характеру, наравне с потребности
быть личностью: развивать индивидуальное сознание и самоопределение -- вполне естественна для человеческой психики.
Э.~Фромм, например, метафорически описывает эту тенденцию как <<бегство от свободы>>. Уставая от необходимости жить
все более рационально, от индивидуальной свободы и, пожалуй, самое главное, от связанной с ней
индивидуальной ответственности, человек добровольно бежит в массу.\autocite{fromm} Там, при отсутствии какого-либо
сопротивления с его стороны, незаметно для него меняется его психика. С течением времени, по мере того
как приведшая его в массу потребность иссякает. Человек выходит из массы -- <<с новым опытом, и с теми
или иными, преходящими или уже хроническими, изменениями психики>>.\autocite{book:olshansky}

Мы говорим в данном случае о так называемых <<естественных>> массах, возникающих стихийно и самопроизвольно.
От них принять отличать <<искусственные>> массы, такие как, например, армия или церковь.\autocite{freid_mass}

Сделаем несколько предварительных обобщений, прежде чем перейти к следующему разделу.
\begin{enumerate}
    \item Массовое общество понимаемое как массовая общность -- это функциональном явление внутри общества,
    базирующееся на временном психологическом единстве людей образующих массу. Массовое сознание,
    присущее массовым общностям, характеризуется неоднородностью, аморфностью, изменчивостью,
    противоречивостью и размытостью.
    \item Ментальность в массовой общности имеет два плана: план выражения и план содержания.
    План выражения получает знаковое оформление преимущественно в категориях абстрактных понятий
    национального языка и иных невербальных знаковых системах культуры по принципу аналогово отождествления.
    План содержания включает в себя дознаковые бессознательные реалии психической жизни человека, внутренние
    потребности и мотивы, побуждающие его соединяться с себе подобными.
    \item В поведенческом плане ментальность в массовой общности приобретает следующие черты:
    \begin{enumerate*}[label=\asbuk*)]
        \item повышенная эмоциональность в восприятии всего, что человек видит и слышит;
        \item повышенная внушаемость и уменьшение степени критичности к самому себе и способности критического
        осмысления воспринимаемой информации;
        \item подавление чувства ответственности за собственное поведение.
    \end{enumerate*}
\end{enumerate}

Иначе говоря, в действиях человека в массе преобладает аффективно-бессознательное поведение,
в то время как сознательная личность исчезает. Но с исчезновением сознательности также происходит и
мутация взаимоотношения человека со знаком. Человек в массе больше не производит знаки.
Он их потребляет, или точнее -- некритично руководствуется ими. Рассмотрим этот вопрос подробнее.
