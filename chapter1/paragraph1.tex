\section{Генезис логотипа}\label{chapter1}

\subsection{Ментальность массового общества: семиотический аспект}\label{1}
Развитие рыночной экономики, индустриализации, демократизация технологий в ХХ веке закономерно привели к этапу развития цивилизации, который принято называть массовым обществом и массовой культурой. На сегодняшний день в культурологии нет единого мнения относительно того, что следует считать массовой культурой и где проходят ее временные границы. В современном русском языке слово «массовая культура», как правило, имеет негативную коннотацию и употребляется скорее как оксюморон, в котором отрицательное значение слова «массовый» «гасит» положительное значение слова «культура» \autocite{elistratov2012}. При таком подходе «массовая культура» - это циничное манипулирование гомогенизированными массами и игра на их первичных инстинктах. Между тем, массовая культура окружает нас со всех сторон, она растёт по экспоненте, неудержимо проникает во все уголки нашей жизни. Претензии к массовой культуре традиционно ограничиваются критикой сравнительно незначительной ее части – массового искусства. В значительной же для жизни каждого человека сфере культуры, названной Ф. Броделем структуры повседневности, большинство форм и предметов массовой культуры не только не  осуждаются, но всячески приветствуются. В аспекте «структур повседневности» массовая культура способна обеспечить большинству населения «достойную жизнь» - возможность красиво одеваться, вкусно питаться, путешествовать, обустраивать свой дом, всемерно удовлетворять все возрастающие потребности. Понятно, что такая культура получает всестороннюю поддержку на всех уровнях общества. Именно в модусе «структура повседневности» массовой культуры функционирует логотип, семиотически организующий пространство массовой потребительской деятельности людей и визуально формирующий символическую среду нашей феноменально-актуальной телесности. Последовательно рассмотрим термины: \emph{массовое общество}, \emph{ментальность},
\emph{семиотика}\autocite{society}.

\subsubsection{Рабочие границы терминов \emph{массовое общество}, \emph{ментальность}, \emph{семиотика} в культурологии}\label{1.1}
\paragraph{Массовое общество}\label{1.1.1}

В культурологии массовое общество принято называть обществом потребления, «в котором “иметь” равносильно тому, чтобы “быть”»\autocite[][199]{razlogov2005}. Массовое общество – закономерный продукт масштабных социально-политических  и производственных процессов в западной и отечественной культуре ХХ века. Главный герой столетия – “средний”, “массовый” человек, оказавший серьёзное, формообразующее воздействие «на все процессы и перемены, от производства до войн, от системы ценностей и социальной мифологии до спорта и досуга» \autocite{levada2001}. Первым понятие массы ввел в научный дискурс французский психолог Г. Лебон, рассматривавшего толпу (массу) в качестве психологического феномена, возникающего при прямом взаимодействии индивидов, независимо от их социального статуса, рода деятельности или национальности. По его мнению, в толпе формируется некое социально-психологическое единство массы, наделенное общими чувствами, взаимовнушением, высоким уровнем энергетики, нейтрализующими сознательную личность. К более поздним серьезным и разноплановым теоретическим исследованиям массового общества, написанным главным образом в 20-60-е годы ХХ века, можно было бы отнести работы Габриеля Тарда, Георга Зиммеля,
Макса Вебера, Карла Маннгейма, Эмиля Ледерера, Герберта Блумера, Дэвида Рисмена,
Даниеля Белла, Хорхе Ортеги-и-Гассета, Ханны Арендт, Льюиса Мамфорда, Фридриха Юнгера,
Эриха Фромма, Герберта Маркузе, Уильяма Корнхаузера, Жана Бодрийяра, Эдварда Шилза и др. Д.В.~Ольшанский в своем историческом экскурсе\autocite[][15]{olshansky}
перечисляет семь таких подходов:
\begin{enumerate}[label={\arabic*)}]
    \item как толпа (традиции Г.~Лебона);
    \item как публика (последователи Г.~Тарда);
    \item как гетерогенная аудитория, противостоящая классам и относительно гомогенным
    группам (Э.~Ледерер и Х.~Арендт, например, считали массы продуктом дестратификации
    общества, своего рода \emph{антиклассом});
    \item как \emph{агрегат людей, в котором не различаются группы или индивидуумы} (У.~Корнхаузер);
    \item как уровень некомпетентности, как снижение цивилизации (X.~Ортега-и-Гассет);
    \item как продукт машинной техники и технологии (Л.~Мамфорд);
    \item как \emph{сверхорганизованное} (К.~Маннгейм) бюрократизированное общество,
    в котором господствуют тенденции к униформизму и отчуждению.
\end{enumerate}

В сущности, список можно с легкостью продолжить, добавив, например, постмодернистскую
трактовку Ж.~Бодрийяра, к чьим работам мы будем неоднократно обращаться позднее.
Бодрийяр, как известно, ввел альтернативный термин для описания массы (\emph{молчаливого большинства}) --
\emph{общество потребления}, эволюции массового общества на новом историческом этапе.
В книге <<В тени молчаливого большинства, или Конец социального>>~\autocite{book:bodriyar} он, в частности,
утверждает, что масса -- это разновидность \emph{черной дыры}, падение в самих в
себя, которое не обладает ни атрибутом, ни предикатом, ни качеством, ни референцией.
Масса, по Бодрийяру, радикально неопределима и не имеет социологической реальности.
Она не является ни субъектом истории, ни ее объектом.

В отечественной гуманитарной традиции проблемы массового общества с разных сторон освещались П.С. Гуревичем, А. Н. Николюкиным, К.Э. Разлоговым, Л.Г. Иониным, Ю.А. Левадой, Г. Л. Тульчинским, М.И. Найдорфом, А.И. Ильиным, Е.Г. Соколовым и др. Так, А. И. Ильин считает, что термин <<массовое общество>> содержит в себе внутренне противоречие, По его мнению, существовать может только либо масса, либо общество, но никак не массовое общество. В тоже время, как масса, так и общество «находят свое одновременное существование (сосуществование) как субъекты массовой культуры»\autocite[][79]{ilin2010}. Соответственно, массовое общество, полагает Ильин, представляет собой «социальное пространство, содержащее внутри себя как массу бессубъектных индивидов, так и общество как находящуюся на более высоких ступенях культурного развития совокупность субъектов»\autocite[][79]{ilin2010}.

В данной работе мы, во-первых, исходим из того представления, что массовое общество – это такая форма общности, в которой «отчетливо проявился “человек массы” как носитель особого типа сознания – массового сознания»\autocite[][289]{kagan2007}. Появление человека массы, как убедительно доказывал Ортега-и-Гассет в “Восстании масс”, следует рассматривать как закономерный результат предшествующего развития новоевропеской социокультурной традиции. Во-вторых, ниже мы не делаем специального различия между терминами “массовое общество” и “массовая культура” и используем их взаимодолнительно. Вслед за Е.Г. Соколовым\autocite{sokolov2002}, мы понимаем под массовой культурой такую культурную ситуацию, которая соответствует рыночной форме социального устроения общества\autocite[][20]{edoshina2000}. Это культура «в присутствии Масс»\autocite[][289]{kagan2007}.

\paragraph{Ментальность}
\label{mentality}
Ментальность принято считать основной характеристикой феномена сознания\autocite[][52]{edoshina2000}. Более широко: в отечественной теоретической культурологии под ментальностью предлагается понимать «глубинный уровень коллективного и индивидуального сознания, включающий и бессознательное; относительно устойчивая совокупность и предрасположенностей индивида или социальной группы воспринимать мир определенным образом»\autocite[][414]{razlogov2005}.

Понятие о коллективных представлениях детализировалось и получило развитие в трудах Э. Фромма~\autocite{book:davydov}, Г. Бутуля,
Л. Февра~\autocite{book:febvre} и М. Блока, Э. Ле Руа, Ж. Ле Гоффа,
М. Фуко~\autocite{book:arch}, К. Г. Юнга~\autocite{book:yung},
Й. Хейзинги~\autocite{book:heizenga}, Г.-В. Гетса, Ф. Броделя  и др. Среди российских ученых, внесших вклад в изучение ментальности следует отметить труды М. М. Бахтина~\autocite{book:tamarchenko}, А. Я. Гуревича, Ю. Л. Бессмертного, А. Л. Ястребицкой, В. П. Даркевичa\autocite{levit1998}, А.С. Мыльникова\autocite{milnikov1996}, В. П. Визгина, Н. В. Кондакова, Л.В. Санжеевой\autocite{sanjeeva2011} и др.

В самом общем виде \emph{ментальность} культурологически определяется как
видение мира и его восприятие, образ мысли и нормы поведения, в которых сочетаются
сознательные и бессознательные моменты. Ж. Ле Гофф, в частности, подчеркивает неизбежно коллективный характер ментальности, т.е. того общего, что «индивид разделяет с другими людьми своего времени … на уровне повседневного автоматизма поведения»\autocite{sharte2004}. Ниже мы будем отталкиваться от родственного определения ментальности: «ментальность есть сама психология, взятая в контексте социальных условий; это обыденность, средний человек и способы чувствования, мышления, силы, формирующие привычки, отношения, безличный культурный контекст»\autocite{online:kulturolog}.

Иначе говоря, в самой идее ментальности синкретично соединяются характеристики психологической и непсихологической (культурной и социальной) реальностей. При этом ментальность, так или иначе, находит свое выражение в повседневной жизни людей. А поскольку повседневностью живет все-таки каждый человек, то категория «повседневности» в данном случае будет служить краеугольным камнем в нашем исследовании. По Ф. Броделю, повседневность внеисторична, расплывчата, неопределенна и неопределима. Тем не менее, к ее организующим принципам можно отнести, во-первых, иерархичность (верхи и низы). Низовым уровням присущи стабильность, консерватизм, устойчивость к переменам. Элита, верхи общества, определяют ценностные ориентиры и создают иерархические порядки. Во-вторых, понятие стиля или стилизации. При этом стиль, как некая культурная целостность, влияет как на материально-предметный мир, так и на поведенческие нормы, проявлясь в речи, манерах и пр. В-третьих, категорию времени и пространства. Логотип, как мы покажем, в полной мере соответствует критериям повседневности и как таковой обладает мощным потенциалом влияния на ментальность стихийных массовых сообществ, социальных групп и общества в целом.

\paragraph{Семиотика}\label{1.1.3}

Хотя литература по семиотике обширна, ее институциональный статус по-прежнему не определен.~\autocite{sirotkin}
Более того, современная семиотическая теория на данный момент также не имеет ни
единой методологии, ни общей теории.~\autocite{gorny}\autocite{gasparov}. Консенсус есть только в понимании семиотики как науки о знаках и знаковых системах. Знак -- это инструмент (или орудие),
с помощью которого решаются две основные задачи:
\begin{inparaenum}[\itshape 1\upshape)]
    \item быстрого получения, передачи, переработки и
    \item надежного хранения знаний/информации в обществе.
\end{inparaenum} Или же, в определении Ю.М.~Лотмана: ``Знак -- это материально выраженная замена предметов, явлений, понятий в процессе обмена информацией в коллективе.''~\autocite{wiki:symbol}
Здесь, по-видимому, следует сделать важное уточнение. Знаком заменяется не столько сам
обозначаемый предмет, взятый в своей целокупности, сколько его отдельные свойства или
качества, метонимически представляющие его целостность в индивидуальном и коллективном сознании. Отсюда естественным образом вытекает важнейшая ценностная образующая знака -- его значение. Помимо указания на предмет, значение знака есть не что иное, как наше представление об означаемом предмете, коллективно признанное и индивидуально осознанное. Более подробно мы рассмотрим этот вопрос в последующих разделах.

Далее, знак, как отмечал еще Августин Аврелий, действует по принципу двойственного подчинения \begin{enumerate*}[label=\asbuk*)]
    \item как обозначающего нечто другое, отличное от самого себя и
    \item как подчиненного нашему представлению об этом обозначаемом.~\autocite{gorny}
\end{enumerate*} Аналогичные интуиции мы находим и в трудах ученых, стоявших у истоков современных семиотических проектов -- Ч.~Пирса, Ч.~Морриса, Ф.~де~Соссюра, Ю.~Лотмана, У.~Эко и заложивших фундамент современного семиотического знания. Тезисно назовем только те, которые значимы для нашей работы.
\begin{enumerate}
\item Знаки, или точнее -- семиозис, возможны только в человеческой деятельности, и невозможны в
  живой и неживой природе. (Ч.~Моррис)
\item Знаки в человеческой деятельности функционируют не сами по себе и не произвольно,
  а образуют особый вид реальности, называемый семиотической реальностью.~\autocite{gorny}
  Эта реальность необходимо отличается от объективной реальности, под которой понимается.
  \begin{enumerate*}[label=\asbuk*)]
  \item Первая природа, т.е. естественная природа, данная нам в готовом виде и
  \item Вторая природа, результат человеческого труда -- культура.
  \end{enumerate*}
  Семиотическая реальность занимает промежуточное положение между ними,
  опосредуя наш контакт с действительностью. Гносеологически, знаки здесь решают две
  основные задачи: обозначение рассматриваемых явлений, их описание и шифрование конечных результатов
  наших исследований. Далее, полученный знаковый результат включается в коммуникацию
  с другими людьми, где он может быть проверен еще раз, и, в конечном итоге, стать отправной
  точкой для планирования качественного улучшения наличной материальной реальности.
  Таким образом, происходит переход из знаковой семиотической реальности в то, что мы назвали выше,
  второй природой, или, с другой стороны, в искусство, духовную культуру, науку и обучение.
  Скажем проще: утилитарно знаковая семиотическая реальность необходима для фиксирования результатов
  анализа первой природы и организации наших действий с реальностями второго плана.
\item Семиотическая реальность носит динамический характер. Знаки вступают в различные
  виды отношений с элементами семиозиса и объединяются в знаковые системы различной
  степени абстрактности, определяемой иерархически по степени удаленности от взаимодействия
  с первичной объективной реальности. Соответственно, анализ этих отношений, равно как и анализ
  самих знаков, с подачи Ч.~Морриса, происходит по следующим параметрам:
  \begin{enumerate*}[label=\asbuk*)]
  \item семантика, понимаемая как отношение знаков к обозначаемым ими предметам, свойствам, связям и качествам;
  \item синтаксис, или взаимоотношения знаков между собой и их связи внутри своей системы;
  \item прагматика, т.е. взаимоотношения знаков и человека, интерпретирующего их значение.
  \end{enumerate*}
  Примечательно, что именно прагматика, оформившаяся в самостоятельное направление прикладной
  семиотики, наиболее интенсивно и, в определенном смысле, стихийно развивается в современном мире.
  Прикладная семиотика занимается, в частности, языком жестов, исследованием общества и его
  составляющих в качестве семиотических объектов (социосемиотика), стратегиями маркетинга,
  технологиями рекламы и т.д. Она не опирается на широкую семиотическую теорию, поскольку
  таковая отсутствует, и представляет собой обширный и постоянно пополняющийся тезаурус
  самых разнообразных прикладных разработок и исследований семиотики всего на свете:
  политики, права, кулинарии, архитектуры, мебели, музыки, восприятия, визуальности и прочее.
  Именно в этом прикладном смысле мы будем пользоваться термином \emph{семиотика}
  в данной работе.
\end{enumerate}

\subsubsection{Семиотические параметры ментальности в массовом обществе}\label{1.2}
Ментальность семиотична -- имеет знаковое выражение -- главным образом в двух смыслах.
Во-первых, она по определению рефлексивна, т.е. указывает на самою себя, на термин \emph{ментальность} и
как таковая имеет высокую степень абстрактности. Во-вторых, знаковое оформление ментальности
имеет ценность лишь в той степени, в какой оно адекватно отражает конкретные бессознательные
реалии жизни человека в обществе. Особая роль в этом процессе отводится метафоре.

Метафора есть способ унификации абстрактного и конкретного. Ментальность как термин,
в этом смысле, необходимо метафоричен. Более того, часто, оперируя абстрактными понятиями,
мы сознательно или подсознательно сравниваем или соединяем их с понятиями конкретными и
более понятными. Так, в обычном речевом употреблении ментальность при случае можно ``прививать'',
``разрушать'', ``переводить'', ``модернизировать'' , ``менять'', на ней можно ``паразитировать'',
по ней можно ``быть кирпичом'', она может быть ``освободительной'',
``террористической'', ``экономической'', ``антикапиталистической'', ``правовой'', ``языковой'',
может быть ``ментальностью жертвы'', ``общества'', ``орды'' и проч. При этом наше
воображение не только обыкновенно опирается на абстрактные сущности, но в случае необходимости
и с легкостью подменяет их. Типичный пример -- как легко мы принимаем виртуальный
мир кибернетической реальности за привычный, фактически за среду обитания,
тождественную естественной, поставив в своем воображении знаки равенства между обычными действиями --
получить, открыть, отправить, запомнить -- и идеальными абстрактными сущностями
\autocite{ivanov_virtual}. М.К.~Голованивская называет это волшебной палочкой метафоры,
которая умножает ``наши интеллектуальные построения кратно совместимости их с реальностью осязаемой''.
По ее мысли, ассоциирование абстрактного понятия с конкретными осязаемыми предметами есть
единственная возможность унификации мира идей и мира вещей для создания однородного
реального мира. Она, в частности, пишет: ``Отождествляя абстрактные понятия с предметами материального мира,
мы ощущаем их как реальные сущности. Абстрактные понятия становятся одушевленными или неодушевленными,
активными или пассивными, \emph{хорошими}, то есть действующими в интересах человека, или
\emph{плохими}, то есть наносящими ему урон и пр.'' Метафора, взятая расширительно как непрямое
говорение~\autocite{lingvo_dictionary} служит решению этой задачи. Экстраполяция --
важное свойство метафоры, поскольку «она строится на основе реального сходства,
проявляющегося в пересечении двух значений, и утверждает полное совпадение этих значений.
Она присваивает объединению двух значений признак, присущий их пересечению.~\autocite{razlogova}
По всей видимости, вот это свойство метафоры и поддерживает поддержать слабый референт
абстрактных понятий и значений, делая его более осязаемым, и по этой причине, более
реальным и понятным. Более того, в последние годы к метафоре стали относиться как к ключу
к пониманию деятельности мышления и сознания, специфически национального видения мира или опять-таки
ментальности. Тот факт, что метафора есть едва ли не единственный способ понять и осознать
объекты высокой абстракции, позволяет совершить переход к по-настоящему концептуальной установке:
метафора дает ``эпистемический доступ'' к понятию как таковому\autocite{boyd}.
При этом роль метафоры значительно расширяется. Уже в первой половине 20 века
Хосе Ортега-и-Гассет констатирует: <<От наших представлений о сознании зависит наша
концепция мира, а она в свою очередь предопределяет нашу мораль, нашу политику,
наше искусство. Получается, что все огромное здание вселенной, преисполненное жизни,
покоится на крохотном и воздушном тельце метафоры>>\autocite{metaphors}.

Далее, главное свойство метафоры -- переносить, отождествлять образы или различно
задуманные содержания, по всей видимости, является основным способом мифологического
переживания и мышления. В мифологическом мышлении и сознании, как отмечают многие
исследователи мифа З.~Фрейд, Г.~Ле~Бон, К.Г.~Юнг, Э.~Кассирер, Э.~Фромм, К.~Леви-Строс,
А.Ф.~Лосев, Ю.М.~Лотман и другие, реальное и идеальное, вещь и образ не расчленяются.
Стоящий за метафорой образ возникает объективно как устойчивая понятная связь. У абстрактного
понятия он приобретает конкретные черты, и само понятие начинает восприниматься как условно,
мифологически конкретное. И если Ю.М.~Лотман рассматривает такой тип семиозиса как специфический
процесс номинации, когда знак в мифологическом сознании становится аналогичным собственному имени\autocite{name_culture},
то М.К.~Голованивская вслед за В.А.~Успенским предлагает называть возникающие конкретные образы
вещественной коннотацией\autocite{uspensky}. Они существуют объективно, мотивируются объективно,
задают законы употребления и ассоциирования, являются фактами языка, но не речи. Будучи атрибутом
коллективного бессознательного, вещественная коннотация образует свое рода метафорический концепт
понятия и отражает специфику национальной ментальности.

Итак, с семиотической точки зрения, у ментальности может быть разный субъект,
и одни и те же вещи могут по-разному пониматься разными людьми. В этом смысле, индивидуальную
систему ценностей каждого человека можно считать его менталитетом (ментальностью). Мы также знаем,
что менталитет (ментальность) может быть разной не только у отдельных людей, но и у различных
социальных групп, каждая из которых имеет свой способ решения жизненных проблем и свои стандарты поведения.
Скажем, современный городской житель, как утверждает Ж. Бодрийяр -- это образцовый гражданин общества
потребления, потребляющий вещи как знаки, в форме мифа потребляющий само потребление\autocite{bodriyar_society}.
Сельский житель, напротив, потребляет и рассматривает свое потребление преимущественно как удовлетворение
насущных потребностей. Различия между мировоззренческими системами часто концентрируются на двух моментах:
установлении жизненной причинно-следственной связи и ассоциативного сцепления, характерного для той или
иной системы восприятия. Так, деньги могут быть средством к существованию для студента, но инструмент инвестирования
для бизнесмена. И, как следствие, оформляется набор ассоциаций: для студента деньги -- сегодняшние удовольствия,
развлечение и радость, для бизнесмена -- опытно-конструкторские разработки, риск, испытание, и
нтеллектуальный драйв.

Наконец, сопоставительное изучение, пусть спорадическое, знаковых систем разных культур -- культурных кодов,
языка, ритуалов, традиций, кухни, архитектуры и проч. культурологами, антропологами, дипломатами,
политиками и переводчиками раскрывает специфику национальных ментальностей. Национальная ментальность
в данном случае -- это своего рода игра в ассоциации, устанавливающая связи между бaзoвыми понятиями
для этого народа, которые выглядят нетривиально с точки зрения другого народа.
Например, в американском менталитете есть такая установка как <<всё что со мной происходит -- зависит от меня>>.
А в российском часто можно встретить характерное <<они, правительство виноваты, милиция, врачи>>.
То есть кто угодно кроме нас самих. Русские обычно воспринимают государство как неуклюжую, непродуктивную,
агрессивную машину, с которой лучше не связываться. Напротив, для американцев государство это инструмент в
их руках, а не они в руках государства. Общество заведомо выше государства, а поскольку еще в 18 веке
пионеры-первопроходцы формировали законы в каждом городке сами, они считают, что и сегодня законы пишутся
ими, и государство не может им сказать, что делать. В этом смысле у них менталитет, прямо
противоположный российскому.

Обобщим сказанное:

Ментальность -- это система абстракций и стоящих за ними метафорическими образами,
которые регулируют нашу жизнь, правила поведения, и через которую мы соизмеряем себя
с нашими внутренними структурами, формирующие наше <<я>>.

В отличие от законченных и продуманных доктрин и идеологических конструкций, ментальность
рассеяна в культуре и в обыденном сознании и как таковая может изменяться.
Более того, очень часто она не осознается самими людьми и проявляется в их поведении
и речевых высказываниях как бы независимо от их желаний и воли.

Наконец, ментальность вбирает в себя не столько индивидуальные личностные установки,
сколько <<внеличную сторону общественного сознания, будучи имплицированы в языке и
других знаковых системах, в обычаях, традициях и верованиях>>\autocite{gurevich_history}. Переходим к ментальности в контексте массового общества.

В 1841 году американский философ Р.У.~Эмерсон опубликовал знаменитый очерк <<О доверии к себе>> (Self-Reliance),
где он, в частности, утверждает следующее: <<Повсюду общество состоит в заговоре против независимой позиции каждого из его членов. Общество подобно акционерной компании, в которой держатели акций, дабы  обеспечить свой куш каждому пайщику, согласились для этого принести в жертву  индивидуальную свободу и культуру. Конформизм  - добродетель, на которую здесь самый большой спрос.  Доверие к себе отторгается. Общество не переносит реальности и деятельности творческих личностей; оно предпочитает им ничего не значащие слова и условности.>>\autocite{emerson1972self}. В сокращенном парафразе это звучит примерно
так: <<Любое общество всегда находится в заговоре против человека. Конформизм считается добродетелью;
уверенность в себе - грехом. Общество любит не человека и жизнь, а имена и обычаи>>. Иначе говоря,
любое общество любит знаки и знаковые системы, и, если это общество массовое, то оно по определению
должно любить знаки и знаковые системы массового общества, формирующие его ментальность.

Известные теории массового общества (см.~\ref{1.1.1}) традиционно называют <<массовой>>
современную социальную организацию общества, при которой человек превращается в безликий
элемент бездушной социальной машины с ущербной нулевой
ментальностью\autocites{macdonald2011}[][3-30]{leavis1930}[][79-202]{eliot2010}. Он
ощущает себя жертвой обезличенного социального процесса и занят главным образом тем, чтобы адекватно
синхронизировать свои потребности с потребностями этой машины. Основой массового общества при таком
подходе принято считать массовое производство стандартизированных вещей и соответствующее
манипулирование взглядами, вкусами, психологией <<одномерных>>, в терминах Г.~Маркузе, людей.

Две основные теории массового общества
с политическим уклоном оформились в середине ХХ века: леворадикальная (Р.~Миллс)
и либерально-критическая (Э.~Фромм, Д.~Рисмен, К.~Маннгейм). Оба лагеря направили острие своей критики
на отчуждающую бюрократизацию власти, усиления контроля над личностью и конформизацию людей.
Наконец, в 1960-1970-е~гг. американские социологи Э.~Шилз и Д.~Белл в поисках компромисса объявили
наличные теории массового общества <<неоправданно критическими>> и попытались направить их в русло
официальной идеологии. Шилз, например, считал, что благодаря массовым коммуникациям <<народные массы>>
усваивают ментальность, ценности и нормы, предлагаемые элитой, и тем самым общество движется в сторону
преодоления социальных антагонизмов. Не только адаптированные массы интегрируются в систему
социальных институтов <<массового общества>>, но и само общество в ХХ веке выступает как масса.
Эта концепция также была подвергнута резкой критике и на этом эволюция теоретического осмысления массового
общества затормозилась. Определение массового общества (равно как и культуры или сознания) по своему носителю, массе, неизбежно заходит в тупик, считает британский культуролог Р.  Уильямс. «Масса» как функциональный термин научного описания и «масса» как исходный референт не находят друг друга. Лингвистически «масса» – это эмоционально заряженное определение, пренебрежительный эпитет для «Другого», непохожего на меня человека или группы. Эпитет для тех людей, кого я не знаю лично и, следовательно, не понимаю, не принимаю и возможно опасаюсь\autocite{williams1985}\autocite{williams1989}\autocite{williams2006}.

Более того, фиксирование массовых процессов в обществе преимущественно в количественных показателях неизбежно приводит к представлению о членах данного общества как гомогенизированной массе пассивных жертв или зрителей\autocite{levada2001}. С другой стороны, следует признать, что оперирование аналитическими категориями массовости   имеет значение при рассмотрении вопроса о возможности управления массовыми процессами через использование специфических средств массового воздействия. К последним относятся средства массовой пропаганды и рекламы, масс-медиа в целом. Ю. А. Левада отмечает: «Нельзя повлиять на политическое или потребительское поведение отдельного человека (как нельзя и предсказать его), но можно с достаточно большой эффективностью воздействовать на поведение многих тысяч и миллионов людей.»\autocite{levada2001}.

К средствам тотального массового воздействия относится, как мы покажем, и логотип. Следует подчеркнуть, однако, что в этом аспекте собственно культурологические рассуждения вплотную подходят к социально-психологическим исследованиям и в известной степени опираются на них.

С точки зрения социальной психологии,  «главная особенность <<массы>> -- временность ее существования. ($\ldots$)
Наконец, масса возникает и функционирует на основе собственных внутренних, психологических,
а не внешних (социологических, философских и т. п.) закономерностей (\ldots)>>\autocite{olshansky}.
В обществе роль масс становится заметной во время разрыва групповых связей и размывания
межгрупповых границ, когда общество переживает период социальных потрясений и деструктурируется.
В этом смысле, <<массы>> -- категория кризисного общества в нестабильное, <<смутное>> время, в периоды
социальных революций, крупномасштабных социальных реформ,
политических переворотов, войн и проч. Следовательно, говоря о массах, мы ведем речь о функциональном явлении,
базирующемся на временном психологическом единстве людей, образующих массу. Такое единство внутренних психологических
характеристик можно назвать по-другому -- массовое сознание. И поскольку массу очень трудно определить морфологически,
попробуем взглянуть на нее со стороны массового сознания.

Под данным углом зрения, массовое сознание -- это специфический вид общественного сознания, присущий крупным
неструктурированным множествам людей. <<Массовое сознание определяется как совпадение в какой-то момент
(совмещение или пересечение) основных и наиболее значимых компонентов сознания большого числа весьма
разнообразных <<классических>> групп общества (больших и малых), однако оно несводимо к ним. Это новое качество,
возникающее из совпадения отдельных фрагментов психологии деструктурированных по каким-то причинам <<классических>>
групп.>>\autocite{olshansky}.  С точки зрения содержания, оно реализуется в массе индивидуальных сознаний,
хотя и не совпадает с каждым из них в отдельности. В этом смысле массовое сознание представляется надындивидуальным
и надгрупповым по содержанию, но индивидуальным по форме функционирования.

Далее, в отличие от группового сознания, для зарождения и функционирования массового сознания совсем не
обязательна совместная деятельность членов массовой общности. Запечатленные в нем знания, нормы, ценности и
образцы поведения вырабатываются спонтанно в процессе общения людей между собой, равно как и коллективного
восприятия социальной, политической и проч. информации, как, например, на политического митинге или на рок-концерте.
Из этого следует, что в основе массового сознания обычно лежит яркое эмоциональное переживание некоего
события или социальной проблемы, вызывающей всеобщую озабоченность. При этом интенсивность переживания
выступает как системообразующий фактор массового сознания. 3.~Фрейд в своем обличительном очерке о массовой
психологии утверждал: <<Масса импульсивна, изменчива и возбудима. Ею почти исключительно руководит
бессознательное.>>\autocite{freid_mass}.  Отсюда естественно вытекают основные характеристики массового сознания,
а именно: аморфность, заразительность, изменчивость, мозаичность, неоднородность, противоречивость и
размытость\autocite{ashin}\autocite{flier}\autocite{prokudin}\autocite{heveshi}\autocite{hevishi2001tolpa}\autocite{streltsov1970}\autocite{dodonov1999}.

Поскольку детальный анализ массового общества все-таки не входит в задачи данного исследования,
мы просто ограничимся здесь кратким резюме:

\textbf{Первое.} В отличие от обычных устойчивых социальных групп, массы являются временными,
  ситуативными общностями. Хотя они разнородны по составу, они объединяются по признаку сопричастной
  значимости психических переживаний входящих в них людей. К главным различительным особенностям масс
  относятся:
  \begin{enumerate*}[label=\asbuk*)]
  \item их размеры,
  \item устойчивость существования во времени,
  \item степень компактности в социальном пространстве,
  \item уровень сплоченности или рассеянности,
  \item преобладание факторов организованности или стихийности в возникновении.
  \end{enumerate*}

\textbf{Второе.} Принципиальное отличие массы от традиционно выделяемых социальных групп,
  классов и слоев общества состоит в наличии особого, не организованного, самопорождающегося и
  плохо структурированного массового сознания -- обыденной разновидности общественного сознания,
  объединяющей представителей разных классических групп общими переживаниями. Такие переживания
  актуализируются при особых обстоятельствах. При этом само сознание может расширяться, вовлекая
  все новых людей из разных социальных групп, но может и сужаться, тем самым уменьшая размеры массы.
  Характерно, что общественное мнение и массовые настроения обычно принято считать ведущими макроформами
  массового сознания.

Таким образом, изменчивость границ массы и массового общества существенно затрудняет создание сколь
либо исчерпывающей типологии массового сознания и сопутствующей ему ментальности. В таких случаях,
как это принято в науке, единственным выходом становится моделирование изучаемого явления, здесь --
построение апроксмированных моделей массового сознания. При этом важно помнить, что любая модель
необходимо условна, абстрактна и, следовательно, семиотична по определению, включенной в семиотическую реальность.
И как таковая, такая знаковая система прежде всего свидетельствует о ментальности исследователя (см.~\ref{1.2}) и
только затем, опосредованно говорит о моделируемом явлении.
Чем сильнее упрощение и чем больше описываемая модель не похожа на исходный референт объективной реальности,
тем обширнее культурологическая составляющая при ее расшифровке. По мере повышения абстрактности используемых
знаков повышается и общий уровень абстракции всей создаваемой из них знаковой системы. Наконец, по мере
повышения степени абстрактности знака и знаковой системы усиливается и ригидность формулируемых правил,
регулирующих их использование. В результате, любое моделирование реальности шифруется в знаках метаязыка и
оформляется в виде заданных метаязыковых параметров и установок. Одной из таких важных, но потенциально опасных
установок является прямая проекция предлагаемой модели в целях ее верификации на реальную жизнь общества.
Опасность заключается в аналоговом отожествлении, метафоризации и мифологизации, имманентных абстрактному
построению. Для наглядности рассмотрим это на примере модели массового общества, предложенной М.И.~Найдорфом\autocite{ocherki}.

За исходную посылку берется здесь тезис о том, что современное общество является массовым по определению.
Жизнь в современном массовом обществе носит коллективный характер и <<основывается на разнообразных технологиях
массообразования, взятых вместе со средствами символизации массовых общностей.>>\autocite{ocherki}.
К таковым прежде всего относятся, по мнению исследователя, политическая и торговая реклама. Приемлемость
и объединяющую разумность массового способа жизни находят свое отражение в системе новых массовых
ценностей, таких как политкорректность и толерантность, идеалы равенства и демократии, философия комфорта
и потребления, идеология развлечений, и проч.

Другой исходной посылкой служит утверждение о том, что массовое общество <<нуждается в иначе устроенном
человеке и производит его>>. Как и в каждой культуре, известной нам из истории, в
массовой культуре необходимо присутствует представление о <<своём» человеке>>\autocite{ocherki}. Такое представление,
объясняет Найдорф, задается его <<местом>> в этом мире, ограниченного культурно допустимым и возможным поведением\autocite{naydof}.
Чье это представление, и каким образом массовое общество вообще может <<производить>>
человека автор пока не поясняет. Вместо этого он развивает свой тезис дальше и аксиоматически
строит обратное утверждение: <<Если верно, что тип массовой культуры полагает соответствующий ему тип человека,
то верно и обратное: современный человек не может жить иначе как в массовом обществе>>\autocite{ocherki}.
Массовый человек воспроизводит массовый способ жизни в своей повседневной практике, <<реализуя возможности,
преимущества и ограничения своего <<места>> всей совокупностью средств, очерченных представлениями массовой
культуры о массовом человеке>>\autocite{ocherki}. Из чего мы узнаем, что субъектом представлений о массовом человеке --
как это не парадоксально выглядит -- является сама массовая культура. Соответственно, полагает Найдорф,
человек массовой культуры или <<массовый человек>> -- это такой человек, который добровольно считает
массовую культуру своей. Наконец, главное: массовый человек -- <<это человек, действующий так, как его
мотивирует массовое сознание>>\autocite{ocherki}.

Разумеется, в такой степени методологически редуцированный человек, производимый массовым обществом и
мотивируемый массовым сознанием, живет полноценной жизнью лишь в рамках абстрактной знаковой модели,
которая, как мы говорили, дает лишь опосредованное представление о предмете. Вести речь о ментальности
здесь, по-видимому, не имеет смысла, так как она представлена лишь одним субъектом: исследователем.
Нас, в свою очередь, интересует совсем не абстрактная модель человека, сознания и общества, а реальный человек,
реальное сознание и реальное общество. По этой причине, термин \emph{массовое общество} для нас, главным образом,
тавтология, прием смещения акцента и привлечения внимания к стихийным, неструктурированным массовым процессам,
объединениям, со-обществам и общностям внутри общества, базирующихся на временном психологическом единстве.
Напротив, мы будем исходить из того классического представления, что человек -- это биосоциальное существо,
развивающиеся по биологическим законам, но отличающиеся от других высших животных членораздельной речью и
сознанием, способностью производить орудия труда и созидательной активностью. Адаптационный <<стадный инстинкт>>
присущ ему в точно такой же мере, как и многим видам животных, живущих в группах, стадах, стаях и проч.

В узком смысле, стадный инстинкт определяет взаимоотношения особей одного вида в группах, расселение молодых животных,
отделившихся от взрослых и т.д. В социальном смысле, в человеческой истории, по всей видимости, масс никогда бы не
было, если бы не возникало особой потребности объединятся в такие массы. Таким образом из внутренней потребности
человека рождается особый мотив -- объединяться с подобными себе <<ради самосохранения, достижения: каких-то выгод
или некоторого внутреннего состояния.>>\autocite{olshansky}.  3.~Фрейд формулировал это так: <<Отдельный человек чувствует
себя незавершенным, если он один>>\autocite{freid_mass} И если самосохранение и достижение выгод все-таки не
нуждаются в специальном разъяснении, то достижение некоторых внутренних состояний требует расшифровки.
Соответственно, ниже мы будем вести разговор о тех внутренних психических внезнаковых реалиях, которые,
собственно и составляют основное содержание массовой ментальности как таковой.

%% TODO: где там же то?
Речь идет о неосознаваемой потребности в эмоционально\hyp{}аффективных состояниях, как положительных,
так и отрицательных, требующих от человека пребывания в массе или своими действиями способствовать ее возникновению.
Внешне, все происходит как бы стихийно, само собой, и масса возникает независимо от людей. Психологически, однако, все выглядит иначе. В основе формирования массы лежат индивидуальная потребность в
самоидентификации с большим количеством людей для регуляции своих эмоциональных состояний. При этом важно, как
отмечают психологи, что <<эта потребность обычно актуализируется в тех случаях, когда речь идет о сильных
эмоциональных состояниях, с которыми сам индивид справиться не может. Тогда ему и необходима особая идентификация --
не психологическое отождествление себя с другими людьми, а физическое соединение с ними.>>\autocite{olshansky}.  Добавим лишь,
что эти потребность в самоидентификации с массой может иметь разную полярность -- как разрядки определенных
(обычно резко отрицательных) эмоциональных состояний, так и усиления состояний связанных с эмоциями положительности,
наблюдаемых, например, в ритуалах и традициях народных гуляний и празднований, семейных торжествах, спортивных
состязаниях, предпраздничных распродажах, сообществах в социальных сетях и т.д.

Таким образом, индивид предшествует массе. Он присоединяется к ней инстинктивно и добровольно.
Потребность в массе мотивируется, в свою очередь, потребностью стабилизировать свои эмоциональные состояния,
усиливая положительные и уменьшая отрицательные эмоции. Другой вопрос, что удовлетворение этой потребности
ведет к временной нейтрализации рациональных компонентов психики, таких как снижение критичности восприятия
и чувства собственного <<я>>. Происходит некоторая временная деиндивидуализация человека. С психологической
точки зрения, потребность раствориться в массе, хотя и регрессивная по характеру, наравне с потребности
быть личностью: развивать индивидуальное сознание и самоопределение -- вполне естественна для человеческой психики.
Э.~Фромм, например, метафорически описывает эту тенденцию как <<бегство от свободы>>. Уставая от необходимости жить
все более рационально, от индивидуальной свободы и, пожалуй, самое главное, от связанной с ней
индивидуальной ответственности, человек добровольно бежит в массу\autocite{fromm}. Там, при отсутствии какого-либо
сопротивления с его стороны, незаметно для него меняется его психика. С течением времени, по мере того
как приведшая его в массу потребность иссякает. Человек выходит из массы -- <<с новым опытом, и с теми
или иными, преходящими или уже хроническими, изменениями психики>>\autocite{olshansky}.

Мы говорим в данном случае о так называемых <<естественных>> массах, возникающих стихийно и самопроизвольно.
От них принять отличать <<искусственные>> массы, такие как, например, армия или церковь\autocite{freid_mass}.

Сделаем несколько предварительных обобщений, прежде чем перейти к следующему разделу.
\begin{enumerate}
    \item В пределах культуры массовое общество – это функциональное явление внутри общества,
    базирующееся на временном психологическом единстве людей, образующих массу. Массовое сознание,
    присущее массовым общностям, характеризуется неоднородностью, аморфностью, изменчивостью,
    противоречивостью и размытостью.
    \item Ментальность в массовой общности имеет два культурно-семиотических плана: план выражения и план содержания.
    План выражения получает знаковое оформление преимущественно в категориях абстрактных понятий
    национального языка и иных невербальных знаковых системах культуры по принципу аналогово отождествления.
    План содержания включает в себя дознаковые бессознательные реалии психической жизни человека, внутренние
    потребности и мотивы, побуждающие его соединяться с себе подобными.
    \item В плоскости культуры повседневности ментальность в массовой общности приобретает следующие черты:
    \begin{enumerate*}[label=\asbuk*)]
        \item повышенная эмоциональность в восприятии всего, что человек видит и слышит;
        \item повышенная внушаемость и уменьшение степени критичности к самому себе и способности критического
        осмысления воспринимаемой информации;
        \item подавление чувства ответственности за собственное поведение.
    \end{enumerate*}
\end{enumerate}

Иначе говоря, в действиях человека в массе преобладает аффективно-бессознательное поведение, а сознательная личность нивелируется. Человек в массе не столько производит знаки, сколько их потребляет.  Или точнее -- некритично руководствуется ими. Рассмотрим этот вопрос подробнее.
