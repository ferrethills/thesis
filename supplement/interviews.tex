\section{Интервью автора с современными мировыми дизайнерами логотипов}

\subsection{Алан Ороноз}

\begin{description}
\item[Имя] Алан Ороноз
\item[Страна] Мексика
\item[Опыт работы] Графический дизайн (лого дизайн, брэндинг, иллюстрация, упаковка). 8 лет работы с клиентами со всего мира.
\end{description}

\textbf{Давайте начнем с общего вопроса: Что такое хороший логотип, по вашему мнению?}

Хороший логотип должен быть уникальным и запоминающимся. Я совершенно не считаю, что лого обязательно должно быть простым. Хорош тот логотип, которому удается передать суть компании. Рынок очень разнообразен. Аудитория, к которой обращен логотип неоднородна и восприимчива к разным визуальным элементам. Поэтому важно правильно оценить, что лучше всего подходит для определенного вида бизнеса.

\textbf{Что вы думаете по поводу брендинга стран? Будет ли в ближайшем будущем свой логотип у каждого места на земном шаре или это просто временный тренд?}

Я думаю, что глобализация совершенно изменила наше восприятие брэнда, и нация сегодня может рассматриваться как брэнд. Этот тренд переживал настоящий бум несколько лет назад, но я не думаю, что в будущем он станет глобальным. Однако некоторые страны возможно будут использовать его для стимулирования подъема экономики.


\textbf{Можно ли говорить о каком-то определенном национальном стиле логотипов?}

Национальные логотипы имеют тенденцию быть <<стилизовано современными>>. Такой стиль конечно же следует определенному тренду. Есть много схожего в том, что касается цвета, образов и даже шрифта.

\textbf{Согласны ли вы с тем, что логотип – это эмблема счастья в нашем потребительском обществе?}

Не сказал бы, что они эмблемы счастья, скорее – комфорта. Я думаю, что мир сегодня движется быстрее, мы находимся в состоянии стресса, и поэтому с каждым днем все более держимся за бренды, которые нам близки.

\textbf{Каково будущее лого дизайна? Как вы думаете, возможен ли сильный бренд без логотипа?}

Для меня бренд не работает без лого. Он, как верхушка айсберга. Совершенно очевидно, что если вы не увидите верхушку, то так никогда и не узнаете, что там, внизу, что-то есть. Я думаю, что лого дизайн будет существовать всегда. Он будет просто адаптироваться к своему времени. В дизайне есть тенденция к цикличности, и возможно через несколько лет минимализм устареет, но кто знает, что будет собой представлять следующий тренд?


\clearpage
\subsection{Георгий Бокуа}

\begin{description}
\item[Имя] Георгий Бокуа
\item[Страна] Грузия
\item[Опыт работы] --
\end{description}

\textbf{Давайте начнем с общего вопроса: Что такое хороший логотип, по вашему мнению?}

Основной ингредиент в хорошем логотипе – это чувство законченности и универсальности. Добавим несколько ложек умных графических решений, щепотку новаторства, талантливую смесь геометрических элементов и получится довольно вкусно!

\textbf{Что вы думаете по поводу брендинга стран? Будет ли в ближайшем будущем свой логотип у каждого места на земном шаре или это просто временный тренд?}

Мне кажется, что брендинг городов более привлекателен, чем брендинг стран. Для стран их национальные флаги и эмблемы важнее логотипов, придуманных для туристских целей. Если люди в мире станут дружелюбнее и гостепримнее, вот тогда возможно и брендинг наций будет более востребован.


\textbf{Можно ли говорить о каком-то определенном национальном стиле логотипов}

Универсальность – одно из лучших качеств логотипов сегодня, так как бренды становятся все более глобальными. Мне нравится грубоватый стиль голландских дизайнеров, которые выбирают нечто старое, грубоватое не только в графике, но и в моде и архитектуре.


\textbf{Согласны ли вы с тем, что логотип – это эмблема счастья в нашем потребительском обществе?}

Да, я с этим согласен. Думаю, что при виде логотипов любимых брендов, явно идет секреция гормонов счастья.


\textbf{Каково будущее лого дизайна? Как вы думаете, возможен ли сильный бренд без логотипа?}

Я думаю, что действительно талантливые логотипы в будущем будут выглядеть примерно также, как и великие логотипы сейчас. В большинстве таких логотипов используется одинаковая символика: крест, звезда, луна, стрела и т.д. Именно это делает их вневременными, вечными. За этими логотипами стоят великолепные продукты. Думаю, что даже плохой логотип может стать великим, если качество того, что он представляет действительно высоко.

Не уверен по поводу брендов без логотипов. Уже были попытки сделать это.
В целом же, если у вашего бренда есть имя, то, как бы вы его не писали, он получает айдентику. Однако, если имени нет, наверное это возможно.


\clearpage
\subsection{Петер Васвари}

\begin{description}
\item[Имя] Петер Васвари
\item[Страна] Венгрия
\item[Опыт работы] 10 лет
\end{description}

\textbf{Давайте начнем с общего вопроса: Что такое хороший логотип, по вашему мнению?}

Хороший логотип должен четко читаться в уменьшенном размере.


\textbf{Что вы думаете по поводу брендинга стран? Будет ли в ближайшем будущем свой логотип у каждого места на земном шаре или это просто временный тренд?}

Национальный брендинг сейчас довольно однообразен, но есть исключения, и это хорошо!


\textbf{Можно ли говорить о каком-то определенном национальном стиле логотипов?}

Национальные стили, конечно, существуют, например, логотипы Перу, Новой Зеландии и других стран.

\textbf{Согласны ли вы с тем, что логотип – это эмблема счастья в нашем потребительском обществе?}

Да, они действительно логотипы счастья, в том случае, если они грамотно визуально выполнены.


\textbf{Каково будущее лого дизайна? Как вы думаете, возможен ли сильный бренд без логотипа?}

Я надеюсь, что бренды сохранят четкую визуальную составляющую. К сожалению, сейчас наблюдается тенденция к упрощению лого дизайна. Например, просто круг или буква <<О>> уже могут представлять собой бренд.



\clearpage
\subsection{Шон О'Грейди}

\begin{description}
\item[Имя] Шон О'Грейди
\item[Страна] Ирландия
\item[Опыт работы] Издательское дело, спортивный брендинг, вывески, реклама, графический дизайн. Последние семь лет: лого дизайн и брендинг.
\end{description}

\textbf{Давайте начнем с общего вопроса: Что такое хороший логотип, по вашему мнению?}

Хороший логотип точно доносит до аудитории информацию, которую он продвигает.


\textbf{Что вы думаете по поводу брендинга стран? Будет ли в ближайшем будущем свой логотип у каждого места на земном шаре или это просто временный тренд?}

Национальный брендинг очень важен сегодня, особенно, если страна хочет развивать туризм. Каждая нация на земле имеет национальный флаг, что само по себе является наиональным брендом.


\textbf{Можно ли говорить о каком-то определенном национальном стиле логотипов?}

В качестве ирландского бренда обычно используется трилистник. Этой же роли зачастую служит арфа.


\textbf{Согласны ли вы с тем, что логотип – это эмблема счастья в нашем потребительском обществе?}

Я полагаю, что логотип должен представлять позитивный взгляд на компанию, поэтому, возможно, он и должен быть эмблемой сачстья.


\textbf{Каково будущее лого дизайна? Как вы думаете, возможен ли сильный бренд без логотипа?}

Лого, символы и другие эмблемы существуют уже столетия и будут существовать еще долго. Брендинг в последне время приобретает все более важный статус, что усиливает значимость  логотипа.


\clearpage
\subsection{Саймон Фроуз}

\begin{description}
\item[Имя] Саймон Фроуз
\item[Страна] Южная Африка
\item[Профессия] Графический дизайнер, творческий директор
\item[Опыт работы] 16 лет
\end{description}

\textbf{Давайте начнем с общего вопроса: Что такое хороший логотип, по вашему мнению?}

В случае успешного логотипа аудитория ощущает эмоциональную связь с брендом. Это может проявляться по-разному: доверие, желание, ощущение счастья. Если лого воплощает желаемые качества бренда, оно является успешным, на мой взгляд.


\textbf{Что вы думаете по поводу брендинга стран? Будет ли в ближайшем будущем свой логотип у каждого места на земном шаре или это просто временный тренд?}

Брендинг наций – это хорошо. Он помогает создать упрощенную визуальную карту, которую люди могут запомнить. Он может способствовать развитию туризма, экономическимх инвестиций и т.д. Брендинг наций словно отправляет послание от данной страны всему человечеству. Он может обладать огромной силой в объединении людей, что и произошло в Африке.

<<<<<<< HEAD
Брендированию в той или иной форме (визуальной или какой-либо иной) подвергается все, поэтому весьма логично, что нации стремятся к созданию индивидуальных или коллективных брендов. Я полагаю, что в будущем все нации будут иметь лого, ибо у всех нас есть желание принадлежать к чему-то большему, чем мы сами. 
=======
Брендированию в той или иной форме (визуальной или какой-либо иной) подвергается все, поэтому весьма
логично, что нации стремятся к созданию индивидуальных
(URL: \url{http://www.southafrica.net/za/en/landing/visitor-home}) или коллективных
(URL: \url{http://europa.eu/geninfo/atoz/en/index_1_en.htm}) брендов. Я полагаю, что в будущем все нации
будут иметь лого, ибо у всех нас есть желание принадлежать к чему-то большему, чем мы сами.

>>>>>>> 8694bbd93e8323cd040e49a7a29569571a4685fd


\textbf{Можно ли говорить о каком-то определенном национальном стиле логотипов?}

Да, в разных странах всегда существуют уникальные культурные влияния на дизайн. Однако из-за современной глобализации и благодаря средствам коммуникации наблюдается все большее слияние стилей, и тренды растпространены более широко, чем прежде. Становится все труднее определить страну создания определенного логотипа. И тем не менее, многие развивающиеся государства пытаются создать четкую национальную идентичность, вызванную стремлением отличаться от других. А такие страны, как США, зачастую демонстрируют в лого дизайне  сильное чувство патриотизма.


\textbf{Согласны ли вы с тем, что логотип – это эмблема счастья в нашем потребительском обществе?}

Я с этим не согласен. Логотипы изображают самые разные эмоции. Счастье – одна из них, но не единственная. Есть примеры успешных логотипов, символизирующих негатиные эмоции, такие как страх и ненависть.

\textbf{Каково будущее лого дизайна?}

Иллюстрированные, фотографические, анимационные, 3-D, голлографические логотипы. Мне кажется, что логотипы станут более сложными. Становится все труднее и труднее создать оригинальный простой логотип. С появлением цифровых технологий правила игры изменились. Традиционное требование к логотипу четко чиаться в одном цвете и маленьком размере со временем исчезнет.

\textbf{Как вы думаете, возможен ли сильный бренд без логотипа?}

Бренд какое-то время может существовать без логотипа, но в нашей природе распределять вещи по категориям и давать им названия, так что со временем кто-то наделит бренд визуальной репрезентацией. Но возможно, если печатные средсвта информации когда-то окончательно устареют, бренд сможет быть представлен, например, лишь звуком. Или вообще одной идеей!
