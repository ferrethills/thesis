\section{Интервью с современными мировыми дизайнерами логотипов}

\subsection{Alan Oronoz}

\begin{description}
\item[Name] Alan Oronoz
\item[Country] Mexico
\item[Work experience] Independent creative focused on logo design and
  illustration, with experience of more than eight years in branding, packaging,
  mascoting and illustration with clients of all around the globe.
\end{description}

\textbf{Let's start with a common question, what makes a good logo in your
  opinion?}

Definitely should be unique and memorable, but far from believing that a good
logo has to be extremely simple, I believe that a good logo is the one that
achieves convey the essence of the company. There are many kinds of markets and
each audience is  receptive to certain visual elements, therefore it is
important to evaluate which is the best option for each business.

\textbf{What do you think about nation branding? Will every place on Earth have
  a logo in the near future or is it just a temporary trend?}

I think the globalization has changed completely our persepcion of what is a
brand and a nation today can be seen as such. This movement had a boom a few
years ago but I don't think it will become a future global trend, but in a
resource that some countries will use to boost its economy.


\textbf{Can we speak about any particular national style of logos?}

Yes, this logos tends to be "stylized-modern" and definitely this style follow a
trend, speaking of colors,icons even the type, there are a lot of similarities.


\textbf{Logos are emblems of happiness in our consumerist culture. Would you
  agree?}

I wouldn't consider it happiness, but comfort, I think the world moves faster
today, keeps us stressed in some way and for this reason we cling each day more
to brands that we have affinity.

\textbf{What is the future of logo design? Do you think it's possible to have a
  strong brand being logoless?}

For me a brand can't work without a logo, this is the "tip of icberg" and for
obvious reasons, if you can't see the tip, you will never know that there is
something down. I think the logo design will always be present and will simply
adapting to their time, would be difficult to talk about trends or styles, as
the design tends to be cyclical so maybe in a few years the minimalism could be
obsolete, how knows what will be the next trend?


\clearpage
\subsection{George Bokhua}

\begin{description}
\item[Name] George Bokhua
\item[Country] Georgia
\item[Work experience] --
\end{description}

\textbf{Let's start with a common question, what makes a good logo in your
  opinion?}

In good logo the main ingredient is sense of completeness and universlity, few
spoons of clever graphic solutions, pinch of inovation, and good mix of
geometric elements make logo quite tasty.


\textbf{What do you think about nation branding? Will every place on Earth have a logo
  in the near future or is it just a temporary trend?}

I find city branding more attractive that nation branding. For nations flags and
emblems are usually larger part of branding then some logos that are made for
turisic purposes. If people of world will become more friendy and hospitable,
perhaps in future the nation branding will be more prominent.


\textbf{Can we speak about any particular national style of logos?}

Universality is one of the best values in logos nowadays, since brands are
trying to be increasingly international. I like the feel of Dutch designers,
that grungy feel. They tend to like old, dirty, rusty look not only in graphics
but also in architecture and fashion.


\textbf{Logos are emblems of happiness in our consumerist culture. Would you
  agree?}

Yes I would. I am sure theres quite a lot happiness hormone secretion when
someone sees the logo of it’s favorite brand.


\textbf{What is the future of logo design? Do you think it's possible to have a
  strong brand being logoless?}

I think great logos in future will have a same look as great logos do now. Most
timless marks have a similar tendencies : cross, star, moon, heart, arrow,
etc. they all have same few characteristics in common that make them
timeless. They stand for something great. So even a badly designed logo can be
great if it stands for great brand.

I am not sure about logoless. There already have been few brands who tried
it. Generally if you brand has a name any way you write it will become it's
identity. If brand has no name then it's possible I guess.


\clearpage
\subsection{Peter Vasvari}

\begin{description}
\item[Name] Peter Vasvari
\item[Country] Hungary
\item[Work experience] more than 10 years
\end{description}

\textbf{Let's start with a common question, what makes a good logo in your
  opinion? }

In the small size must be clearly identifiable to be


\textbf{What do you think about nation branding? Will every place on Earth have
  a logo in the near future or is it just a temporary trend?}

Different nation branding, yes, currently unified. But there are exceptions. And
this is good!


\textbf{Can we speak about any particular national style of logos?}

Yes, of course they exist. For example, Peru, New Zealand logo etc.


\textbf{Logos are emblems of happiness in our consumerist culture. Would you
  agree?}

Yes, If have made a visually intelligent.


\textbf{What is the future of logo design? Do you think it's possible to have a
  strong brand being logoless?}

I hope remain strong distinctive visual appearance! Yes, now unfortunately
simplified the logo design! For example, every letter O or circle. And it
represents a brand.


\clearpage
\subsection{Sean O'Grady}

\begin{description}
\item[Name] Sean O'Grady
\item[Country] Ireland
\item[Work experience] Previously worked in print, signage, sports branding,
  advertising agency and general graphic design industry. Freelance for the past
  7 years focussing mostly on logos \& branding.
\end{description}

\textbf{Let's start with a common question, what makes a good logo in your
  opinion?}

A good logo is one which is appropriately designed to accurately communicate to
the general audience what it wishes to promote.


\textbf{What do you think about nation branding? Will every place on Earth have
  a logo in the near future or is it just a temporary trend?}

Nation branding is very essential these days especially if governments wish to
promote tourism. Every nation on earth has a national flag which is a nation
brand in itself.


\textbf{Can we speak about any particular national style of logos?}

The shamrock is generally used in Irish branding although the harp is also used.


\textbf{Logos are emblems of happiness in our consumerist culture. Would you
  agree?}

I suppose logos are meant to represent a positive view of their company. So I
reckon they should be emblems of happiness.


\textbf{What is the future of logo design? Do you think it's possible to have a
  strong brand being logoless?}

Logos, symbols and other emblems have been around for centuries and should be
for a long time to come. Branding has become much more important though and the
overall brand experience is enhanced.


\clearpage
\subsection{Simon Frouws}

\begin{description}
\item[Name] Simon Frouws
\item[Country] South Africa
\item[Work experience] 16 years
\end{description}

\textbf{Let's start with a common question, what makes a good logo in your
  opinion?}

When the audience has an emotional connection with a brand. This can come in
many different forms. Trust, aspiration, happiness, etc. If the logo embodies
the desired qualities of the brand, it is successful in my opinion.


\textbf{What do you think about nation branding?}

Nation branding is good. It helps create a simplified visual "map" for people to
remember. It can promote tourism, economic investment, etc. Unified nation
branding helps create a consistent message delivered to the rest of the
world. Nation branding can be very powerful in uniting its people, as happened
in South Africa.


\textbf{Will every place on Earth have a logo in the near future or is it just a
  temporary trend?}

Everything is branded in some form or other, whether visually or by other
means. So it's logical for nations to brand themselves, whether individually
(\url{http://www.southafrica.net/za/en/landing/visitor-home}) or collectively
(\url{http://europa.eu/index_en.htm}). I think that all nations will have a logo in
the future. As people we have a desire to belong to something larger than
ourselves.


\textbf{Can we speak about any particular national style of logos?}

Yes, as there are always unique cultural influences on design in different
nations. But through modern globalization and communications, especially social
media, there is far greater fusion of styles, and trends are more widespread
than before. It is increasingly difficult to discern the origin of a particular
logo. In saying that though, there are many growing nations that try to create a
distinct national identity in an attempt to stand out in the world. And
established nations, like the USA, have a strong sense of patriotism in many of
their logo designs.


\textbf{Logos are emblems of happiness in our consumerist culture. Would you
  agree?}

I do not agree. Logos portray many different emotions. Happiness is one of them,
but not the only one. There are logos that have been equally <<successful>>
symbolizing negative emotions, such as fear and hatred.


\textbf{What is the future of logo design?}

Illustrated logos, photographic logos, animated logos, 3-D logos, holographic
logos. I predict that logos will become more complex in an attempt to
differentiate themselves. It's increasingly difficult to create a unique logo
that's simple. With digital technology, the rules have changed. The traditional
need for a logo to work in one-colour and small sizes will cease to exist.


\textbf{Do you think it's possible to have a strong brand being logoless?}

A brand could be logoless for a time, but it's human nature to categorize and
label things, so eventually someone will put a visual representation on it! But
perhaps if printed media is totally obsolete one day, a brand may be represented
by a sound only, for example. Or even by just an idea!
