\section{Эмблематическая сущность логотипа}

В заключительном параграфе мы обратимся к определению границ семантического поля и социального
статуса логотипа. Номинально логотип, мы говорили ранее, есть общераспространенный знак-переменная
обозначения или маркировки объектов потребления и их производителей. В повседневном общении и в
специальной литературе он может называться товарным знаком, торговой маркой, маркой,
этикеткой-лейблом, символом, брендом и эмблемой.  Отсутствие единого представления о логотипе, с
одной стороны, отражает  субъективное желание пользователей и потребителей наделить знак удобными
или воображаемыми качествами, в целях использования в социально-экономической практике. С другой
стороны, с точки зрения практики лого дизайна, известная произвольность  использования термина
отражает  эклектичность его дизайна и принципов композиции, объективно вытекающих из необходимости
учитывать и удовлетворять желание заказчика. Как показывает опыт, отношение заказчика к выбору
своего логотипа сегодня носит отчетливо выраженный мифотворческий характер. Логотип в данном случае
представляется ему элементом  своего публичного образа или имиджа, тем, что А.Ф. Лосев называет
разрисовкой, картинным излучением личности. \autocite[][94]{losev1991} Для лояльного потребителя
однако логотип, по всей видимости, значим не столько своими мифотворческими потенциями, сколько
тотемической привлекательностью, способностью быть знаком клановой, родовой
принадлежности. Или иначе: знаком принадлежности некоему массовому сообществу. Терминологически
мифотворческие и тотемические возможности логотипа реализуются через его эмблематическую ипостась.
Мы уже рассуждали об исторической связи логотипа с геральдической эмблемой.  Теперь мы рассмотрим
эту аналогию более подробно.

\subsubsection{Границы эмблематического в логотипе}

Эмблематичность представлена в логотипе главным образом тремя свойствами. Во-первых, по внешней
форме исполнения -- сочетание изображения и слова -- логотип может композиционно напоминать
эмблему. Даже в тех случаях, когда мы имеем дело с логотипами, состоящими из одного вербального
элемента или слова, форма исполнения данного элемента носит отчетливо выраженный изобразительный
характер. Слово-название в логотипе не пишется, а изображается. Во-вторых, по своей внутренней
мотивировке и первоочередному практическому назначению, логотип, точно также как и эмблема,
синхронически может иметь только один фиксированный смысл. В данном случае: дифференциально
обозначать и сигнализировать о присутствии и деятельности компании-производителя на рынке товаров
услуг и предполагаемом высоком качестве ее продукции. В-третьих, по своему замыслу и основному
символическому назначению логотип, подобно эмблеме, является свернутым мнемоническим блоком,
призванный транслировать во времени уникальный ценностный статус компании-производителя и ее
продукции, с одной стороны.  С другой стороны, внушать целевому массовому потребителю ощущение
эмоциональной сопричастности древней традиции, мистически глубокому, но не до конца понятному
смыслу, или, напротив, сопричастность новому технологическому прорыву, модному тренду,
революционному движению в будущее.  Иными словами, однозначность логотипа во временном континууме
предрасположена к энтропии, когда изначально жестко фиксированные внешние и внутренние связи
переформулируются, давая рождение новым смыслам.

Таким образом, логотип и эмблема нетождественны понятийно, и смысловое и формальное сходство между
ними носит лишь временный, этапный характер.  Развернем наши тезисы.

\paragraph{Эмблематическое сочетание}
Современная эмблема – это амальгам формы, образа и имени-идентификатора. Она может представлять
собой простое замкнутое построение со стилизованной надписью или изысканно-витиеватым изображением
скрытых смыслов или мотивов. Как таковая эмблема сегодня обычно используются для идентификации
спортивных команд и престижных организаций или корпораций – предположительно, как мы отмечали,  по
аналогии с использованием фамильных гербов средневековой знатью. По своему замыслу современная
эмблема позволяет  массовому потребителю, -- каковым является, в том числе и спортивный болельщик
-- выстроить более тесную коммуникацию с брендом, особенно если она представляют команду или клуб.
С другой стороны, на утилитарном уровне, эмблема выразительно смотрится и на упаковке современных
товаров, как, например, эмблема одного из старейших брендов в Великобритании Lyle’s Golden
Syrup (рис.~\ref{fig:syrop}). С момента своего появления в 1904 году упаковка и логотип компании изменились
лишь незначительно, придавая имиджу продукции компании значимый и хорошо продающийся ностальгический
статус старинной традиции.

\begin{figure}
  \centering
  \includegraphics[width=.5\linewidth]{images/goldensyrop}
  \caption{Lyle’s Golden Syrup}
  \label{fig:syrop}
\end{figure}

В современной эмблеме соположение вербального элемента и изображения-рисунка далеко не всегда
самоочевидно, что само по себе имеет принципиальное значение. Изображение должно сообщать
потребителю нечто иное, не совпадающее и не дублирующее словесное обозначение-подпись. Объяснение
рисунка в эмблеме интерпретатором призвано одновременно как истолковать смысл изображаемого, так и
констатировать семантическую обусловленность изображения и слова. Иначе говоря, оба текста,
изначально представляющие собой разные языки, необходимо приравниваются друг к другу волевым
решением автора, с одной стороны, и интерпретирующим прочтением потребителя, с другой. Эмблема
представляет собой разновидность усиленного фразеологизма,  поскольку за элементами эмблемы
сохраняется право определенной автономии, равно как и возможность образования нового смысла при их
сопоставлении, не равного смыслу каждого из этих элементов в отдельности. Такое сочетание мы
называем эмблематическим. То есть, свободное сочетание некоторого стандартного набора иконических
и вербальных элементов, когда одно и то же изображение может сопровождаться различными подписями, и
одно одна и та же подпись может прилагаться к различным изображениям. Последнее, в частности,
объясняет острую проблему защиты авторского права в лого дизайне, где плагиат и визуальная кража
оригинальных дизайнерских решений носят массовый характер.

Далее, как вербальный, так иконический элементы в эмблеме по объему вмещаемых понятий могут занимать
по отношению друг к другу
\begin{enumerate*}[label=\asbuk*)]
\item обобщающую (свертывающую) позицию и
\item индивидуализирующую (развертывающую) позицию.
\end{enumerate*}
Или проще: по принципу <<свертывание-развертывание>>.  Иконический знак
ориентирован на обозначение единичного, конкретного объекта. Конвенциональный вербальный знак, по
причине своей абстрактности, стремится обозначить любой объект или даже все объекты.

Возникающее внутреннее противоречие оказывается чрезвычайно продуктивным.

Разная природа знаков, составляющих эмблему, позволяет эффективно разрешать его, поскольку в
эмблематическом сочетании значим не сколько логико-семантический объем каждого отдельного
сополагаемого понятия, сколько приведение того и другого к общему знаменателю. Более конкретному,
чем конвенциональный словесный знак, и более абстрактному, чем иконический знак. Если каждая
отдельная составляющая эмблемы может быть многозначной, то их сочетание дает однозначный смысл. По
аналогии с естественным языком, можно было бы сказать, что элементы, образующие эмблему, выполняют
смыслоразличительную функцию, в то время как эмблема в целом -- смыслообразующую. При этом эмблема,
возникающая как результат симметрически замкнутого процесса: от конкретного к абстрактному и снова к
конкретному, может означать только самое себя, т.е. знак с жесткой системой внутренней детерминации
составляющих элементов.

\paragraph{Мнемоническая функция эмблемы}

Из сказанного следует, что эмблема, выполняя свое задание в процессе взаимного указания элементов
друг на друга, оказывается исключительно важным знаком для исполнения ею функции мнемонической
стабилизации смысла. Эмблематический знак, будучи идеальным футляром для памяти, дает возможность
одновременно хранить информацию о некоторых абстрактных свойствах, составляющих ее означаемое, с
одной стороны, и информацию о конкретном объекте предметного мира, потенциально или реально им
соответствующих, с другой. Идея замкнутости, адекватности себе, сопротивляемости переменам, как мы
показали выше, составляет суть эмблемы и позволяет рассматривать ее в качестве механизма хранения
информации. Во всяком случае, еще древние руководства по мнемотехнике советовали запоминать
словесные тексты и абстрактные конструкции, сополагая их с яркими изображениями.\autocite[][49]{grigoreva2005}

Эмблематический по форме прием запоминания использовал, например, И. В. Гете: <<Интересный для него
предмет или местность он набрасывал на бумаге с помощью немногих штрихов, детали же он восполнял
словами, которые вписывал тут же на рисунке. Эти удивительные художественные гибриды позволяли ему в
точности восстанавливать в памяти любую местность (Localitat), которая могла понадобиться ему для
стихотворения или рассказа>>. \autocite{bahtin1979}\autocite[][219]{bahtin1979full} Аналогичный
синтетический прием описывает А. Р. Лурия на примере человека, обладающего феноменальной памятью,
для которого наиболее простым способом запоминания словесного текста было последовательное
расположение слов в ряды знакомых с детства образов вдоль какой-нибудь дороги, домов, заборов,
витрин магазинов и т.д. в родном городе. \autocite[]{luria1979}
Примечательно, что механизм памяти срабатывал даже в том случае, когда не
существовало никакой очевидной семантической зависимости между словом и тем объектом, у которого оно
помещалось. Семантическая связь возникает как бы сама собой. Наконец, на основе жизненного опыта,
схожие приемы мнемотехники рекомендует и Д. Карнеги в своих популярных руководствах о том, как
добиться личного успеха в обществе. \autocite[][237--420]{karnegi1996}

Должно быть понятно, что запечатленная в эмблеме информация одновременно является смысловой
единицей. Точнее: единицей одного смысла, при этом непродолжительно действующей единицей.  Символу,
как мы говорили в конце первой главы, свойственно медленное накопление социокультурной памяти.  Его
новый смысл складывается из постепенных незаметных изменений в оттенках значения знака и неизбежно
возникающая при энтропия создает возможность полной трансформации его первоначального
смысла. Напротив, эмблематическое сочетание, по выражению И. Григорьевой, это <<информационный взрыв,
за которым следует полный штиль>> \autocite[][50]{grigoreva2005} В этом ее сильная и слабая стороны. Сильная
состоит в том, что превращает эмблему в готовый блок памяти, который при соответствующих условиях
внедрять в массовое сознание. Такой блок будет убеждать автоматически, практически не оставляя места
для сомнения в правильности предлагаемого высказывания. На этом принципе, как известно, базируется
рекламное сообщение, предполагающее максимально пассивное поглощение готовой информации.  Слабая
сторона выражается в том, что неизбежный распад эмблемы как единицы во времени, при котором
составляющие ее элементы получают возможность независимого функционирования, выводит ее из сферы
строгого эмблематического употребления в сферу стихийного и слабоуправляемого символического
смыслопорождения. Означаемое и означающее здесь зависят в первую очередь от позиции наблюдателя и,
следовательно, носят относительный характер. Логотип, рассматриваемый как часть маркетинговой
стратегии, разумеется, по целевому назначению ничем не отличается от готового мнемонического блока
эмблемы.

\paragraph{Рекламное воздействие коммерческой эмблемы}
