\subsection{Классификационный подход к логотипу. Задачи. Преимущества и недостатки}

В этом разделе мы сосредоточим свое внимание на четырех подходах к классификации логотипов:
\begin{enumerate*}[label=\arabic*)]
\item юридическом,
\item эмпирическом,
\item семиотическом.
\end{enumerate*}
Принципы отбора:
\begin{enumerate*}[label=\asbuk*)]
\item практическое назначение,
\item мотивы и темы,
\item степень суггестивного воздействия,
\item порождение значения.
\end{enumerate*}

В общем виде, классификация есть операция, заключающаяся в группировании определенного количества
фактов или явлений, обладающих общими характеристиками или признаками. В данном случае, предлагаемый
классификационный подход к систематизации практики лого дизайна преследует следующие цели:
\begin{enumerate}
\item Объединение логотипов в классы и группы согласно обозначенным принципам отбора.
\item Выявление сходств и различий в принципах, способах и приемах конструирования знака-образа коммерческого назначения.
\item Детализация композиции логотипа на примере его отдельных конкретных аспектов.
\item Определение роли исторических и социокультурных факторов в практической деятельности лого-дизайнера.
\item Помощь в адекватном прочтении логотипа, анализе и интерпретации языка лого дизайна.
\end{enumerate}

В общем и целом смысл разработки такой классификации в создание материальной базы для последующей
разработки общей методологии анализа практики лого дизайна.

Ниже, как и прежде, мы будем двигаться от общего к частному и последовательно рассмотрим
официальные, эмпирические, семиотические, функциональные и авторские таксономии.

\subsubsection{Юридические, эмпирические, семиотические виды классификаций}

%% TODO: как всё-таки писать век?
Доподлинно неизвестно кто первым стал заниматься классификацией логотипов. Можно предположить, тем
не менее, что практическая потребность в классификации возникла, вероятно, с самого момента
зарождения торговых отношений, но стала по-настоящему актуальной только с развитием массового
производства. В конце XIX в., в 1883 г. Парижская конвенция об охране промышленной собственности
выделила товарные знаки из общего понятия клейм и признала их объектом исключительного права. В
Великобритании же <<Закон о регистрации торговых марок>> был принят ранее, в 1875 г., и вступил в силу
в 1876.

На сегодняшний день Всемирная организация интеллектуальной собственности (ВОИС) пользуется
специально разработанными классификационными системами, организующими информацию о изобретениях,
товарных знаках-логотипах и промышленных образцах, в индексированные, управляемые
таксономии. \autocite{vois} Самой распространенной системой является
Международная классификация товаров и услуг (Ниццкая классификация товаров и услуг для регистрации
знаков), которая состоит из 34 классов товаров и 11 классов услуг. Помимо Ниццкой
классификации, существуют также Локарнская, Венская и Страсбургская. Все логотипы (знаки) с
юридической точки зрения делятся на:
\begin{enumerate*}[label=\arabic*)]
\item товарные знаки,
\item знаки обслуживания,
\item коллективные знаки,
\item сертификационные знаки,
\item общеизвестные знаки.
\end{enumerate*}

Приоритетным для юридических схем классификации можно считать официальную регистрацию действующих и
планируемых знаков самоидентификации в целях обеспечения гарантий охраны права собственности. Таких
классификаций, по определению, может быть немного.

Напротив, неофициальных, стихийных классификаций логотипов существует довольно много. Они могут
носить неформальный характер, как, например, классификация Тибора Кальмана <<A New Identity>> для
журнала Print Magazine (2000). Могут представлять собой просто компилятивные подборки различных
видов графических символов, как в книге Elinor Selame из книги <<Developing a Corporate Identity>>
(1975). Наконец, могут использовать сложные комбинации структурирования материала, как в работе Пера
Моллерапа <<Marks of Excellence: The History and Taxonomy of Trademarks>> (1999). Так или иначе, все
исследователи, независимо от выделяемых категорий, склонны разделять логотипы по трём видам:
\begin{enumerate*}[label=\arabic*)]
\item Типографические или буквенные (logotype) (в основе слово)
\item Графические (картинные, иконические)
\item Смешанные или комбинированные (содержащие типографический и графический элемент).
\end{enumerate*}

Обратимся к примерам. Сначала компилятивные подборки.

В своей книге <<Торговые знаки и символы мира>> Ясабуро Кувайама разделяет их на четыре класса
(алфавит; конкретные фигуры; абстрактные фигуры; символы, числа и т. д.). Классификация Кувайамы
учитывает только материальные качества, а предложенные категории не являются ни взаимно
исключающими, ни исчерпывающими. Так, например, в книге <<Trademarks \& Symbols. Volume 1:
Alphabetical Designs>> (1973) Кувайама классифицирует логотипы в соответствии с буквами английского
алфавита A-Z. \autocite[][5]{kuwayama1973alphabetical}
Вторая книга <<Trademarks \& Symbols. Volume 2: Symbolical Designs>> (1973) включает
классификацию по мотивам и темам. \autocite[][5]{kuwayama1973trademarks}
Логотипы, основанные на: фигурах людей, лицах, глазах, руках, животных, птицах, рыбах, насекомых,
цветах, деревьях, архитектуре, транспорте. Средства (1) Ёмкости; волокна, провода и т. д.; короны;
письменные принадлежности, книги; смеси, механизмы, Астрономические объекты (Солнце, Земля, Луна,
звезды); Ископаемые (Горы, камни, огонь, вода, снег); Окружности (2) Радиальные лучи; прямые линии;
кривые; Окружности (3) Вращение; Прямоугольники (Линии; поверхности); Треугольники (Точки; линии;
поверхности); Полигоны; Кривые; Логотипы, основанные на стрелках; Логотипы, основанные на цифрах;
Логотипы, основанные на знаках; Логотипы, основанные на китайско-японских идеограммах; Логотипы,
основанные на японской слоговой азбуке (1) Катакана; Логотипы, основанные на японской слоговой
азбуке (2) Хирагана.
%% TODO: this is crap, нужно переструктурировать эту классификацию.

В статье <<Типограф как аналитик>> Ганс Векерле\autocite{weckerle1968typographer} представил торговые марки в
симметричной матрице 9 x 9 (Вербальный символ: логотип; Вербальный символ: аббревиатура; Вербальный
символ: инициал; Графический символ: ориентированный на продукт; Графический символ: метафорический;
Знак: образный; Знак: цветной; Эмблема: персональная; Эмблема: общественная). Смысл использования
симметричной матрицы заключается в том, что многие торговые знаки попадают в более чем один
класс (см. рис.~\ref{fig:wekerle}).

\begin{figure}
  \centering
  \includegraphics[width=.5\linewidth]{images/wekerle}
  \caption{Классификация Ганса Векерле: симметричная матрица 9 x 9}
  \label{fig:wekerle}
\end{figure}

В комментариях, редактор журнала Design Magazine писал: <<Since the mid-50s, there has
been a gradual, but often entirely arbitrary move toward the use of abstract symbols for company
recognition. These have almost always replaced figurative illustrations which, for all their limited
value as information, had a humanitarian realism. Many of the new symbols have lost true symbolism
in having nothing precise to say. Hans Weckerle’s classification has discovered a wide range of
symbol sub-elements. By defining the nature of individual symbols in a very precise way he makes
possible a more functional approach to both the design and use of company symbols, for different
kinds of organisation and different purposes. Until now, there has been no attempt to make an exact
classification in this field>>.

Как видим, авторы компилятивных сборников руководствуются главным образом интуицией и желанием найти
оптимальный функциональный принцип сведения всего многообразия логотипов к нескольким простым
доминантам, чтобы затем удобно расположить знаковые кластеры по правилу концентричности. Такие
классификации носят откровенно прикладной характер и ориентированы на то, чтобы быстро информировать
дизайнера, как любителя, так и профессионала, о положение дел в практике лого дизайна на данный
момент.

Более строгий, аналитический подход к классификации логотипов пока встречается крайне редко. По этой
причине любая попытка выйти за пределы наличной эмпирической данности и отрефлексировать накопленный
опыт всегда привлекает к себе особое внимание. Такой пионерской работой можно считать обстоятельное
исследование по истории торговых марок и графических символов П. Моллерапа, в которой он предлагает
семиотический взгляд на родовую классификацию знаков. Обобщив опыт предшественников и современников,
Моллерап подразделяет логотипы (торговые знаки), по признаку функциональности и суггестивной
действенности дизайна, определяя их материальные (то, что торговые знаки изображают) и референтные
(то, что торговые знаки означают) качества. Поговорим об этом чуть подробнее.

Любая классификация, согласно Моллерапу\autocite[][98]{mollerup1999marks}, чтобы соответствовать
своему целевому назначению, должна отвечать следующим критериям:
\begin{enumerate}
\item Идеальная классификация состоит из отдельных групп, между которыми есть чёткая
  граница. Классификация любого объекта должна быть понятна.
\item Критерии, согласно которым строится классификация, должны применяться одновременно. Каждый
  определяющий шаг классификации должен разделять объекты по какому-то одному единому принципу.
\item Выделенные группы одного уровня не должны пересекаться, то есть ни один объект не должен
  принадлежать более чем одной группе.
\item Все группы одного уровня должны охватывать всю совокупность объектов. Любой объект должен
  принадлежать какой-то группе.
\item Формирование групп должно быть обосновано с точки зрения цели построения классификации.
\end{enumerate}

Несмотря не очевидные достоинства такого продуманного подхода, на практике, таксономия, построенная
по этим критериям, как выясняется, объективно не может им следовать до конца. Трудности при
проведении процедур отбора и упорядочения возникают сразу по целому ряду направлений. Первое. Не
всегда возможно провести разделение между группами, т.е. первое и третье правила классификации
вынужденно нарушаются, чтобы выполнялось пятое. Второе. Невозможно установить однозначные границы
между всеми группами одного уровня. Соответственно, заявленные свойства логотипов далеко не всегда
ясны и эксплицитно выражены, а периодически, просто количественны. Третье. Буквы логотипа могут быть
настолько особенными по своей форме, что почти никто не узнаёт в них сами буквы. Объяснение, которое
придаёт смысл логотипу, может быть почти забыто, и для большинства пользователей логотип будет
неявным и даже нереалистичным. Однако свойство используются согласованно, и на каждом шаге
происходит деление только по одному признаку. Четвертое. Группы одного уровня могут
пересекаться. Логотип может содержать одну или несколько букв и картинку, поэтому может и
принадлежать сразу двум группам одного уровня одновременно. Наконец, возможны и другие комбинации
групп. Название может содержать и имя собственное, и обычное безличное имя существительное. Иными
словами, по каждому отдельно взятому свойству таксономия действительно позволяет чётко провести
разделение и упорядочение групп. В то время как на практике свойства, оправдывающие формирование
конкретной группы, зачастую не только не взаимоисключают друг друга, но и проявляются
одновременно. Один логотип может обладать свойствами, присущими двум или более группам.

Тем не менее, в совокупности группы одного уровня действительно покрывают все возможные логотипы
данной категории. Например, если логотип не является графическим, то он и не графический, и т
д. Другим достоинством классификации Моллерапа является тот факт, что она строится на логотипе как
таковом. На его материальных свойствах, на его связи с предметом обозначения, на его референтных
признаках, стилистической пластичности и способности превращения в аллюзию. На самом деле:
материальные свойства логотипа, скорее всего, примерно одинаково воспринимаются большинством
потребителей. При этом культура и культурный контекст, в котором потребитель встречает логотип,
могут значительно варьироваться, что влечет за собой вариативность интерпретаций и, тем самым,
вариативность приписываемых ему референтных качеств.

Из сказанного следует, что один и тот же логотип может быть классифицирован по-разному, и,
следовательно, все классификации логотипа в этой таксономии -- это лишь <<возможные>>
классификации. Номинально исходная классификация Моллерапа представлена двумя семиотическими
категориями (свойства), восемью критериями разделения, семью промежуточными и тридцатью конечными
группами таксономии (см. таблицу~\ref{tab:mollerup1}). Образно она показана в виде дерева
(см. таблицу~\ref{tab:mollerup2}) Корни: \emph{summum genus} -- группа, из которой вырастают надземная
часть дерева. Тринадцать ветвей вправо -- \emph{infima species} -- итоговые группы. В промежутке
находятся \emph{subaltern genera} -- промежуточные группы. Далее, акронимы, сокращения по начальным
и не начальным буквам могут быть дальше разделены как названия, на собственные имена, описательные
имена, метафоры, неочевидные имена и искусственные имена, но это, видимо, слишком. Большая часть
сокращений -- это нарицательные имена. (Подробнее см. \autocite[][98-123]{mollerup1999marks})
Остановимся здесь.

Итак:
\begin{enumerate}
\item Мы сделали краткий вводный обзор подходов к классификации логотипов по трем направлениям:
  официально-юридическому, неформально-эмпирическому и популярно-семиотическому.
\item Мы постарались показать сходство между этими подходами и проиллюстрировать различия.
\item Основная задача нашего изложения в этой части сводилась к тому, чтобы показать масштабность
  проблемы создания классификации логотипов вообще и создания единой объективной классификации в
  частности.
\item Важное здесь -- неизбежно эвристический характер создаваемых классификаций, их <<возможный>>
  статус. С одной стороны, это объясняется социокультурной обусловленностью чтения логотипов в
  практике жизни. С другой, спецификой деятельности лого дизайнеров, суть которой -- постоянный поиск
  новых идей.
\end{enumerate}

Одним источником вдохновения для лого дизайнеров являются сборники-классификации логотипов,
составленных самими дизайнерами. К разбору одного такого сборника мы и переходим.

\subsubsection{Авторские сборники-классификации}

%% TODO: и ещё одна путаница с веками.
На сегодняшний день одним из самых авторитетных сборников-классификаций логотипов в профессиональной
среде принято считать два тома визуальной хроники немецкой торговой марки <<A Treasury of German
Trademarks>> (1982, 1985), собранных американским шрифтологом и лого дизайнером Лесли
Кабаргой. Книги представляют собой себя собрание визуальных символов, логотипов, импринтов, обширную
картотеку торговых знаков, созданных в Германии во второй половине XIX в. -- первой половине XX
в. Первый сборник охватывает период с 1850-1925 гг., второй -- 1900-1950 гг.

Думается, выбор Германии в качестве объекта исследования в данном случае вполне закономерен,
учитывая то влияние, которое он оказал на мировую практику лого дизайна в ХХ веке. В совокупности в
сборниках представлены 600 знаков, созданных в Германии в указанный период. Многие логотипы показаны
в двух цветах, как они и были первоначально созданы. Ценность такой подборки, помимо сугубо
исторического значения, состоит и в том, что немецкие логотипы того времени по-прежнему продолжают
жить в современном лого дизайне в виде многочисленных заимствований, цитат, аллюзий, подражаний и
просто как пример красивых безупречно выполненных форм.

В содержательном плане классификация Кабарга включает в себя главным образом личные
идентификационные знаки известных дизайнеров или богатых влиятельных фирм, ценивших стильный дизайн
и имеющих достаточно средств на оплату услуг признанных немецких мастеров в этой области. Таких как
Карл Шульпиг, Джозеф Биндер, Конрад Йоахим, Ф.Х. Эмке, Людвиг Хольвайн, Люциан Бернхард, Вильгельм
Деффке и Карл Эрнст Хинкефус, Карлом Шульпиг, Рудольф Кох, Вальтер Керстнг, О. Г. В. Хаданк, Филипп
Зайц и др. Изобретательность, находчивость и остроумие именно этих дизайнеров Л. Кабарга кладет в
основу своей классификации. С другой стороны, именно эти качества побуждают современных лого
дизайнеров в обязательном порядке приобретать личную копию двухтомника, чтобы иметь ее всегда под
рукой.

Структурно материал сборника организован не по абстрактным категориям, как это обычно принято делать
в классификациях, а по персоналиям и тенденциям. Мы также последуем этому принципу ниже. Но для
начала выборочно представим наиболее ярких представителей немецкой школы лого дизайна.

\begin{enumerate}
\item \textbf{Фриц Хельмут Эмке} (1878--1965 гг.). Кроме создания логотипов, он, как и многие
  немецкие дизайнеры в те времена, занимался архитектурой, дизайном интерьера, плакатов, иллюстрации
  и разработкой шрифтов. Эмке даже придумал своё собственное слово для своих логотипов --
  <<kennbilder>>, которое приблизительно можно перевести, как мгновенно узнаваемая картинка; простой
  дизайн. Тот, который может рассказать историю при одном взгляде на него и при этом находить
  эмоциональный отклик у зрителя. Собственные дизайнерские находки Эмке, в свою очередь, были ни чем
  иным как откликом на орнаменталистику арт-нуво (рис.~\ref{fig:trademarks:german:emke}).
\item \textbf{Вильгельм Деффке} (1887--1950 гг.), \textbf{Карл Эрнст Хинкефус} (1881--1970 гг.). Оба
  дизайнера входили в ударную группу известной в Германии студии дизайна <<Вильгельмверк>> (1916--1920
  гг.), в начале столетия позиционировавших себя как наследников традиций старых немецких
  мастеров. Фирменный «черный» стиль немецкого лого дизайна сегодня -- наследие творческого успеха
  <<Вильгельмверк>> тех времен (см. рис.~\ref{fig:trademarks:german:dh}. В то же время, по
  исторической иронии, дизайнерская деятельность этой жеd студии косвенно ответственна за перерождение
  древнего символа жизненной энергии в нацистскую эмблему немецкого Рейха.
\item \textbf{Карл Шульпиг} (1884--1948 гг.) Один из наиболее часто копируемых немецких
  дизайнеров. Характерная черта его стиля -- органичное сочетание метафорического изображения товара
  с монограммной графикой в вербальной части знака. Возникающий эффект может носить легкий
  иронический характер, равно как и содержать ненавязчивый медитативный подтекст. Тем не менее,
  понять суть его подхода к дизайну непросто. Внешне неброский, стиль Шульпига требует даже от
  имитатора вдумчивого, вникающего отношения (рис.~\ref{fig:trademarks:german:shulpig}).
\end{enumerate}

%% TODO: и опять век!
Как мы уже отмечали, немецкие логотипы оказала большое влияние на визуальный язык XX века в целом. В
отличие, скажем, от Швейцарии, страны, известной своим добротным <<еловым>> дизайном, Германия начала
ХХ века была экспериментальной площадкой для поиска новых нестандартных форм визуальной экспрессии,
принципов композиции и технологии производства. В этом смысле, можно было бы сказать, что самые
талантливые немецкие лого-дизайнеры стали частью той широкой и неоднородной традиции, которая
позднее получила название модернизма, провозгласившего разрыв с предшествующим историческим опытом
художественного творчества. Стандартизация технологии производства, базовые принципы, такие как
<<принцип асимметрии, свободы от орнамента, ясности, соответствия духу времени, типографической
самодостаточности (``типографика – не живопись!'') остаются и сегодня в силе>>.\autocite[][8]{chihold2011}

Эстетический протест немецких графиков модернистов был направлен в первую очередь против засилья
тяжеловесных и вычурных форм уходящей эпохи, форм, трудных для быстрого прочтения и не
соответствовавших стремительному духу нового времени. Прежние приемы типографики ориентировались на
медленное, осмысленное прочтение сообщения. Соответственно, вопросы формы и эстетики при разработке
дизайна (выбор шрифта, орнаментов, сочетания шрифтов) имели приоритет перед утилитарной
функциональностью. Так, например, немецкие логотипы XIV-ХVII вв. типично включали в себя элементы
античной мифологии или геральдики -- королевские животные, щиты, мечи и различные комбинации простых
геометрических форм. Подразумевалось, что символическое наполнение логотипа по аналогии
воспроизводит символическое содержание ритуализованных знаков статуса и престижа. Как правило, они
были иллюстративны, сюжетны и описывали некое действо. Для примера: в ХVII в Германии большой
популярностью пользовались массивные наружные вывески, содержащие реалистичное изображение. Так,
вывеска с именем купца -- Churchill дополнялась изображением церкви и поля, формируя расширенный
ассоциативный ряд для создания символического контекста восприятия знака. В тоже время, в городской
среде широкое распространение получили знаки-вывески прикладного, профессионально-ремесленного
назначения, представлявшие собой простые, стилизованные изображения какого-то ремесла или рода
занятий: очки -- врач-окулист, наковальня -- мастер-кузнец.

Напротив, главной стилеобразующей особенностью немецкого логотипа эпохи модернизма можно считать
выразительную лаконичность его формы и использование древних символов – молнии, креста, стрелы,
животного, птицы. Более того, в начале XX в. немецкая промышленость, ориентированная на массовое
производство товаров, придавала особое значение маркировке своей продукции посредством
логотипа. Часто такой знак был единственным визуальным элементом на упаковке и в рекламном
объявлении. Массовое производство, как отмечалось ранее, порождает массовый спрос, равно как и
создает условия для возникновения массового спроса на услуги лого-дизайнера и способствует
формированию массовых профессиональных сообществ дизайнеров, как это и случилось в Германии --
<<Баухаус>>, <<Тотальный дизайн>>, <<Метадизайн>>. Для сравнения, американские логотипы того времени в
основном состояли из названий компаний написанных от руки или же стилизованных изображений
комических персонажей, таких как The Old Dutch Maid, Aunt Jemima, Rastus и
Mr. Peanut. \autocite{link:mpr} Лого-дизайн как самостоятельный вид профессиональной деятельности
оформился в США гораздо позже, следуя заразительному примеру инновационного немецкого дизайна и
благодаря регулярным публикациям лучших работ немецких дизайнеров в авторских монографиях и
специализированных журналах, таких как <<Graphis>>, <<Gebrauchsgraphik>>, <<The Studio>>, <<Modern
Publicity>>. Возьмем для примера монографию типографа, дизайнера и педагога Яна Чихольда <<Новая
типографика>>\autocite{chihold2011}, в которой изложены базовые принципы модернистской типографики и
описан функциональный подход к оформлению печатных изданий.

Современный логотип, по Чихольду, должен быть простым и самобытным. Он должен легко запоминаться, то
есть быть узнаваемым и заметным. Как типограф, он убеждён, что хороший логотип совсем не обязательно
рисовать, поскольку эффективный товарный знак можно сделать и типографическим способом. Такой
логотип будет по определению функционален, и <<как и все вещи, выполненные при помощи техники, такой
знак несёт в себе заряд энергии>>\autocite[][118]{chihold2011}. И далее: <<Однако надо серьезно
предостеречь их от того, чтобы называть логотипом простую комбинацию литер из наборной кассы
(монограмму). Логотип -- это нечто совершенно иное, он вмещает в себя гораздо больше смысла, чем
простая монограмма, а плохой логотип гораздо хуже, чем его отсутствие. Сейчас типографы часто
соблазняются возможностью вместо прежних типографских виньеток печатать типомонограммы. Но в рекламе
нужны только логотипы; в наше время монограммы там бессмысленны. Монограмма в качестве фирменного
знака всегда хуже, чем логотип. А плохой логотип может испортить всю рекламную кампанию>>.
\autocite[][119]{chihold2011}

В виде программных требований -- своего рода манифест современного лого-дизайна -- эта позиция
оформилась в 1929 году: <<1. Выразительность. Логотип должен выражать качество
продукта. 2. Простота. Логотип должен быть простой, потому что её необходимо будет воспроизводить в
любых размерах и в любой среде. 3. Оригинальность. Логотип не должен быть похожа на другие
трейдмарки. 4. Вневременность. Он не должна возникать из просто модных поветрий. 5. Правильный знак
– основа эффективной рекламы компани>>. \autocite{cabarga1982treasury}

Обращает на себя внимание тот факт, что процитированные слова, сказанные почти столетие тому назад,
по-прежнему звучат свежо и актуально. Представления немецких дизайнеров-модернистов тогда
практически не отличаются от ожиданий и требований к логотипу сегодня. В каком-то смысле, быть
современным всегда -- был, есть и будет главным принципом модернистского мировоззрения. И если
модернизм, в широком смысле, есть <<другое искусство>>, главной целью которого является создание
оригинальных произведений, то лого дизайн в своей модернистской версии -- это <<другой дизайн>>,
базирующийся на внутренней свободе, остраненном видении мира и новых выразительных средствах
изобразительного языка. Само слово \emph{дизайн} означает -- проект, план, замысел. В свою очередь
проект (projectum), следуя логике латинского языка, и вовсе означает -- брошенный вперед, что само
по себе роднит немецкий модернистский дизайн с современным ему футуризмом. Футуристы, как известно,
придумывали не только новые слова, профессиональный жаргон, но и язык афиш, и язык плакатов -- язык
рекламы. <<Реклама--– имя вещи\ldots Думайте о рекламе!>>, писал
В.В. Маяковский.\autocite{mayakovsky1959}

Функционализм лого дизайна, т.е. что утилитарно, удобно, то и красиво, естественным образом вытекает
из манифеста художественного объединения Баухаус (1919--1933 гг.), выпущенного в 1919 г. В манифесте, в
частности, провозглашалось равенство между прикладными и изящными искусствами, декларировалось
программное намерение повысить качество выпускаемой продукции в целях удовлетворения массовых
потребностей населения. Для этого промышленные товары нужно сделать красивыми, доступными по цене и
максимально удобными. Иными словами, функциональными.

Другой вопрос, конечно, что функциональная красивость предметов массового потребления, предлагаемых
дизайнерами-модернистами, все-таки не имманентна самому предмету. По сути дела, она свидетельствует
в первую очередь о вкусе и художественном чутье дизайнера.

%% TODO: и опять век!
Таким образом, европейский модернизм в первых двух десятилетиях ХХ века, с его энергичным
стремлением утвердить новые нетрадиционные начала в художественной практике, пропагандой
непрерывного обновления художественных форм и условности (схематизации) стиля, можно назвать первым
значимым источником вдохновения и ориентиром для практики авангардного лого дизайна сегодня. Вторым
важным источником следует назвать <<модерн>> (ар-нуво, югендстиль и т. д.), другое масштабное
направление в европейском искусстве на рубеже ХIХ-XX веков. Модерну свойственно стремление к
сочетанию художественных и утилитарных функций в создаваемых объектах, вовлечение в сферу
прекрасного всех областей деятельности человека, живой интерес к технологиям, отказ от прямых линий
и углов в пользу более естественных, <<природных линий>>. Расцвет прикладных искусств -- еще одна черта
модерна. В том числе и расцвет лого дизайна, как мы отметили ранее, разбирая примеры из сборника
Л. Кабарга.

Таким образом, в самом начале ХХ века логотип оформляется и закрепляется в культуре в виде
концептуального понятия. В современном лого дизайне можно найти достаточно примеров влияний, как
идеологии модернизма, так и философии модерна. В ХХI веке произошли существенные изменения в
техническом решении логотипов. Если раньше логотипы сначала чеканились и рисовались, затем
печаталась, то теперь их дизайн целиком и полностью зависит от того, какое программное обеспечение
для работы с графикой появляется на рынке компьютерных программ. Персональный компьютер как
технология самым решительным образом перестраивают практику графического дизайна и формируют новую
визуальную эстетику. Трудоёмкие зрительные эффекты, оптические иллюзии и сложные градиенты теперь
создаются щелчком мыши, быстро и точно.

Логотипы сегодня изготавливаются серийно и с учетом многочисленных перспективных
носителей. Соответственно повышаются требования к их полифункциональным возможностям. Так, они
должны хорошо читаться в любом диапазоне: в максимально крупных и минимально мелких размерах, любых
форматах и расширениях, включая иконки меню мобильного телефона, URL строки на веб-сайте. Это
касается в равной степени и логотипов в телевизионных программах и широкоформатных кинофильмах, а
также логотипов на любом печатном носителе. В отдельных случаях можно проследить, как
провозглашенная немецкими модернистами лаконичность визуального знака явно вытесняется стремлением к
интенсивной визуальной красивости модерна, наглядно иллюстрируя принцип единства и борьбы
противоположностей. Вместе с тем, как отмечает один из ведущих американских дизайнеров-модернистов
Р. Рэнд: <<You can’t criticize geometry. It’s never wrong.>> Действительно, канонические логотипы NBC,
CBS, PBS, Mobil и др., созданные дизайн студией <<Chermayeff \& Geismar>> еще в середине ХХ века не
подверглись существенным изменениям. Сегодня мы видим их точно такими же, какими они были 50 и более
лет назад.

Подведем черту.
\begin{enumerate}
\item В этом разделе мы вели речь о традиции классификационных сборников логотипов, составляемых
  практиками лого дизайна. Такие сборники, как правило, носят историографический характер и
  выполняют функции профессионального справочника, учебного пособия, документальной хроники
  становления традиции.
\item В качестве примера мы выбрали двухтомное издание американского дизайнера Л. Кабарга, в котором
  задокументирована почти столетняя история немецкой торговой марки (логотипа).
\item Помимо богатого иллюстративного материала, сборники Л. Кабарга содержат ценный
  культурно-исторический комментарий о деятельности немецких лого дизайнеров в контексте своего
  времени -- влияниях, заимствованиях, примерах сотрудничества и кооперации.
\item Мы говорили преимущественно о периоде на рубеже двух предыдущих веков, условно совпадающем с
  функционированием двух несовпадающих художественных направлений в искусстве -- модернизм и модерн.
\item Мы попытались показать, во-первых, как немецкие лого дизайнеры в начале ХХ века активно
  использовали визуальный язык, художественные принципы и приемы работы, характерные для того и
  другого направления. Во-вторых, мы предложили рассматривать современные логотипы, с одной стороны,
  как продолжение сформировавшейся традиции, и как ее преодоление, с другой.
\end{enumerate}

Теперь рассмотрим вопрос о классификации логотипов по темам и мотивам.

\subsubsection{Культурно-историческая классификация}
