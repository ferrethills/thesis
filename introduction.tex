\section{Введение}

\textbf{Актуальность исследования} определяется необходимостью дифференцированного изучения эмблематической природы логотипа в культурном пространстве современности. Сознание человека информационного общества подвержено активному воздействию различных знаков. Изучение информации, заложенной в эти знаки, поможет глубже понять экзистенциальную ситуацию современного человека и то время, в котором он живёт. Если классическая культурная традиция, по определению постмодернистских философов, логоцентрична (Деррида, Фуко), то современная культура скорее логотипична, чем логоцентрична. Традиционные культурные знаки, символы и эмблемы все чаще переформулируются в медийном дискурсе в терминах коммерческой деятельности и называются логотипами. Национальный флаг, герб суть логотипы. Есть общепринятые эмблемы-логотипы стран и городов, как, например, «большое яблоко» Нью-Йорка или «сапожок» Италии. А логотип-слоган «I love NY», разработанный в середине 1970-х американским графическим дизайнером Милтоном Глейзером, является одной из самых популярных коммерческих икон современности. Другими словами, логотип перешёл из сферы товарных взаимоотношений в сферу культуры в целом. Логотипы окружают нас в цифровом и физическом пространстве, однако, в повседневной жизни мы редко задумываемся о культурной составляющей логотипа. Логотип в культурно-исторической перспективе – это значимый артефакт, поскольку он документирует человеческую историю, историю вещей, историю потребления.
Актуальность данного исследования определяется следующими \textbf{противоречиями:}
\begin{itemize}
\item между недостаточной изученностью процессов визуализации культуры и понимания суггестивных механизмов воздействия логотипов, формирующих и отражающих современное сознание;
\item между большим количеством ненаучных подходов к пониманию сущности логотипа и недостаточной разработанностью научно-методической базы;
\item между массовыми процессами брендирования в современной культуре и утратой личностного начала в человеке.
\end{itemize}
Выявленные противоречия дают основание сформулировать \textbf{научную проблему} исследования: какие стороны в современной культуре отражает и формирует логотип.

\textbf{Степень научной разработанности проблемы}

Среди источников по исследуемой теме – работы культурологов, философов, историков, семиотиков, психологов и др., изучающих концепты «массовое сознание», «массовое общество», «массовая культура» и др.

Так, в семиотической области диссертант опирался на труды Р. Барта, Ж. Бодрийяра, М.К. Голованивской, Е. Горного, Е. Григорьевой, Э. Кассирера, Ю.М. Лотмана, Ч. Морриса, Ч. Пирса, А.Б. Соломоника, Ф. де Соссюра, Б.А. Успенского, Л.Ф. Чертова, У. Эко.

Культурологический аспект находит своё отражение в работах, посвященных проблемному полю массовой культуры А.Я. Гуревича, И.А. Едошиной, В.С. Елистратова, И.В. Кондакова, М.И. Найдорфа, К.Э. Разлогова, Н.В. Серова, Е.Г. Соколова, Г.Л. Тульчинского, Р. Уильямса, Й. Хёйзинги.

Важную роль сыграли работы французской историографической школы «Анналов»: М. Блока, Ф. Броделя, Ж. Ле Гоффа, Л. Февра.

При описании теоретических исследований современного общества учитывались работы социологов: Х. Арендт, Д. Белла, Г. Блумера, М. Вебера, Г. Зиммеля, У. Корнхаузера, Ю.А. Левады, Э. Ледерера, Д. Макдональда, Л. Мамфорда, К. Маннгейма, Г. Маркузе, Х. Ортеги-и-Гассета, Б.Ф. Поршнева, Д. Рисмена, Г. Тарда, А. Тоффлера, Э. Фромма, Э. Шилза, Ф. Юнгера и других.

Философский аспект представлен работами М.М. Бахтина, Г. Гегеля, А. де Боттона, А.Н. Ильина, Л.Г. Ионина, А.Ф. Лосева, М. Маклюэна, М.К. Мамардашвили, П.А. Флоренского, М. Фуко, Р.У. Эмерсона и других.

Психологический подход представлен работами В.М. Бехтерева, Л.И. Божович, Р. Броуди, П.Я. Гальперина, Л.С. Выготского, А.А. Леонтьева, А.Н. Леонтьева, Г. Лебона, А.Р. Лурии, Д.В. Ольшанского, В.С. Рамачандрана, З. Фрейда, Д.А. Эльконина, К.Г. Юнга и других.

Из известных авторов, пишущих о логотипе как о предмете научного анализа, можно выделить Ж.-Л. Азизола, Э. Амиот, Д. Боуи, К. Дж. Веркмана, Б. Гарднера, Л. Кабарга, М. Макнаб, П. Моллерапа, С. И. Серова, Б. Эльбрюнна и ряд других. Феномен логотипа, в силу своего весьма специального характера, связан либо с решением инструментальных задач графического дизайна (Ю. Гордон, Я. Чихольд), либо с изучением логотипа в рамках экономических наук (С. Анхольт, К. Динни, Н. Кляйн, Ф. Котлер, М. Линдстром, А. Уиллер и др.). При этом культурологическое осмысление логотипа пока остаётся вне поля внимания специалистов.

Указанные исследования в общей сложности составляют солидную теоретическую базу, позволяющую системно изучить логотип как форму массового сознания, его структуру и социокультурное назначение.

В качестве \textbf {объекта} исследования выступает логотип как инструмент формирования и отражения массового сознания.

\textbf {Предметом исследования} является логотип как форма современного массового сознания, его структура, функции и отличающуюся эмблематической природой.

\textbf {Целью исследования} является представление логотипа как формы современного массового сознания, обладающей структурой, имеющей определённые функции и отличающейся эмблематической природой.

Данная цель реализуется в ряде основных \textbf {задач}:
\begin{enumerate}
\item На основе изучения ментальности массового общества выявить её культурно\hyp{}семиотические параметры.
\item Представить сознание человека массы как объект знаковой манипуляции.
\item Обозначить культурно-значимые тенденции в современном лого дизайне.
\item Используя классификационный подход, представить логотип как культурный артефакт, отражающий сознание массового человека.
\item Определить эмблематическую сущность логотипа и выявить суггестивные механизмы воздействия логотипов на современное массовое сознание.
\end{enumerate}

\textbf{Хронологические рамки} исследования определяются концом XX – началом XXI в.

\textbf{Материалом исследования} послужили труды отечественных и зарубежных специалистов в области визуальной культуры: культурологов, историков, семиотиков, социологов, психологов, маркетологов, рекламистов; словари и справочники; классификации-сборники логотипов.

Выбор \textbf{методологии} обусловлен намеченными задачами и поставленной целью. В основе исследования применён \emph{структурно-функциональный} метод, который определён знаковой природой логотипа и спецификой современного сознания, часто опирающегося на визуализацию. Этот метод позволил рассмотреть не только практическое решение и разработку логотипов в графическом дизайне, но прежде всего – внутреннее смысловое содержание логотипа как знака современной визуальной культуры.

Кроме того, в диссертационном исследовании задействован \emph{дескриптивный метод}, позволяющий не только дифференцировать логотип как культурный феномен, но и представить его как некое культурное единство на основе понятий «массовое сознание», «ментальность», «массовое общество», «массовая культура», «знак», «логотип», «бренд».

Для истолкования фактов практики лого дизайна в исторической перспективе применён \emph {сравнительно-исторический} метод.

Для получения актуальной информации по проблемным вопросом исследования в диссертации использован метод \emph{интервью} (Было проведено интервьюирование современных специалистов в сфере лого дизайна).

В работе используются также и традиционные методы теоретического исследования: \emph{анализ}, \emph{синтез}, \emph{индукция}, \emph{дедукция}, \emph{реферирование}, \emph{научная классификация}.

\textbf{Гипотеза исследования}
\begin{enumerate}
\item Логотип является формой иконической репрезентации современного массового сознания и отражением его ментальности.
\item Массовое сознание отличается от других форм общественного сознания специфическими свойствами его носителя, т.е. массы. В частности – массы потребителей. Логотип, будучи формой массового сознания, с одной стороны, отражает его основные социально-психологические свойства. К таковым относятся визуальная зависимость, мозаичность, эмоциональность, заразительность, изменчивость и подвижность. С другой стороны, он выполняет функцию управления содержанием сознания массового потребителя и его потребительской активностью.
\item Структурно логотип представляет собой соединение визуальных и вербальных составляющих, образующих единый смысловой комплекс, подобно эмблеме.
\item Эмблематическая сущность логотипа заключается в выполнении им мнемонической функции. Иначе говоря, логотип представляет собой готовый блок памяти, внедряемый в массовое сознание. Основная задача такого блока состоит в том, чтобы, во-первых, заложить в долговременную память симулятивные образы и установки на потребление. Во-вторых, обеспечить их быстрое извлечение из памяти в каждой конкретной ситуации потребления.
\item Высшей точкой семантической эволюции логотипа в сознании массового человека-потребителя следует считать бренд, под которым понимается статусный символ-икона современного массового общества.
\end{enumerate}

\textbf{Научная новизна исследования}
\begin{enumerate}
\item Логотип впервые представлен как форма современного массового сознания, определены его функции, выявлена эмблематическая природа.
\item Предпринята попытка комплексного культурологического анализа логотипа в массовой культуре. Именно благодаря культурологическому подходу удалось синтезировать и творчески использовать теоретические исследования из семиотики, психологии, философии, социологии, маркетинга и других областей знания. Разработанный базис может быть использован для дальнейшего более подробного и более специализированного изучения феномена логотипа.
\item Выявлены отдельные культурно значимые тенденции современного массового общества, например, массовое брендирование. Если логотип является визуальным знаком потребительской ориентации в массовой культуре, то бренд можно считать знаком-эталоном качества, престижа и статуса в сознании массового потребителя. Бренды формируют особый культурный код или язык в современной массовой культуре. В частности, интерес представляют процессы территориального брендирования, под которыми понимается массовое переназывание и символическое визуальное переосмысление устоявшихся знаковых обозначений традиционной культуры и национальной идентичности. Другой тенденцией в эволюции современного массового сознания является практика индивидуального самобрендирования по образцу формирования корпоративной идентичности в сфере экономической деятельности. Визуальная самоидентификация через обладание персональным лого становится неотъемлемой частью самобрендирования.
\item Доказано, что в сфере потребления по своему замыслу и основному символическому назначению логотип, подобно эмблеме, является свернутым мнемоническим блоком, призванным, с одной стороны, транслировать уникальный ценностный статус компании-производителя и ее продукции, а, с другой стороны, внушать целевому массовому потребителю ощущение эмоциональной сопричастности некоему симулятивному таинству или псевдо-мистическому опыту, будь то коммерческая утилизация древних символов или маркетинговая мифологизация нового технологического чуда.
\end{enumerate}

\textbf{Теоретическая значимость} исследования состоит в том, что его результаты синтетически обобщают имеющиеся знания о логотипе и вносят вклад в создание теории логотипа на культурной платформе. Разработаны методологические подходы к уточнению форм бытийствования массового сознания. Скорректировано понимание знаковой природы современного массового сознания. Знание форм иконической репрезентации массового сознания позволяет расширить понятийную базу культурологических исследований и углубить понимание процессов и феноменов массовой культуры. Помимо этого, такой подход позволяет выявить интегративную природу культурологической рефлексии, уточнить некоторые характеристики культурологического метода, в частности, как это показано на примере массовизации управляемых процессов по формированию бренд-идентичности.

\textbf{Практическая значимость} заключается в выявлении форм массового сознания и механизмов влияния на него. Знание этих механизмов может быть использовано в прикладной культурологии, в частности, в семиотическом маркетинге, принципах разработки брендов. Полученные данные также могут быть применены в курсах по культурологии, истории культур, межкультурной коммуникации, семиотике, а также в рамках курса графического дизайна и брендинга.

\textbf{Личный вклад} автора работы заключается в том, что впервые предпринята попытка культурологического исследования логотипа как формы массового сознания и феномена массовой культуры. Выявлена эмблематическая сущность логотипа, определены его функции и структура.

Комплексное изучение логотипа представляет собой анализ следующих аспектов:
\begin{enumerate}
\item Психологический аспект заключается в учёте особенностей человеческого восприятия образа, формы, цвета, композиции и т.д.
\item Семиотический аспект выражается в том, что эффективность воздействия знака может быть усилена через использование культурно насыщенных знаковых комплексов и символов.
<<<<<<< HEAD
\item Культурологический аспект представлен эвристическим анализом культурно\hyp{}значимых контекстов, таких как традиционные вкусовые предпочтения массового потребителя с учётом социокультурного признака, национальных и временных особенностей. Историко-культурологические классификации, представленные в работе, наглядно демонстрируют, как логотип отображает ментальность, потребности и нравы массового человека в определённую временную эпоху. Анализ тенденций в лого дизайне позволяет увидеть, насколько быстро меняются предпочтения потребителя, его стремления и чаяния. Подобно сущностным характеристикам массового сознания, эти тенденции не регламентируемы и носят спонтанный, диффузный характер. 
=======
\item Культурологический аспект представлен эвристическим анализом культурно-значимых контекстов, таких как традиционные вкусовые предпочтения массового потребителя с учётом социокультурного признака, национальных и временных особенностей. Историко-культурологические классификации, представленные в работе, наглядно демонстрируют, как логотип отображает ментальность, потребности и нравы массового человека в определённую временную эпоху. Анализ тенденций в лого дизайне позволяет увидеть, насколько быстро меняются предпочтения потребителя, его стремления и чаяния. Подобно сущностным характеристикам массового сознания, эти тенденции не регламентируемы и носят спонтанный, диффузный характер.
>>>>>>> 962a998a2e2b5294b14f601b334e9ed475c68b41
\end{enumerate}

\textbf{Соответствие диссертации паспорту научной специальности}

Работа соответствует специальности 24.00.01 «Теория и история культуры» и выполнена в соответствии со следующими пунктами паспорта специальности ВАК: 1.21 – традиционная, массовая и элитарная культура; 1.23 – личность и культура; 1.24 – культура и коммуникация; 3.21– социокультурные последствия коммерциализации культуры; 3.6 – «массовая культура» как социальный феномен.

\textbf{На защиту выносятся следующие положения:}
\begin{enumerate}
\item Логотип есть форма современного массового сознания. По своей сути логотип ориентирован на массовое сознание и потребление. Он не только отражает или выражает массовое сознание, но и формирует его.
\item Структурно логотип есть иконический знак, основанный на синтезе вербальных и невербальных элементов. По способности обобщения логотип является эвристическим знаком-символом, формирующим позитивный имидж компании. По способности отражения сегментов реальности логотип конвенционален и условен. По функциональному назначению логотип есть коммерческий знак-образ.
\item Помимо своих основных функций (идентификационная, рекламная, информационная), логотип выполняет ещё и идеологическую функцию, транслируя массовые общественные ценности и ориентиры. Основная смысловая нагрузка, возлагаемая на логотип в современной культуре и массовом сознании – это придание высокого статуса компании производителю и ее продукции, с одной стороны, и потребителю продукции компании, с другой.
\item Логотип предполагает манипуляцию массовым сознанием с использованием внешних и внутренних форм суггестии. При этом внешние формы возможны благодаря повсеместному использованию логотипов-эмблем на различных носителях, а внутренние основаны на приёмах аффектации и обращении к образам и паттернам коллективного бессознательного.
\item Логотип может быть уподоблен эмблеме по структурной сочетаемости изобразительных и вербальных элементов. Логотип, понимаемый как эмблематическое сочетание, подвержен энтропии, т.е. распаду фиксированного смысла, традиционно присущего эмблеме.
\item В культурно-исторической перспективе логотип является артефактом культуры, отражающим ментальность людей, их потребительские привычки и вкусы.
\end{enumerate}

\textbf{Апробация результатов} исследования

Основные результаты исследования обсуждались на занятиях по специальности, на заседаниях кафедры теории и истории культур Костромского государственного университета (2010–2013 гг.), апробировались в докладах и тезисах на международных, региональных и межвузовских научных конференциях: «Костромская земля: памятники культуры малых городов» (Кострома, апрель 2011); «Женские мотивы и образы в культуре русской провинции» (Кострома, март 2012); «Культура земли Костромской: традиции и современность» (Кострома, апрель 2012); «Актуальные проблемы современного российского общества: традиции и новации» (Кострома, ноябрь 2012); «Сапоговские штудии. Современное гуманитарное знание в России» (Кострома, ноябрь 2012); «Актуальные вопросы культурологии, истории и филологии» (Кострома, март 2012); «Пропилеи на Волге 2013. Кавказ и русская культура» (Кострома, май 2013). Основное содержание результатов исследования отражено в 10 публикациях автора.

\textbf{Структура диссертации}

Диссертационное исследование состоит из введения, двух глав, шести параграфов, заключения, библиографического списка использованных источников и литературы общим объемом 211 наименований, 10 приложений, включающих таблицы и рисунки. Содержание диссертации изложено на 180 страницах. Общий объём работы 256 страниц.
