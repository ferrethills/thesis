\section*{Заключение}

В кинофильме <<Новые времена>> (Modern Times, 1936) асоциальный герой Чарли Чаплина становится
фабричным рабочим Электросталелитейной компании, затягивающим болты на ленте быстродвижущегося
конвейера. Хозяин-капиталист наблюдает за производственным процессом с помощью скрытой камеры и
постоянно требует увеличения скорости линии. Чарли не поспевает за темпом и в нервном истощении
падает на гигантские шестерни. Рабочее название <<Новых времен>> -- <<Массы>> (Masses).  Одна из
основных тем фильма -- конфликт между человеком и массовым обществом индустриального
капитализма. Кроме того, в фильме выразительно представлены и совмещены главные символы
современности -- механизированная фабрика и крупный универмаг, в котором продаются товары массового
производства.

<<Новые времена>> в конце ХIХ и первой половине ХХ века означали склонность к социальной инженерии,
преклонение перед техникой,  ускорение темпов жизни и уменьшение расстояний.  Но прежде всего <<новые
времена>> означали выход <<масс>> на историческую сцену. Массовое производство, массовое общество,
массовая культура выросли из новых технологий, в то время как массовое сознание оформилось благодаря
появлению новых информационных средств. Массовое производство породило массовое потребление, которое
одновременно является и источником, и продуктом массовых форм их удовлетворения.  Но если массовое
производство есть процесс стандартизации предметов производства, запуск на конвейер одинаковых
вещей, то именно массовое потребление распространяет процесс стандартизации не только на
производство, но и на все сферы жизни, подводя под единый стандарт вкусы, привычки, поведение, образ
мысли потребителей, независимо от уровня образования, профессиональной занятости и т.д.

Процессы омассовления можно рассматривать двояко. С одной стороны, они приводят к усреднению,
нивелировке образа жизни всех слоев населения в обществе. С развитием технологий массовой информации
сегодня речь идет уже не просто о массовом читателе, слушателе или зрителе, а об универсальной
публике, повсеместно потребляющей одинаковую информацию, смотрящей одинаковые фильмы, слушающей
одинаковую  музыку и т.д. Традиционные границы в культуре  разрушаются, и сама культура превращается
в товар. Влияние переходит от индивидуальных вкусов к авторитету рынка. Масс-культурная модель
современного общества, по выражению Дж. Сибрука, есть модель <<ноубрау>> (Nobrow), в которой
коммерческая культура супермаркета -- многогранный источник статуса и престижа. В культуре
супермаркета, в отличие от вертикальной иерархической культуры, ценность определяется не столько
качеством, сколько аутентичностью, принадлежностью к доминантной субкультуре или культу. \autocite{sibruk2005}

Другой взгляд на глобальную массовизацию современной жизни подчеркивает ее демократический
характер. Согласно этой позиции, массовое производство и массовое потребление объективно говорят о
преодолении социального неравенства, об улучшению материального качества жизни членов общества, о
возвращении в аудиальный мир одновременных событий и всеобщего сознания. Энди Уорхол, основатель
поп-арта, суммирует это следующим образом: <<Ты смотришь телевизор и видишь кока колу, и ты знаешь,
что Президент пьет кока колу, Лиз Тейлор пьет кока колу и только подумай -- ты тоже можешь пить кока
колу. Кока кола есть кока кола, и ни за какие деньги ты не купишь кока колы лучше, чем та, что пьет
бродяга на углу. Все кока колы одинаковы, и все они хороши. Лиз Тейлор иэто знает, Президент это
знает, и ты это знаешь>>. \autocite[][]{warhol2002}  %% TODO: page?

Логотип есть универсальная письменная технология, визуальный код знаковой организации визуальных
элементов в целях выделения товара или услуги среди множества других конкурирующих товаров или услуг
на рынке. В замкнутой системе рекламной коммуникации логотип -- это инструмент стандартизации и
визуальное средство однотипной потребительской информации. Это также оригинальное изображение
наименования фирмы или направления ее деятельности. В функции идентификатора логотип может быть
максимально конкретным и предельно абстрактным по форме исполнения. В рекламной функции визуальной
репрезентации бренда логотип, будучи письменным знаком, абстрактно иллюзорен. Как и любая визуальная
технология он лишен элемента личной обращенности и может быть интерпретирован так или иначе, по
желанию. В то же время, наиболее существенной чертами прагматики логотипа являются его
воспроизводимость и повторяемость. Повторяемость, как известно, ведет к гипнотическому воздействию и
суггестии. В этом смысле, логотип способствует не столько продаже товара потребителю, сколько
продаже потребителя товару. Современный товар не становится лучше или проще, но, напротив,
регрессирует и опрощается потребитель, некритично смотрящий на мир как исключительно визуальное
пространство.

Изобретенные в середине XV столетия способы точного воспроизведения изобразительных сообщений с
помощью разборного типографического шрифта стали одним из важнейших орудий современной жизни и
мысли. Современный логотип -- этимологически  понимаемый как <<отпечаток слова>>  или текстовое
клише -- значимое развитие старой техники печатания. Повсеместность и тотальность его использования
в изображениях товарных знаков и статусных эмблемах  делают логотип знаковым социокультурным
явлением в визуально организованной жизни современного общества потребления. Тем самым логотип не
столько отражает или выражает массовое сознание, сколько его формирует.  В этом заключается его
эмблематическая сущность -- быть штамповым следом логоса. В легко узнаваемой однородности,
повторяемости и типичности заключается сила его массового воздействия.

Ч. Чаплин в интервью 1966 года рассуждает: <<Я не боюсь штампов, если они правдивы. Вся жизнь –
штамп. Мы все живем, и умираем, и едим три раза в день,  и влюбляемся, и разочаровываемся. (\ldots)
Люди, как говорится, делали это и раньше. Ну и что же из того? Если избегать штампов, то станешь
скучным>>. \autocite[][138]{chaplin2005}

В финальной сцене <<Новых времен>> мы видим парочку, бредущую вниз по опустевшей дороге после
очередного увольнения -- но рука об руку. <<Маленький человек>>  Чарли Чаплина снова оказывается в
трудных обстоятельствах, но он ничуть не обескуражен. Даже торжествует.
